\section{Weighted Descent Tree Automata}
The basis of the tree automata we want to present in this section is given in
\cite{RandAutoInfTrees}. There non-deterministic tree automata are equipped 
with an alternative semantic, namely the runs are equipped with a uniform 
probability measure at each branching point and the run is considered 
accepted if the measure of those paths that satisfy an associated condition
(e.g. Büchi- or Parity-condition) have a certain measure. As for \acp{PBA} this
allows for almost-sure or positive acceptance of runs. In the following we 
follow suggested generalizations of \cite{RandAutoInfTrees} by designing 
transitions more expressive by adding individual probability measures for them.
Furthermore, we incorporate the concept of \emph{alternation} by allowing to 
send different states on the same path through the tree with individual weight.
Formally, we define in analogy to clause-functions a notion of a weighted 
clause function, i.e.
\begin{definition}[Generator]
  For a set of states $Q$ and a set of directions $D$ we call a probability
  function on $Q\times D$, i.e. $G: Q\times D\rightarrow [0,1]$ with
  \begin{equation*}
    \sum\limits_{q\in Q,d\in D}G(q,d) = 1,
  \end{equation*} 
  a generator over $Q$ and $D$. 
\end{definition}
These generators are used to substitute clause-functions in alternating tree
automata and we obtain the following tree automata:
\begin{definition}[Weighted Descent Tree Automata]
  We define a \acl{WDTA} as tuple
  $\mathcal{A} = \tuple{Q, q_{0}, D, \Sigma, \Delta, \Acc}$ where $Q$ is a 
  finite set of states, $q_{0}$ the initial state, $D$ a finite set of 
  directions, $\Sigma$ is a finite Alphabet and $\Acc\subseteq Q^{\omega}$ is a
  $\omega$-regular target of infinite words of elements in $Q$. Again, we 
  consider $\Acc$ to be expressed as Büchi-, Parity-, Rabin- or 
  Muller-condition. $\Delta$ is defined as set of transitions 
  $\tuple{q,\sigma,G}$ with $q\in Q, \sigma\in\Sigma$ and $G$ is a generator 
  over $Q$ and $D$. Additionally, we define the set of all used generators 
  \begin{equation*}
    \mathcal{G}(\mathcal{A}) = \set{
      G:\text{exist }q\in Q\text{ and }\sigma\in\Sigma
      \text{ such that }\tuple{q,\sigma,G}\in\Delta
    }.
  \end{equation*}
  The semantics of a \ac{WDTA} $\mathcal{A}$ is given as a run
  $r:\tuple{Q\times D}^{*}\rightarrow \mathcal{G}(\mathcal{A})$ on a tree 
  $t:D^{*}\rightarrow\Sigma$ with similar requirements as a run for an 
  alternating tree automata, namely
  \begin{enumerate}
    \item $\tuple{q_{0}, t(\epsilon), r(\epsilon)}\in\Delta$,
    \item and for every $r(\tuple{q_{1}, d_{1}}\dots\tuple{q_{n}, d_{n}}) = G$ 
      there is $\tuple{q_{n}, t(d_{1}\dots d_{n}), G}\in\Delta$.
  \end{enumerate}
\end{definition}
Analogously to the semantic of \acp{PBA} or a strategy in a \ac{MDP} a run 
induces a probability measure on $\mathcal{B}(Q\times D)$ by fixing the measure
of cylinders by 
\begin{equation*}
  \mu_{r}(\cyl(\tuple{q_{1},d_{1}}\dots\tuple{q_{n},d_{n}}))
    = r(\epsilon)(q_{1},d_{1})\cdot\prod\limits_{i = 1}^{n - 1}
    r(\tuple{q_{1},d_{1}}\dots\tuple{q_{i},d_{i}})(q_{i+1},d_{i+1})
\end{equation*}
for every $\tuple{q_{1},d_{1}}\dots\tuple{q_{n},d_{n}}\in\tuple{Q\times D}^{*}$
and uniquely extend\footnote{
  Again, using \cite[Theorem 5.6 (Carath\'{e}odory's extension theorem) and 
  Theorem 5.4]{Bauer}.
} this measure to $\mathcal{B}(Q\times D)$. Naturally, the acceptance of $r$ is 
defined by the measure on those paths accepting the condition in the state 
component, i.e. $\Acc\interval{Q}$. In order to ensure the well-formedness of 
this defintion it is necessary to show the measurability of $\Acc\interval{Q}$. 
But this is obtained by the same argument as for Lemma 
\ref{lem:measureabilityAcceptance} or respectively Corollary 
\ref{cor:borelAcceptance} with setting the base-set of the Borel-algebra as
$Q\times D$ and extending the acceptance conditions accordingly. Hence, we 
state
\begin{lemma}
  For a Büchi-, Parity-, Rabin- or Muller-condition represented as $F$, 
  $\parity$, $R$, $\mathcal{F}$ respectively, the sets $\Acc\interval{Q}(F),
  \Acc\interval{Q}(\parity),\Acc\interval{Q}(R),\Acc\interval{Q}(\mathcal{F})$
  are measureable in $\mathcal{B}(Q\times D)$.
  \label{lem:treeBorelAcceptance}
\end{lemma}
\begin{proof}
  Given a set of states $Q$ with an associated Büchi-condition $F$ and a set of 
  directions $D$ we proof the measureability of $\Acc\interval{Q}(F)$ by 
  setting $Q' = Q\times D$ and $F' = \set{\tuple{q, d}\in Q'\mid q\in F}$. With
  Lemma \ref{lem:measureabilityAcceptance} we obtain that 
  $\Acc(F')\in\mathcal{B}(Q') = \mathcal{B}(Q\times D)$. It suffices to show
  $\Acc(F') = \Acc\interval{Q}(F)$. For every non-empty
  $A = \set{\tuple{p_{1},b_{1}}\dots\tuple{p_{k},b_{k}}}\subseteq Q'$ we have
  $A\cap F'\neq\emptyset$ if and only if there is one $i$ such that 
  $\tuple{p_{i},b_{i}}\in F'$ which holds by choice of $F'$ if and only if
  $p_{i}\in F$ and therefore 
  $A\interval{Q} = \set{p_{1},\dots,p_{k}}\cap F\neq\emptyset$. Then we have
  for every $\alpha = \tuple{q_{1},d_{1}}\tuple{q_{2},d_{2}}\dots\in\Acc(F')$ 
  if and only if $A = \Inf(\alpha)\cap F'\neq\emptyset$ if and only if 
  $A\interval{Q}\cap F\neq\emptyset$ if and only if $q_{1}q_{2}\dots\in\Acc(F)$
  if and only if $\alpha\in\Acc\interval{Q}(F)$. The measureability for 
  Muller-conditions (and therefore also for Parity- or Rabin-conditions) 
  follows analogously to the proof of Corollary \ref{cor:borelAcceptance}.
\end{proof}

In the following we often focus on Parity-conditions and motivate this by the
observation that Parity-conditions are sufficient to express all other 
conditions (as suggested in \cite[page 24:9, Proposition 6]{RandAutoInfTrees}
and which is an integral part of the omitted proof of Theorem 
\ref{thm:probautoequiv}).
\begin{lemma}
  For every \ac{WDTA} $\mathcal{A}$ with a Büchi-, Rabin- or Muller-condition 
  we can construct an equivalent \ac{WDTA} $\mathcal{B}$ with a 
  Parity-condition.
  \label{lem:wdtaparityexpressiveness}
\end{lemma}
\begin{proof}
  Firstly, we observe that Büchi- and Rabin-conditions can be expressed
  as Muller-conditions. Secondly, we construct for a 
  \ac{WDTA}
  \begin{equation*}
    \mathcal{A} = \tuple{
      Q, q_{0}, D, \Sigma, \Delta, \mathcal{F} = \set{F_{1},\dots,F_{n}}
    }
  \end{equation*}
  an equivalent \ac{WDTA} $\mathcal{B}$ with a Parity-condition via the use of
  an \ac{LAR}. Using an \ac{LAR} increases the state space but preserves the 
  structure, especially the weighting, of a run and thus yields the same 
  measure of the set of accepted paths. We construct
  \begin{equation*}
    \mathcal{B}=\tuple{
      \lar(Q),\ell_{0}, D, \Sigma, \Delta', \parity
    }
  \end{equation*} 
  where $\ell_{0}$ is one arbitrary element of $\lar(Q)_{\interval{q_{0}}}$ and 
  $\parity$ is defined as before for \acp{LAR}\footnote{
    Recall that parities grow proportional with the size of the hit-set and are 
    even for hit-sets that are part of $\mathcal{F}$ and odd otherwise.
  } (see proof of Theorem \ref{thm:omegaregularexp}). For every transition 
  $\tuple{q,\sigma,G}\in\Delta$ we define
  \begin{equation*}
    G_{\ell}'(\ell', d) = \begin{cases}
      G_{i}(p,d) &\text{if }\ell' = \up(\ell,p),\\
      0          &\text{otherwise,}
    \end{cases}
  \end{equation*}
  and add
  \begin{equation*}
    \set{
      \tuple{\ell,\sigma,G_{\ell}'}:\text{for every }\ell\in
        \lar(Q)_{\interval{q}}
    }
  \end{equation*}
  to $\Delta'$. Fix a tree $t$ and we observe that every run $r$ of 
  $\mathcal{A}$ can be embedded in a run $r'$ of $\mathcal{B}$ by an 
  inductive construction: for $\tuple{q_{0}, t(\epsilon), r(\epsilon)}\in
  \Delta$ exists an associated $\tuple{\ell_{0}, t(\epsilon), G}$ and we set
  $r'(\epsilon) = G$. Notably, we can pick any element 
  $\tuple{\ell, d}\in\lar(Q)\times D$ with $G(\ell, d) > 0$ and project it to
  $\tuple{p, d}\in Q\times D$ such that $\up(\ell_{0}, p) = \ell$ and iterate
  this argument. This embeds $r$ in $r'$. Moreover, we can observe that this 
  allows to obtain for any run $r'$ of $\mathcal{B}$ on $t$ a run $r$ of 
  $\mathcal{A}$ on $t$ by iteratively using the corresponding original 
  transitions from which transitions in $\mathcal{B}$ are derived. Importantly, 
  we can relate by the inductive construction paths in $r$ and $r'$ in both 
  directions such that
  \begin{center}
    \begin{tabular}{ccccc}
      $q_{0}$ & $\tuple{q_{1}, d_{1}}$ & $\tuple{q_{2}, d_{2}}$ & 
        $\tuple{q_{3}, d_{3}}$ & \dots\\
      $\ell_{0}$ & $\tuple{\ell_{1}, d_{1}}$ & $\tuple{\ell_{2}, d_{2}}$ & 
        $\tuple{\ell_{3}, d_{3}}$ & \dots
    \end{tabular}
  \end{center}
  with $\ell_{i} = \up(\ell_{i-1}, q_{i})$ for all $i>0$ and state that all 
  other paths in the domain of $r'$ have a measure of $0$ by definition of the
  generators in $\mathcal{B}$. Thus, in the \enquote{weighted} part of the 
  domain of $r'$ there is a one-to-one correspondance to elements in the domain
  of $r$. Specifically, we obtain
  \begin{equation*}
    \mu_{r}(\cyl(\tuple{q_{1},d_{1}}\dots\tuple{q_{n},d_{n}})) = 
      \mu_{r'}(\cyl(\tuple{\ell_{1},d_{1}}\dots\tuple{\ell_{n},d_{n}}))
  \end{equation*}
  for the related paths. This allows to conclude with Lemma \ref{lem:larhitset}
  \begin{equation*}
    \mu_{r}(\Acc(\mathcal{F})) = \mu_{r'}(\Acc(\parity)),
  \end{equation*} 
  which yields the claimed acceptance equivalence of $\mathcal{A}$ and 
  $\mathcal{B}$.
\end{proof}

We distinguish certain structural properties on \acp{WDTA} which allows us to
categorize \acp{WDTA}, namely
\begin{definition}[Structural Properties]
  A \ac{WDTA} $\mathcal{A}$ is called
  \begin{description}
    \item [choiceless] if there is at most one transition for every pair 
      $q\in Q, \sigma\in\Sigma$ in $\Delta$,
    \item [uni-directional] if for every clause there is at most one state send
      to every direction, i.e. for every $G\in\mathcal{G}_{\mathcal{A}}$
      there is at most one $q\in Q$ such that $G(q, d) > 0$ for all
      $d\in D$. Intuitively this means that the automaton explores every path
      with at most one state,
    \item [uniformly distributed] if $\mathcal{A}$ is uni-directional
      and for a direction $d\in D$ every clause agrees on the weight that is
      sent down that direction. Thus, for all $d\in D$ holds that for all
      $G_{1}, G_{2}\in\mathcal{G}_{\mathcal{A}}$ we have
      $G_{1}(q, d) = G_{2}(p, d)$ for the unique $q, p$ for which
      $G_{1}(q, d) > 0$, respectively $G_{2}(p, d) > 0$ or
      $G_{1}(q, d) = G_{2}(p, d) = 0$ for all $q, p$. Thus, we can fix a
      probability distribution $B:D\rightarrow \interval{0,1}$ such that for
      every $G\in\mathcal{G}_{\mathcal{A}}$ there is one $q\in Q$ with
      $B(d) = G(q, d)$. We call this distribution $B$ the \emph{blueprint} of
      $\mathcal{A}$. 
  \end{description}
\end{definition}
This allows us to categorize the mainly examined tree automata in 
\cite{RandAutoInfTrees} as uniformly distributed \acp{WDTA} over a fixed set of
directions $D = \set{0,1}$ with a blueprint $B$ such that
\begin{equation*}
  B(0) = B(1) = \frac{1}{2}.
\end{equation*}
Given these definitions we explore the theory of \acp{WDTA}.

\subsection{Closure Properties}
In the following we explore the closure properties of \acp{WDTA} regarding 
union, intersection and negation. Mainly, we re-iterate ideas for \acp{PBA} and
alternating tree automata. Also, we examine these properties selective, 
predominantly focusing on almost-surely accepting models because they prove
most relevant for the examined use-cases later on. Unsuprisingly, the 
non-determinism of the model induces closure under union for certain classes in 
a straightforward manner (as suggested by 
\cite[Proposition 14]{RandAutoInfTrees}). 
\begin{proposition}[Union - uni-directional, unrestricted]
  Uni-directional and unrestricted \acp{WDTA} are closed under union for 
  either positive and almost-sure acceptance.
  \label{prop:uniunrunion}
\end{proposition}
\begin{proof}
  For two automata
  $\mathcal{A}_{1} = \tuple{Q_{1}, q^{0}_{1}, D, \Sigma, \Delta_{1}, 
    \parity_{1}}$ and $\mathcal{A}_{2} = \tuple{Q_{2}, q^{0}_{2}, D, \Sigma, 
  \Delta_{2}, \parity_{2}}$ we can w.l.o.g. assume that 
  $Q_{1}\cap Q_{2} = \emptyset$. The union \ac{WDTA} guesses 
  in the very first transition which of the automata is checked for the tree:
  \begin{equation*}
    \mathcal{C} = \tuple{Q_{1}\cup Q_{2}\uplus\set{q^{i}}, q^{i}, D, \Sigma,
    \Delta' = \Delta_{1}'\cup\Delta_{2}'\cup\Delta^{i},\parity_{\cup}}
  \end{equation*}
  where $q^{i}$ is a new state and the transitions separate in
  \begin{equation*}
    \Delta_{i}' = \set{
      \tuple{q,\sigma,G\interval{Q_{3-j}\mapsto 0}}:
        \tuple{q,\sigma,G}\in\Delta_{i}
    }\text{ for }i = 1,2
  \end{equation*}
  and
  \begin{equation*}
    \Delta^{i} = \set{
      \tuple{q^{i},\sigma,G\interval{Q_{j}\times D\mapsto 0}}:
        \text{for }j = 1,2 \text{ and every } 
        \tuple{q_{0},\sigma,G}\in\Delta_{3-j}
    }.
  \end{equation*}
  Each transition in $\Delta_{1}'$ or $\Delta_{2}'$ mirrors one original 
  transition in either $\mathcal{A}_{1}$ or $\mathcal{A}_{2}$ respectively, but
  its generator is extended to the necessary domain for formal reasons. By 
  setting all transitions for states of the other automaton to $0$ ensures that
  it precisely simulates the original transition. On the other hand we have
  $\Delta^{i}$ as the union of all initial transitions from 
  $\Delta_{1}$ and $\Delta_{2}$ which are again formally defined for the 
  complete domain $\tuple{Q_{1}\cup Q_{2}}\times D$ but effectively mirrors one
  transition from either $\Delta_{1}$ or $\Delta_{2}$. Finally, we define
  \begin{equation*}
    \parity_{\cup}(q) = \begin{cases}
      \parity_{i}&\text{if }q\in Q_{i}\text{ for }i = 1,2,\\
      \min\tuple{\parity_{1}(Q)\cup\parity_{2}(P)}.&\text{if }q = q^{i}.
    \end{cases}
  \end{equation*}
  For any run $r$ of $\mathcal{C}$ the element $r(\epsilon)$ corresponds with a
  transition in $\Delta_{i}$ for $i$ either $1$ or $2$. By the definition of
  the transitions in $\Delta'$ we can state that $\mu_{r}(\cyl(u)) = 0$ for any
  $u\notin\tuple{Q_{i}\times D}^{*}$. More importantly, for any run $r$ of
  $\mathcal{C}$ on $t$ we find a run $r'$ of $\mathcal{A}$ on $t$ (and vice
  verca) such that $\mu_{r}(\cyl(u)) = \mu_{r'}(\cyl(u))$ for any 
  $u\in\tuple{Q_{i}\times D}^{*}$ and therefore $\mu_{r}(A) = \mu_{r'}(A)$ for
  all $A\in\mathcal{B}(Q_{i}\times D)$ and $\mu_{r}(B) = 0$ for all 
  $B\in\mathcal{B}(\tuple{Q_{1}\cup Q_{2}}\times D)
    \setminus\mathcal{B}(Q_{i}\times D)$. Hence, all runs in $\mathcal{C}$
  can be bijected to all runs of either $\mathcal{A}_{1}$ or $\mathcal{A}_{2}$.
  By definition of $\parity_{\cup}$ we have 
  $\Acc(\parity_{\cup})\cap\mathcal{B}(Q_{i}\times D) = \Acc(\parity_{i})$ for
  $i = 1,2$ and therefore $\mathcal{C}$ accepts the trees which are accepted by 
  either $\mathcal{A}_{1}$ or $\mathcal{A}_{2}$. This construction respects 
  uni-directionality and even a common blueprint of $\mathcal{A}_{1}$ and 
  $\mathcal{A}_{2}$ since transition are simply mirrored.
\end{proof}
\begin{corollary}
  The class of uniformly distributed \acp{WDTA} with a fixed blueprint $B$ is 
  closed under union.
\end{corollary}
The natural question arises if this closure persists for the complete class of 
uniformly distributed \acp{WDTA}. Considering this question we examine the
relevance of the local probabilities regarding the resulting measure. For 
\acp{PBA} we already established with Lemma \ref{lem:measureinpba} and its 
consequences that the choice of the transition probabilities does matter. In 
this context we discuss some results from \cite{RandAutoInfTrees}. Firstly, we 
introduce the following example:
\begin{example}
  \cite[Example 7]{RandAutoInfTrees}
  Let $\mathcal{L}$ be the set of all binary trees over $\set{a,b}$ such that
  under a uniform probability distribution of successors the set of all 
  branches containing infinitely many $a$ has measure $1$. We can accept 
  $\mathcal{L}$ with a deterministic uniformly distributed \ac{WDTA} with 
  Büchi-condition and almost-sure acceptance
  \begin{equation*}
    \mathcal{A} = \tuple{\set{q_{a}, q_{b}}, q_{s}, \set{0,1}, \set{a,b}, 
    \Delta, \set{q_{a}}}
  \end{equation*}
  with
  \begin{equation*}
    \Delta = \set{
      \tuple{q_{\sigma}, \sigma, G_{\sigma}}:
        \text{for every }\sigma\in\Sigma
    }
    \text{ and } G_{\sigma}(q, d) = \begin{cases}
      \frac{1}{2}&\text{if }q = q_{\sigma}\text{ and both }d,\\
      0&\text{otherwise}.
    \end{cases}
  \end{equation*}
  The argument that this automaton accepts $\mathcal{L}$ is straightforward 
  since only those paths that contain infinitely often an $a$ do have 
  infinitely often a weighted occurence of $q_{a}$; thus, the run of 
  $\mathcal{A}$ mirrors the input tree $t$ and the language $\mathcal{L}$ is
  defined analogously to the semantic of \acp{WDTA}. Note that $\mathcal{A}$
  respects a blueprint $B(0) = B(1) = \frac{1}{2}$ and can therefore be 
  considered uniformly distributed.
  \label{ex:udwdta}
\end{example}
To this example we additionally introduce
\begin{proposition}
  \cite[Proposition 11]{RandAutoInfTrees}
  For two reals $0 < p < q < 1$ and we define $\mathcal{A}_{p}$ and 
  $\mathcal{A}_{q}$ analogously to $\mathcal{A}$ from Example \ref{ex:udwdta}
  but with blueprint $B(0) = p, B(1) = 1-p$ and $B(0) = q, B(1) = 1-q$ 
  respectively. Then, there is a tree $t$ such that for the unique runs $r_{p}$
  (and $r_{q}$) of $\mathcal{A}_{p}$ (and $\mathcal{A}_{q}$ respectively) on 
  $t$ holds that $\mu_{r_{p}}(\Acc(F)) = 0$ while $\mu_{r_{q}}(\Acc(F)) = 1$.
\end{proposition}
Thus, the \enquote{local} probabilities are relevant for the global measure
which is similar to \acp{PBA}. But in both arguments the examined models are
deterministic in the choice of their transitions. Naturally, this rises 
questions about the benefit to use non-determinism in probabilistic models. As 
we see later on the usage of non-determinism is not a limiting factor in some 
examined decision procedures, e.g. for deciding emptiness. Nevertheless, we 
leave the question for the closure under union for uniformly distributed
\acp{WDTA} open (cp. also \cite[Remark 12]{RandAutoInfTrees}).

Regarding intersection we provide a construction to show closure under 
intersection for \acp{WDTA} under almost-sure acceptance. For positive 
acceptance the closure under intersection is provided for uniformly distributed
\acp{WDTA} with common blueprint. As we will see the defined construction for
almost-sure acceptance actually can be used to show closure under union for
positively accepting \acp{WDTA}. Since it utilizes the same construction as for
positively accepting \acp{PBA} or alternating tree automata respectively.
Thus, for almost-sure acceptance a general intersection operator can separate
in the first transition the runs of both \ac{WDTA} with a probability of
$\frac{1}{2}$ each. Both these runs do individually need an acceptance measure 
of $1$ to render the new run almost-surely accepted.
\begin{proposition}
  \acp{WDTA} with almost-sure acceptance are closed under intersection.
\end{proposition}
\begin{proof}
  We introduce the intersection automaton as follows
  \begin{definition}
    For 
    \begin{equation*}
      \mathcal{A}_{1} = \tuple{
        Q_{1}, q^{1}_{0}, D, \Sigma, \Delta_{1}, \parity_{1}}
      \text{ and }
      \mathcal{A}_{2} = \tuple{
        Q_{2}, q^{2}_{0}, D, \Sigma, \Delta_{2}, \parity_{2}}
    \end{equation*}
    (again assuming w.l.o.g. that $Q_{1}\cap Q_{2} = \emptyset$) we define
    \begin{equation*}
      \mathcal{C} = \tuple{
        Q_{1}\cup Q_{2}\uplus\set{q_{\cap}}, q_{\cap}, D, \Sigma, 
        \Delta_{1}'\cup\Delta_{2}'\cup\Delta_{\cap}, \parity_{\cap}
      }
    \end{equation*}
    with
    \begin{equation*}
      \Delta_{i}' = \set{
        \tuple{q,\sigma,G\interval{Q_{3-i}\cup\set{
          q_{\cap}}\mapsto 0}}:\tuple{q,\sigma, G}\in\Delta_{i}
      }\text{ for }i = 1,2,
    \end{equation*}
    and
    \begin{equation*}
      \Delta_{\cap} = \set{
        \tuple{q_{\cap},\sigma, G^{G_{1}}_{G_{2}}}: \text{for all }
        \tuple{q^{i}_{0},\sigma, G_{i}}\in\Delta_{i}, i = 1,2
      }
    \end{equation*}
    where
    \begin{equation*}
      G^{G_{1}}_{G_{2}}(q,d) = \begin{cases}
        \frac{1}{2}\cdot G_{i}(q,d)&\text{if }q\in Q_{i}\text{ for }i = 1,2,\\
        0&\text{otherwise}.
      \end{cases}
    \end{equation*}
    Regarding the acceptance condition we choose to use every individual 
    acceptance condition and render $q_{\cap}$ irrelevant by setting
    \begin{equation*}
      \parity_{\cap}(q) = \begin{cases}
        \parity_{i}(q)&\text{if }q\in Q_{i}\\
        \min\set{\parity_{1}(q_{0}^{1}),\parity_{2}(q_{0}^{2})}&\text{
          otherwise}.
      \end{cases}
    \end{equation*}
  \end{definition}
  Analogously to the proof of Corollary \ref{cor:treeintersection} we can 
  identify two independent runs on the tree. The main observation is that no
  weighting is exchanged between the $Q_{1}$ and $Q_{2}$ part of any run. Let
  $r$ be a run of $\mathcal{C}$ on a tree $t$, then for all $u\in \tuple{
    \tuple{Q_{1}\cup Q_{2}}\times D}^{*}\setminus\tuple{
    \tuple{Q_{1}\times D}^{*}\cup\tuple{Q_{2}\times D}^{*}}$ we have 
  $\mu_{r}(\cyl(u)) = 0$. Additionally, we can find two runs $r_{1}$ and 
  $r_{2}$ for $\mathcal{A}_{1}$ and $\mathcal{A}_{2}$ respectively on $t$ such
  that for every $u\in\tuple{Q_{1}\times D}^{+}$ and 
  $w\in\tuple{Q_{2}\times D}^{+}$ holds $r_{1}(u)=r(u)$ and $r_{2}(w)=r(w)$ 
  since $\Delta_{i}'$ mirrors $\Delta_{i}$ for $i= 1,2$ respectively (note that
  also for every two such runs $r_{1}$ and $r_{2}$ a run $r$ in $\mathcal{C}$
  exists. By construction of $\Delta_{\cap}$ we can additionally state that 
  $\mu_{r}(\cyl(u)) = \frac{1}{2}\cdot\mu_{r_{1}}(\cyl(u))$ and 
  $\mu_{r}(\cyl(u)) = \frac{1}{2}\cdot\mu_{r_{2}}(\cyl(w))$ respectively. Using
  the fact that $\Acc(\parity_{\cap})$ coincides with $\Acc(\parity_{1})$ and
  $\Acc(\parity_{2})$ on $\tuple{Q_{1}\times D}^{\omega}$ and 
  $\tuple{Q_{2}\times D}^{\omega}$ allows us to concludingly state that
  \begin{equation*}
    \mu_{r}(\Acc(\parity_{\cap})) = 
      \frac{1}{2}\cdot\mu_{r_{1}}(\Acc(\parity_{1})) 
    + \frac{1}{2}\cdot\mu_{r_{2}}(\Acc(\parity_{2})).
  \end{equation*}
  Hence, $\mu_{r}(\Acc(\parity_{\cap})) = 1$ if and only if 
  $\mu_{r_{1}}(\Acc(\parity_{1})) = 1$ and 
  $\mu_{r_{2}}(\Acc(\parity_{2})) = 1$ which ensures that $\mathcal{C}$ indeed
  accepts the intersection of $\mathcal{A}_{1}$ and $\mathcal{A}_{2}$. Note 
  that this construction respects choicelessness of the original automata.

  Additionally, we can observe that $\mu_{r}(\Acc(\parity_{\cap})) > 0$ if and 
  only if $\mu_{r_{1}}(\Acc(\parity_{1}))>0$ or 
  $\mu_{r_{2}}(\Acc(\parity_{2}))>0$; and therefore we can use $\mathcal{C}$ as
  alternative construction to ensure the closure of \acp{WDTA} with positive
  acceptance.
\end{proof}
\begin{corollary}
  The class of languages that can be recognized by choiceless \acp{WDTA} with
  almost-sure (positive) acceptance is closed under intersection (union).  
\end{corollary}
Actually, this results hints at the duality between choiceless \acp{WDTA} with 
positive acceptance and almost-sure acceptance.
\begin{proposition}
  For every choiceless \acp{WDTA}-skeleton (which are the structural components
  but not an acceptance condition) 
  $\mathcal{S} = \tuple{Q, q_{0}, D, \Sigma, \Delta}$ the
  \ac{WDTA} $\mathcal{A} = \tuple{\mathcal{S}, \parity}$ with positive 
  (almost-sure) acceptance is dual to the \ac{WDTA} 
  $\mathcal{C} = \tuple{\mathcal{S}, \parity + 1}$ with almost-sure (positive)
  acceptance.
\end{proposition}
\begin{proof}
  We already argued in the proof of Theorem \ref{thm:omegaregboolean} that 
  increasing all parities by one exchanges accepting and non-accepting paths,
  therefore
  \begin{equation*}
    \tuple{Q\times D}^{\omega} = 
      \Acc\interval{Q}(\parity)\cup\Acc\interval{Q}(\parity + 1)
    \text{ with }
      \Acc\interval{Q}(\parity)\cap\Acc\interval{Q}(\parity + 1) = \emptyset.
  \end{equation*}
  Hence, for the unique run $r_{t}$ of $\mathcal{S}$ on every $t$ we obtain 
  that $\mu_{r_{t}}(\Acc\interval{Q}(\parity)) = 
    1 - \mu_{r_{t}}(\Acc\interval{Q}(\parity + 1)$ and therefore we get that
  $\mathcal{A}$ accepts $t$ almost-surely (positively) if and only if
  $\mathcal{C}$ does not accept $t$ positively (almost-surely).
\end{proof}
For uni-directional \acp{WDTA} with common blueprint we can relate on the 
common product construction for intersection to yield an intersection operator.
The details of this argument can be found in 
\cite[Proposition 14]{RandAutoInfTrees}.
\begin{proposition}
  Uni-directional \acp{WDTA} with common blueprint are closed under
  intersection.
\end{proposition}

For the negation we initially consider one example:
\begin{example}
  We consider $\mathcal{L}$ as the language of all binary $\set{a,b}$-trees 
  such that there is at least one path without an $a$. It is easy to accept 
  this language with a \acp{WDTA} by implementing a search of such a path. We 
  define
  \begin{equation*}
    \mathcal{A} = \tuple{
      \set{q_{s}, q_{f}}, q_{s}, \set{0,1}, \set{a,b}, \Delta, \parity
    }
  \end{equation*}
  with
  \begin{equation*}
    \Delta = \set{\tuple{q_{s}, b, G_{d}}:\text{ for }d\in\set{0,1}}\cup
    \set{
      \tuple{q_{s}, a, G_{f}},\tuple{q_{f}, a, G_{f}},\tuple{q_{f}, a, G_{f}}
    }
  \end{equation*}
  with
  \begin{equation*}
    G_{d}(q, b) = \begin{cases}
      1&\text{if }b = d\text{ and }q = q_{s},\\
      0&\text{otherwise},
    \end{cases}
    \text{ and }
    G_{f}(q, d) = \begin{cases}
      \frac{1}{2}&\text{if }q = q_{f},\\
      0&\text{otherwise}.
    \end{cases}
  \end{equation*}
  Furthermore, we set
  \begin{equation*}
    \parity(q_{s}) = 0\text{ and }\parity(q_{f}) = 1.
  \end{equation*}
  The automaton accepts by guessing a path which only contains of $b$ and thus
  staying within state $q_{s}$ and obtain an accepting run with measure $1$.
  On the other hand, if any $a$ is observed at any time, the run is deemed to
  fail, since the weighting shifts to a uniform distribution on state $q_{f}$
  rendering all weighted paths in the run non-accepting.
  \label{ex:wdtaapath}
\end{example}
This example allows us to argue about the negation, namely we state (as for the
more restricted environment from \cite[Proposition 15]{RandAutoInfTrees})
\begin{proposition}
  The class of languages that are almost-surely accepted by \acp{WDTA} is 
  \emph{not} closed under negation.
\end{proposition}
\begin{proof}
  Naturally, we consider the language $\mathcal{L}$ from Example 
  \ref{ex:wdtaapath} and its complement, i.e. the language 
  $\overline{\mathcal{L}}$ of binary $\set{a,b}$-trees such that every path 
  contains at least one $a$. We show that $\overline{\mathcal{L}}$ is not 
  recognizeable by a \ac{WDTA} and the argument is conceptually simple. The 
  idea is that all paths can wait arbitrarily long to present an $a$. The 
  automaton on the other hand has to distribute weight into all path to not
  miss any occurence of an $a$. This constant distributions renders some path
  negligeable.

  \fxfatal{end proof}
\end{proof}

\section{Modelling in \acp*{WDTA}}
We want to motivate the introduction of \acp{WDTA} by showing their strength to
model the behaviour of other probabilistic models. We want to initially 
consider \acp{PBA}. By reducing the directions to a singelton set a tree 
degenerates to a single path. This path can be interpreted as one word and the 
run of a \ac{PBA} on that word can be modelled as a run of a \ac{WDTA}:
\begin{theorem}
  For every \ac{PBA} $\mathcal{P}$ existent an equivalent \ac{WDTA}
  $\mathcal{A}$ over a singleton direction set and Büchi-condition and for 
  every \emph{choiceless} \ac{WDTA} $\mathcal{A}$ with a Büchi-condition and a
  singleton direction set exists an equivalent \ac{PBA}.
  \label{theorem:pbaequiv}
\end{theorem}
\begin{proof}
  For a \ac{PBA} $\mathcal{P} = \tuple{Q, \Sigma, \delta, q_{0}, F}$ we define
  \ac{WDTA} $\mathcal{A} = \tuple{Q, q_{0}, \set{0}, \Sigma, \Delta, F}$ with
  \begin{equation*}
    \Delta = \set{
      \tuple{q,\sigma,G_{q}^{\sigma}}:\text{for every }q\in Q,\sigma\in\Sigma
    }\text{ with }G_{q}^{\sigma}(p, 0) = \delta(q,\sigma,p).
  \end{equation*}
  By interpreting words $\alpha = \alpha_{0}\alpha_{1}\dots\in\Sigma^{\omega}$
  as \enquote{trees} $t_{\alpha}:\set{0}^{*}\rightarrow\Sigma$ with 
  $t(u) = \alpha_{\size{u}}$ allows us to state that for the unique run 
  $r_{t_{\alpha}}$ of $\mathcal{A}$ on $t_{\alpha}$ and any 
  $u = \tuple{q_{1}, 0}\dots\tuple{q_{n}, 0}\in\tuple{Q\times D}^{*}$
  \begin{align*}
    \mu_{r}(\cyl(u)) &= G_{q_{0}}^{t_{\alpha}(\epsilon)}(q_{1},0)\cdot
      \prod\limits_{i=1}^{n-1}G_{q_{i}}^{t_{\alpha}(0^{i})}(q_{i+1}, 0)\\
      &=\delta(q_{0},t_{\alpha}(\epsilon),q_{1})\cdot
      \delta(q_{1},t_{\alpha}(0),q_{2})\dots
        \delta(q_{n-1},t_{\alpha}(0^{n}),q_{n})\\
      &=\delta(q_{0},\alpha_{0},q_{1})\cdot
      \delta(q_{1},\alpha_{1},q_{2})\dots\delta(q_{n-1},\alpha_{n-1},q_{n}).
  \end{align*}
  Thus, projecting the run $r$ to the state sequences yields an equivalent 
  measure on $\mathcal{B}(Q)$ as the stochastic process of $\mathcal{P}$ (under
  the word $t_{\alpha}$). Since the acceptance conditions also coincide we can 
  ensure that the automata are equivalent regarding positive as well as 
  almost-sure acceptance.

  For the opposite direction, we fix a choiceless \ac{WDTA} with a 
  Büchi-condition and a singleton direction set (w.l.o.g. $D = \set{0}$)
  \begin{equation*}
    \mathcal{A} = \tuple{Q, q_{0}, D , \Sigma, \Delta, F}.
  \end{equation*}
  By the choicelessness of $\mathcal{A}$ there is exactly one 
  $\tuple{q,\sigma, G}\in\Delta$ for every pair $q\in Q$ and $\sigma\in\Sigma$.
  We denote with $G_{q}^{\sigma}$ this unique $G$. The construction of an
  equivalent \ac{PBA}
  \begin{equation*}
    \mathcal{P} = \tuple{Q, \Sigma, \delta, q_{0}, F}
  \end{equation*}
  with $\delta(q,\sigma, p) = G_{q}^{\sigma}(p, 0)$. Translating any tree
  $t:\set{0}^{*}\rightarrow\Sigma$ to an associated word 
  $\alpha_{t} = t(\epsilon)t(0)t(00)t(000)\dots$ allows us to argue that for 
  the unique run $r$ of $\mathcal{A}$ on any $t$ the can be projected to a 
  function $r':Q^{*}\rightarrow\mathcal{G}_{\mathcal{A}}$ by dropping the 
  direction set (since $\size{D} = 1$ this is actually a bijection between
  $\tuple{Q\times D}^{*}$ and $Q^{*}$). This $r'$ induces for all $u\in Q^{*}$
  the same measure for $\cyl(u)$ as the stochastic process of $\mathcal{P}$.
  Naturally, the induced measure on $\mathcal{B}(Q)$ coincide and by 
  equivalence of the acceptance condition also the measure of all accepted 
  paths.
\end{proof}
The central concept here is the equivalence of the measures on 
$\mathcal{B}(Q)$. Naturally, the equivalence can be expanded to all measureable
acceptance conditions. This directly entails
\begin{corollary}
  For every probabilistic word automaton $\mathcal{P}$ with a Rabin-, Parity- 
  or Muller-condition exists an equivalent \ac{WDTA} $\mathcal{A}$ over a 
  singleton direction set and equivalent acceptance condition and for 
  every \emph{choiceless} \ac{WDTA} $\mathcal{A}$ with a Rabin-, Parity- or
  Muller-condition over a singleton direction set exists an equivalent
  probabilistic word automaton $\mathcal{P}$.
\end{corollary}
By the combination of this result and the sufficiency of Parity-conditions to
express all other acceptance conditions for \acp{WDTA} (Lemma 
\ref{lem:wdtaparityexpressiveness}) provides the basis to infer the equivalence 
of \enquote{strong} acceptance conditions for \acp{PBA} (Theorem 
\ref{thm:probautoequiv}) as a corollary.

In the following we want to examine a very interesting observation, namely that
any run of a \ac{WDTA} effectively is an unrollment of a transition system 
which is induced by a tree $t$. This behavior appears very similar to the 
concept of \acp{MDP} where the strategy of the actor unrolls the transition 
system.  Indeed, we establish a very deep connection between \acp{POMDP}. 
Initially, we show that \acp{WDTA} can be used to model \acp{POMDP} where trees 
are interpreted as strategies. Furthermore, we use the mechanism of \acp{POMDP} 
to establish decideability results for algorithmic questions about \acp{WDTA},
e.g. deciding emptiness. This allows us to reveal intriguing connections 
between tree automata and \acp{POMDP}; for example the fact that alternation in
tree automata models partial observability in \acp{MDP}.
\begin{theorem}
  For every \ac{POMDP} $\mathcal{M}$ exists a choiceless \ac{WDTA}-skeleton 
  $\mathcal{S}$ such that strategies for $\mathcal{M}$ when used as input
  for $\mathcal{S}$ induce equivalent probability spaces on the executions of 
  $\mathcal{M}$ and state sequences of $\mathcal{S}$.
  \label{thm:POMDPequivWDTA}
\end{theorem}
\begin{proof}
  Let $\mathcal{M} = \tuple{S, A, \tuple{\tau_{a}}_{a\in A}, s_{0}, \sim}$ be a
  \ac{POMDP}. We can extract the possible observations in $\mathcal{M}$ and
  gather them as $O = \set{\interval{s}_{\sim}\mid:s\in S}$. A strategy for
  $\mathcal{M}$ therefore is only dependent on the sequence of observations up
  to the current state of play, i.e. a word in $u\in O^{*}$. Fixing the
  set of directions to $O$ and the alphabet to $A$ we get $O$-ary $A$-trees as
  equivalent objects to strategies for \ac{POMDP} $\mathcal{M}$. This leads us 
  to 
  \begin{definition}
    For $\mathcal{M} = \tuple{S, A, \tuple{\tau_{a}}_{a\in A}, s_{0}, \sim}$
    with $O = \set{\interval{s}_{\sim}: s\in S}$ define
    \begin{equation*}
      \mathcal{A} = \tuple{S, s_{0}, O, A, \Delta}
    \end{equation*}
    where
    \begin{equation*}
      \Delta = \set{
        \tuple{s, a, G_{s}^{a}}:\text{for all }s\in S\text{ and }a\in A
      }\text{ with }
      G_{s}^{a}(s', o) = \begin{cases}
        \tau_{a}(s, s')&\text{if }o = \interval{s'}_{\sim},\\
        0              &\text{otherwise.}
      \end{cases}
    \end{equation*}
  \end{definition}
  Thus, from the current state those states with observation $o\in O$ are sent
  in direction $o$. This allows us to consider only the effective part of the 
  unique run $r$ on a tree $t$; that is the weighted part. Precisely, any state 
  sequence $u\in S^{*}$ induces a unique associated observation sequence 
  $\interval{u}_{\sim}\in O^{\size{u}}$. By definition of all generators any
  path along a pair $\tuple{s, o}$ such that $o\neq\interval{s}_{\sim}$ is 
  awarded with a probability of $0$ rendering for any word 
  $v = \tuple{s_{1}, o_{1}}\dots\tuple{s_{n}, o_{n}}\in\tuple{S\times O}^{*}$ 
  such that there is $1\leq i\leq n$ with $o_{i}\neq\interval{s_{i}}_{\sim}$
  $\mu_{r}(\cyl(v)) = 0$. Concludingly, we get
  \begin{equation}
    \mu_{r}(\cyl(\tuple{s_{1},o_{1}}\dots\tuple{s_{n},o_{n}})) =
    \begin{cases}
      \prod_{0\leq i\leq n-1}\tau_{t(o_{1}\dots o_{i})}(s_{i}, s_{i+1})
        &\text{if }\interval{s_{i}}_{\sim} = o_{i}\text{ for }1\leq i\leq n,\\
      0 &\text{otherwise}.
    \end{cases}
    \label{eq:wdtapomdp}
  \end{equation}
  Notably, we can define a synchronized subdomain 
  $D\subseteq \tuple{S\times O}^{*}$, i.e. those words such that the second 
  component always coincides with the equivalence class of the first component.
  And reduce $r$ to $D$ and still obtain a viable probability space from $r$.
  Furthermore, since the first component uniquely identifies the second 
  component this can be bijected to $S^{*}$. Hence we consider $r$ and its 
  induced measure as measure on $\mathcal{B}(S)$ rather than on 
  $\mathcal{B}(S\times O)$. And, using Equation \ref{eq:wdtapomdp}
  allows us to conclude that any object $t:O^{*}\rightarrow A$ yields the same
  measure on $\mathcal{B}(S)$ for \ac{WDTA}-sceleton $\mathcal{S}$ (if 
  interpreted as tree) and \ac{POMDP} $\mathcal{M}$ (if interpreted as 
  strategy).
\end{proof}
On the other hand, we want to relate \acp{WDTA} to \acp{POMDP}. Especially, we 
obtain arguments to solve algorithmic questions via the correspondent questions
for \acp{POMDP}. Example \ref{ex:pbaaspomdp} introduced the idea to consider 
\acp{PBA} as \acp{POMDP} with one equivalence class (for the consistency of
generated word). We adapt this idea to obtain a comparable equivalence. Namely,
we construct a \ac{POMDP} for a given \ac{WDTA} such that any run $r$ of 
\ac{WDTA} is the unrollment of the \ac{POMDP} if the tree is used as strategy.
This allows to formulate the emptiness problem of a \ac{WDTA} as strategy 
synthesis question for \acp{POMDP}. The choices of the player (i.e. the 
strategy) are which transition is used at which position of the run. Hence, the
player effectively constructs with his choices the run $r$ by providing for 
every situation the corresponding generator.
\begin{definition}[Emptiness Game]
  For a \ac{WDTA} $\mathcal{A} = \tuple{Q, q_{0}, D, \Sigma, \Delta, T}$ we
  fix the set of actions $A$ for the player as a set
  \begin{equation*}
    A = Q\rightarrow\Delta_{\Sigma}
  \end{equation*}
  where $\tuple{Q\rightarrow\Delta_{\Sigma}}$ is the set of functions that map
  for one fixed $\sigma$ a state $q\in Q$ to a generator in $\Delta(q,\sigma)$.
  Let $\sigma(f)$ for $f\in\tuple{Q\rightarrow\Delta_{\Sigma}}$ be the
  associated element in $\Sigma$ for $f$. The emptiness game is then defined as
  a \ac{POMDP}
  \begin{equation*}
    \mathcal{M}_{\mathcal{A}} = \tuple{\tuple{Q\times D}\uplus \set{q_{0}}, A,
    \tuple{\tau_{a}}_{a\in A}, q_{0}}
  \end{equation*}
  where
  \begin{equation*}
    \tau_{f}(\tuple{q,d},\tuple{q',d'}) = f(q)(q',d')
  \end{equation*}
  and the associated observation relation is $\sim$ with
  \begin{equation*}
    \tuple{q, d}\sim\tuple{p, d'}\text{ if and only if } d = d'
  \end{equation*}
\end{definition}
In order to prove the viability of this game definition we show that a strategy
$s$ can be understood as a tree $t$ and that the induced probability measure on
paths coincides with a run $r$ of $\mathcal{A}$ on $t$. And for any associated
measureable objective $\mathcal{O}\subseteq Q^{\omega}$ which can be projected
onto the states of $\mathcal{M}_{\mathcal{A}}$ by only considering the state
component of any position in the play the measures of plays in
$\mathcal{M}_{\mathcal{A}}$ and in the run-tree coincide. Thus, we get
\begin{theorem}
  There is a strategy $s$ which ensures that an objective
  $\mathcal{O}\subseteq Q^{\omega}$ is positively (resp. almost-surely)
  satisfied in $\mathcal{M}_{\mathcal{A}}$ if and only if there is tree $t$ and
  a run $r$ of $\tuple{\mathcal{A},\mathcal{O}}$ on $t$ which is positively
  (resp. almost-surely) accepted.
  \label{thm:emptinessgame}
\end{theorem}
\begin{proof}
  First of all, we recall that any strategy for $\mathcal{M}_{\mathcal{A}}$ is
  defined as a function which chooses for any finite play the next action, i.e.
  $s:\tuple{\tuple{Q\times D}\uplus\set{q_{0}}}^{+}\rightarrow A$. But
  considering the properties of the game closely the strategy can be reduced to
  its effective core. By the restriction of observation every such strategy
  only observes the directions and thus can be expressed as a function of the
  form $s:\tuple{D\uplus\set{q_{0}}}^{+}\rightarrow A$. Secondly, we define
  $P = \set{\alpha_{1}\alpha_{2}\dots\in\tuple{\tuple{Q\times D}\uplus
  \set{q_{0}}}^{+}\mid \alpha_{0} = q_{0}\text{ and there is no }i>0
  \text{ s.t. }\alpha_{i} =q_{0}}$
  and argue that only plays in $P$ actually do matter in the game.
  \begin{lemma}
    Every two strategies $s,s'$ that agree on all finite prefixes of
    observations of plays in $P$ induce the same measure for plays in
    $\mathcal{M}_{\mathcal{A}}$.
  \end{lemma}
  \begin{proof}
    The begin of any play in $q_{0}$ is by definition of \acp{POMDP} and the
    fact that $q_{0}$ is the initial state. Furthermore, consider any finite
    prefix $u\in\tuple{\tuple{Q\times D}\uplus \set{q_{0}}}^{+}$ of a play
    $\beta\notin P$. And let $i$ be the index of the second occurence of
    $q_{0}$ in $u$. By the definition of $\mathcal{M}_{\mathcal{A}}$ the
    movement from $u_{i-1}$ to $u_{i}$ has a probability of $0$ regardless of
    the choice of either $s$ or $s'$. This yields
    $\mu_{s}(\cyl(u_{0}\dots u_{i})) = \mu_{s'}(\cyl(u_{0}\dots u_{i})) = 0$.
    Thus, only plays in $P$ possibly do have a positive measure, hence
    strategies that coincide in these plays are equivalent.
  \end{proof}
  We can therefore regard any strategy as a function
  $s:q_{0}\cdot D^{*}\rightarrow A$ and extract an associated tree $t$ by
  $t(u) = \sigma(s(q_{0}\cdot u))$ for any $u\in D^{*}$ and also a run $r$ of
  $\mathcal{A}$ on $t$ by setting for
  $\tuple{q_{0}, d_{0}}\dots\tuple{q_{n}, d_{n}}\in\tuple{Q\times D}^{*}$ and
  $f = s(q_{0}d_{0}d_{1}\dots d_{n})$:
  $r(\tuple{q_{0}, d_{0}}\dots\tuple{q_{n}, d_{n}}) = f(q_{n})$ which is by
  definition of $f$ a valid generator from $\Delta(q_{n}, t(d_{0}\dots d_{n}))$.
  In this fashion we can biject plays of the emptiness game to paths in the
  run-tree which induce the same probability measure implying the claim.
\end{proof}
\begin{corollary}[Emptiness Almost-Sure Büchi WDTA]
  The emptiness problem for a \ac{WDTA} with almost-sure acceptance measure of
  a Büchi condition can be decided in exponential time.
  \label{cor:emptiness}
\end{corollary}
\begin{proof}
  Apply \cite[Theorem 5]{QualAnaPOMDP} to the corresponding emptiness game.
\end{proof}
\begin{corollary}[Emptiness Almost-Sure Uni-Directional WDTA]
  The emptiness problem for an uni-directional \ac{WDTA} with almost-sure
  acceptance measure of a $\omega$-regular condition can be decided in
  polynomial time.
\end{corollary}
\begin{proof}
  Since only one state is at any time on a position in the tree the restriction
  of observation can be discarded without losing the consistency of the
  associated tree to the strategy. Then, apply
  \cite[Theorem 3]{RandAutoInfTrees} to the corresponding emptiness game.
\end{proof}


This directly entails some undecideability results carried over from \acp{PBA}.
\begin{corollary}
  The undecideability of the emptiness problem for \acp{PBA} with positive
  acceptance \cite{Groesser} renders the emptiness problem for \acp{WDTA} with
  positive acceptance undecideable.
\end{corollary}
\begin{corollary}
  The emptiness problem for \acp{WDTA} with almost-sure acceptance is
  undecideable.
\end{corollary}
\begin{proof}
  By \cite[Proof of Theorem 1]{DecProblemsForProbAuto} there is an equivalent
  probabilistic Rabin automaton for which the accepted paths have either
  measure $0$ or $1$ for every \ac{PBA} with positive acceptance. Thus, for the
  the equivalent \ac{WDTA} to this probabilistic Rabin automaton positive and
  almost-sure acceptance coincides rendering the emptiness problem for both
  acceptance criteria undecideable.
\end{proof}
Note that this undecideability result does not contradict the decideable
emptiness for \acp{WDTA} with Büchi-condition and almost-sure acceptance but
simply shows that Büchi conditions are less expressive (as it is for
non-weighted tree automata\fxfatal{find reference}).

Having used the fact that moving from Büchi to Rabin acceptance conditions
allow for \acp{PBA} to allow an almost-sure acceptance rather than a positive
acceptance sparks the question if this can be translated to choiceless
\acp{WDTA} as well. Unfortunately, this question can be answered negatively
with
\begin{proposition}
  There exists a language that is accepted by a choiceless \ac{WDTA} with a
  positive acceptance of a Büchi condition that is not accepted by any
  choiceless \ac{WDTA} with a Muller condition.
\end{proposition}
\begin{proof}
  We define $\mathcal{L}_{\exists a}$ as the language of all trees with
  directions $\set{0,1}$ over the alphabet $\set{a,b}$ that do contain an $a$.
  First, we show that this language can be accepted by a choiceless Büchi
  \ac{WDTA} with positive acceptance of a Büchi condition. But this is
  straightforward by constructing a \ac{WDTA} with two states $q_{a}, q_{b}$
  where $q_{b}$ is the initial state. The final state is $q_{a}$ and we
  distribute uniformely over both branches $q_{b}$ until an $a$ occurs where
  the automaton distributes uniformely the state $q_{a}$ regardless of the read
  letter, thus $q_{a}$ is a positive sink-state. Any tree that is accepted does
  necessarily contain an $a$ since it is the only chance to produce an
  acceptance subtree and conversely, if a tree does contain an $a$ there is a
  finite path of length $n$ towards the subtree rooted at this occurence of $a$,
  hence in the run of the automaton on this tree there is an accepted subtree
  which is reached with probability $(\frac{1}{2})^{n} > 0$ rendering the run
  accepting for this tree.

  On the other hand we show that there is no choiceless \ac{WDTA} with an
  almost-sure Muller condition that accepts precisely $\mathcal{L}_{\exists a}$.
  For this consider three trees $t_{0}, t_{1}, t_{b}$ where every letter in
  $t_{i}$ is $b$ except for $t_{i}(i) = a$ for $i = 0,1$ and $t_{b}$ only
  contains $b$. Assume there is a choiceless \ac{WDTA} $\mathcal{A}$ with an
  almost-sure acceptance of a Muller condition to accept
  $\mathcal{L}_{\exists a}$. The unique runs of $A$ on $t_{0},t_{1}, t_{b}$ do
  start with the same distribution in the root since
  $t_{0}(\epsilon) = t_{1}(\epsilon) = t_{b}(\epsilon) = b$ and $\mathcal{A}$
  is choiceless and the runs on $t_{0}$ and $t_{1}$ are accepting.
  Furthermore, we can savely assume that at least on state in each direction
  does carry weight. Because if w.l.o.g. no state in direction $0$ carries any
  weight the measure on all accepted runs for the run on $t_{0}$ and $t_{b}$ is
  equal. Thus, either $t_{0}$ is not accepted or $t_{b}$ is accepted. Both
  cases contradict the assumption that $\mathcal{A}$ precisely accepts
  $\mathcal{L}_{\exists a}$. Additionally, for every state that carries weight
  in the first level of the run of $\mathcal{A}$ on either $t_{0}$ or $t_{1}$
  does need to have a almost-sure measure on its accepted paths. But those
  states on the left side produce for $t_{1}$ and $t_{b}$ the same paths, while
  the states on the right side produce for $t_{0}$ and $t_{b}$ produce the same
  paths. These paths are almost-surely accepted for $t_{0}$ (resp. $t_{1}$) and
  therefore also for $t_{b}$ rendering the unique run on $t_{b}$ almost-surely
  accepted yielding the desired contradiction and concluding the proof.
\end{proof}

\subsection{A forrest of languages}
The different structural properties of \acp{WDTA} introduce different language
classes. We explore in the following how these classes are related to each
other. First of all, we notice that determinism obviously is a restriction.
\begin{proposition}
  The class of languages recognizable with \acp{WDTA} almost-sure acceptance
  is a proper superset of languages recognizable with determinisitc \acp{WDTA}
  with almost-sure acceptance.
\end{proposition}
\begin{proof}
  For u.d. \acp{WDTA} with a common blueprint this follows from
  \cite[Proposition 10]{RandAutoInfTrees}.
  The general case is a little more elaborate but yields an interesting
  intuition: non-deterministic \acp{WDTA} allow for search of properties on
  one path while determinism entails a spread of weight and thus cannot
  recognize properties on one, finitly many or countably many paths if all
  paths are examined. \fxfatal{elaborate}
\end{proof}
It is a very interesting result that alternation in non-weighted tree automata
does not increase the general expressiveness (see \cite{SimAltTreeAuto}).
This result does unfortunately not translate unconstraint to the weighted case.
\begin{proposition}
  The class of languages recognizable with uni-directional \acp{WDTA} with
  Büchi condition is incomparabel with the class of languages recognizable with
  choiceless \acp{WDTA} with Büchi condition under almost-sure acceptance.
\end{proposition}
\begin{proof}
  It is known from \cite[Theorem 4 (b), (c)]{DecProblemsForProbAuto} that
  $\omega$-regular languages and languages recognizable by \acp{PBA} are
  incomparable. We consider the inverse of Theorem \ref{theorem:pbaequiv} for
  \acp{WDTA} over a singleton direction set. Let $D$ be a set with $\size{D}=1$
  and $\mathcal{D} = \tuple{Q, q_{0}, D, \Sigma, \Delta, F}$ a deterministic
  \ac{WDTA}, hence w.l.o.g. (under almost-sure acceptance we can complete
  automata with a non-accepting sink state)
  $\Delta(q,\sigma) = \set{G^{q}_{\sigma}}$. We can straightforwardly define an
  equivalent \ac{PBA} $\mathcal{P} = \tuple{Q, \Sigma, \delta, q_{0}}$ with
  $\delta(q, \sigma, p) = G^{q}_{\sigma}(p, d)$ for the one $d\in D$. Since the
  runtree of $\mathcal{P}$ coincides with the unique run of $\mathcal{D}$ on
  any $\alpha$ the equivalence is obvious. Let
  $\mathcal{U} = \tuple{P, p_{0}, D, \Sigma, \Delta', F'}$ be a uni-directional
  \ac{WDTA}. Since there is only one $d\in D$ and every transition in
  $\mathcal{U}$ is only allowed to put weight into one state per direction
  every run of $\mathcal{U}$ on any $\alpha$ can be reduced to one path with
  weight $1$. This path is either accepting or non-accepting and thus
  the class of accepted languages of those \ac{WDTA} is the class of
  $\omega$-regular languages. Remarkably, we can for any \ac{PBA} define an
  equivalent uni-directional \ac{WDTA} (for any transition from $q$ to $p$ in
  the word automaton introduce a transition in the tree automaton that puts all
  its weight from $q$ to $p$). The claimed incomparability translates
  immediately from the incomparability of the word automata.
\end{proof}
\begin{corollary}
  For Büchi conditions there are deterministic \acp{WDTA} that do not have an
  equivalent uni-directional \ac{WDTA} with any acceptance measure.
\end{corollary}
\begin{proof}
  In above construction the measure of the one path of a run in an
  uni-directional \ac{WDTA} is binary ($1$ if the path is accepting and $0$
  otherwise). Thus the acceptance measures coincide. For the \ac{PBA} the
  runtree is in bijection to the run of the \ac{WDTA} and
  \cite[Theorem 4]{RecOmeLangProbAuto} states that \acp{PBA} with positive
  measure are more expressive than $\omega$-regularity.
\end{proof}

\section{Depth vs. Width}
In \cite[Chapter 4]{RandAutoInfTrees} the concept of resolving non-determinism
by probability distributions (as suggested for \acp{PBA}) is explored in the
context of weighted tree automata. Thus, the automaton does not
\enquote{build} a run by choosing fitting transitions but every transition is
explored with a certain probability. We refer to this concept as \emph{Depth}
since it can be illustrated as a number of runs executed in parallel and
weighted accordingly to the probabilities of possible choices to build the run
(see Figure \ref{fig:parallelruns}). We argue that the introduced concept of
\enquote{alternation} for weighted tree automata actually captures this concept
by allowing for a certain \emph{Width} in runs, hence we can resolve
probabilities of transitions by moving states downwards in the runtree according
to the transitions with an adapted weighting. Since resolving non-determinism
probabilistically renders an automaton literally \enquote{choiceless} we expect
the corresponding equivalent \ac{WDTA} to indeed have the choiceless property.
Additionally, we restrict the transitions to obey a certain structure, i.e. we
enforce a property which we reffered to as uniform distribution for \acp{WDTA}.
This de-couples the probability of a movement through the individual runs and
the probability of the run itself which ensures that the definition is
well-founded. Also, since resolving non-determinism with probabilities in
\acp{PBA} yields interesting results this inclusion further strengthens the
relevance of \acp{WDTA}. \fxwarning{sounds as if we are unsure about the
relevance in the first place}

\subsection{Probabilistic Weighted Automata}
In order to explore the setting of resolving non-deterministic choices by
probability distributions we introduce analogously to
\cite[Definition 4.1.1]{RandAutoInfTrees} the class of \ac{PWA} with
\begin{definition}[Probabilistic Weighted Automata]
  Fix a finite alphabet $\Sigma$, a finite set of directions $D$ and a finite
  set of states $Q$. We fix a probability distribution
  $B:D\rightarrow\interval{0,1}$ which fixes the movement through any run. The
  transitions are captured by a finite non-empty set of functions
  $\set{\delta_{1},\dots,\delta_{n}}$ with $\delta_{i}:D\rightarrow Q$ for
  $1\leq i\leq n$. A function $\Delta$ is used to map every tuple of any
  $q\in Q$ and $\sigma\in\Sigma$ to a probability distribution of the
  associated set of transitions $\set{\delta_{1},\dots,\delta_{n}}$.
  Furthermore, we introduce an $\omega$-regular target set
  $T\subseteq Q^{\omega}$ and pick one $q_{0}\in Q$ to combine all these
  elements to a \ac{PWA}
  \begin{equation*}
    \mathcal{A} = \tuple{Q, q_{0}, D, \Sigma, B, \Delta, T}.
  \end{equation*}
\end{definition}
One run of such an automaton $\mathcal{A}$ on a tree $t:D\rightarrow\Sigma$ is
differently defined than runs for \acp{WDTA} since the run does not need to
encode the weighting of paths through the run. Thus, we define one run
$r:D^{*}\rightarrow Q$ such that for every $w\in D^{*}$ there exists one
$\delta\in\supp(\Delta(r(w), t(w)))$ with $r(w\cdot d) = \delta(d)$ for
all $d\in D$ to which we refer to as $\delta_{r(w)}$ and additionally,
$r(\epsilon) = q_{0}$. As seen before, the distribution $B$ induces a measure
$\mu_{B}$ on the $\sigma$-algebra $\mathcal{F}_{\mathcal{A}}$ which is
generated by the cylinders of all $w\in D^{*}$. The set of accepting paths are
those paths which associated state sequence satisfy the target $T$:
$\Acc(r)=\set{\alpha\in D^{\omega}\mid r(\alpha)\in T}$. Semantically, $r$ is
considered almost-surely accepting if $\mu_{B}(\Acc(r)) = 1$.

Additionally, to examine the probabilities induced by $\Delta$ we introduce a
$\sigma$-algebra which is generated by partial runs, i.e. finite beginnings of
runs. Therefore, we define a finite tree as a prefix-closed subset of finite
words over directions, hence
\begin{definition}[Prefix-Tree]
  For a finite set of directions $D$ we call a finite $T\subseteq D^{*}$ a
  prefix-tree if for every $w\in T$ and every finite prefix $v$ of $w$ holds
  $v\in T$. Additionally, we call a prefix-tree $T$ \emph{proper} if for any
  $w\in T$ exists $d\in D$ such that $w\cdot d\in T$ also $w\cdot d'\in T$ for
  all other $d'\in D$ holds. We define the \enquote{inner} nodes of $T$ as
  those notes that do have an extension in $T$ as
  $\inner(T) = \set{w\in T\mid w\cdot d\in T\text{ for any }d\in D}$.
\end{definition}
For a \ac{PWA} $\mathcal{A}$ and a tree $t:D\rightarrow\Sigma$ 
we can - depending on $t$ - define all partial runs of $\mathcal{A}$ on $t$ as
$r:T\rightarrow Q$ for any proper prefix-tree $T$. This partial run has to be
consistent with $t$. Considering such a partial run $r$ we can define the set
of runs $r'$ of $\mathcal{A}$ on $t$ such that these runs agree on $T$ with
$r$, i.e. for all $w\in T$ holds $r(w) = r'(w)$, as $\compRuns(r)$.
Analogously, to the concept of cylinders over a set we can capture a
$\sigma$-algebra $\mathcal{R}_{\mathcal{A}}$ over runs by using the sets
$\compRuns(r)$ for every partial run $r$ as generating sets. For any partial
run $r$ over the prefix-tree $T$ (and thus for the set of compatible runs with
$r$) there is an associated probability with
\begin{equation*}
  \mu_{t}(\compRuns(r)) = \prod\limits_{w\in\inner(T)}
  \Delta(r(w),t(w))(\delta_{r(w)}).
\end{equation*}
Again, as for \acp{WDTA}, this yields a pre-measure on
$\mathcal{R}_{\mathcal{A}}$ which can be uniquely extended to a measure by
\cite[Theorem 5.4]{Bauer}, hence we obtain a measurable space
$\tuple{\mathcal{R}_{\mathcal{A}}, \mu_{t}}$. In terms of this measurable space
we define the acceptance condition. In \cite{RandAutoInfTrees} different
notions of acceptance are discussed, e.g.
\begin{enumerate}
  \item without regard for the weighting induced by $B$ the semantic can be
    defined by the measure of $\mu_{t}$ of those runs that satisfy a Büchi
    condition \cite[analogously to Proposition 38]{RandAutoInfTrees},
  \item regarding the weighting of the generators multiple combinations of
    positive or almost-sure acceptance for the individual runs and positive and
    almost-sure acceptance for those runs deemed accepting.
\end{enumerate}
We want to focus on the acceptance condition of almost-sure property of those
individual runs that are almost-surely accepting. Thus, we considere a tree
accepted if almost-all runs do satisfy the almost-sure acceptance of the
objective $T$, i.e. $\mu_{t}(\set{r\in\Runs\mid\mu_{B}(\Acc(r)) = 1}) = 1$.
Hence, we fix the set of all viable runs on $t$ as $\Runs$ and
the set of all possible paths in a run $\Paths = D^{\omega}$ and define for an
automaton $\mathcal{A}$ the following function within the product
$\sigma$-algebra of all runs and paths
\begin{equation*}
  f_{\mathcal{A}}:\Runs\times\Paths\rightarrow\interval{0,1}
  \text{ with }
  f_{\mathcal{A}}(r, p) = \begin{cases}
    1&\text{ if }r(p)\text{ satisfies }T\\
    0&\text{ otherwise.}
  \end{cases}
\end{equation*}
In the following we show the integrability of the function $f$ and with this
integrability we argue that the definition of the semantic of the \ac{PWA}
$\mathcal{A}$ is well-founded.
\begin{proposition}
  $f_{\mathcal{A}}$ is a measureable function w.r.t. the $\sigma$-algebra
  $\mathcal{R}_{\mathcal{A}}\otimes\mathcal{F}_{\mathcal{A}}$. And thus,
  integrable in $\tuple{\mathcal{R}_{\mathcal{A}},\mu_{t}}\otimes
  \tuple{\mathcal{F}_{\mathcal{A}},\mu_{B}}$.
\end{proposition}
\begin{proof}
  In order to show the measureability of $f_{\mathcal{A}}$ it suffices to show
  the measureability of $f_{\mathcal{A}}^{-1} =
  \set{\tuple{r,p}\in\Runs\times\Paths\mid r(p)\in T}$ which we do in the
  following analogously to \cite[Lemma 36]{RandAutoInfTrees}. And since more
  complex conditions can be expressed as Boolean combinations of Büchi
  conditions (as used in the proof of e.g. Lemma
  \ref{lem:measureabilityAcceptance}) we examine only a Büchi condition and
  infer the result for other $\omega$-regular $T$. By
  \cite[Theorem 22.1]{Bauer} is the product $\sigma$-algebra
  $\mathcal{R}_{\mathcal{A}}\otimes\mathcal{F}_{\mathcal{A}}$ generated by the
  sets $\compRuns(r)\times\cyl(p)$ for all proper partial runs $r$ and
  $p\in D^{*}$. To ease the following argument we restrict our attention to
  partial runs that are \enquote{balanced}(as suggested by
  \cite[Remark 35]{RandAutoInfTrees}, i.e. all paths have the same length.
  That this restriction is purely argumentativ is ensured by
  \begin{lemma}
    The $\sigma$-algebra that is generated by balanced proper partial runs and
    the $\sigma$-algebra generated by all proper partial runs are identical.
  \end{lemma}
  \begin{proof}
    Easily to see is that the set of all proper partial runs contain all
    balanced partial runs which ensures that the algebra generated by proper
    partial runs is at most more complex than the algebra generated by balanced
    proper partial runs. On the other hand, considering one proper partial run
    $r:T\rightarrow Q$. Fix $n = \max{\size{w}:w\in T}$. Pick any $v\in T$ with
    $\size{v}<n$ and $v\cdot d\notin T$ for all $d\in D$. If no such $v$ exists
    $T$ is balanced. If not, it holds that there are only finitely many
    successor functions $\set{\delta_{1},\dots,\delta_{n}}$ associated with the
    state $r(v)$ and the letter $t(v)$. For every $1\leq i\leq n$ construct run
    $r_{i}$ with $r_{i}(u) = r(u)$ for all $u\in T$ and
    $r_{i}(v\cdot d) = \delta(d)$ for all $d\in D$. The domain of every $r_{i}$
    is again a proper partial run and repeating this construction yields
    finitely many balanced proper partial runs $\set{r_{1},\dots, r_{m}}$ that
    agree with $r$ on $T$ and are consistent with $t$. For any $w\in D^{*}$ we
    get that the union of all sets $\compRuns(r_{i})\times\cyl(w)$ for
    $1\leq i\leq n$ coincides with $\compRuns(r)\cyl(w)$ and thus,
    the $\sigma$-algebra induced by balanced proper partial runs contains
    all generating set of the $\sigma$-algebra induced by proper partial runs
    rendering both algebras identical.
  \end{proof}
  As noted above we examine in the following w.l.o.g. a Büchi condition for $T$
  induced by $F\subseteq Q$. This proof re-iterates concepts of the proof for
  Lemma \ref{lem:measureabilityAcceptance} in a more complex setting. For any
  $p\in D^{*}$ fix the set of all $R_{p}\subseteq\Runs\times\Paths$ such that
  $\tuple{r,\rho}\in R_{p}$ if and only if $p\sqsubset\rho$ and for all
  $p\sqsubset u\sqsubset\rho$ holds $r(u)\notin F$. The complement of
  $\cup_{w\in D^{*}}R_{w}$ describes $f_{\mathcal{A}}^{-1}$, since every
  element $\tuple{r,p}$ in the complement of $\cup_{w\in D^{*}}R_{w}$ is not
  part of any $R_{w}$. Obviously, it is not part of those $R_{w}$ for which
  $w\not\sqsubset p$ and for those $w\sqsubset p$ all $w$ do have a prolongation
  $v$ with $w\sqsubset v\sqsubset p$ with $r(v)\in F$. If on the other hand
  $\tuple{r,p}\in f_{\mathcal{A}}^{-1}$ then $p$ is a path in $r$ which
  satisfies the Büchi condition and thus for every prefix $w\sqsubset p$ there
  is a later point $v$ with $w\sqsubset v\sqsubset p$ with $r(v)\in F$ which
  makes $\tuple{r,p}$ not part of any $R_{w}$.

  It remains to show that $R_{w}$ are measureable for every $w\in D^{*}$. Fix
  one such $w$ and for every $n>\size{w}$ gather for all balanced proper
  partial runs $r$ with depth $n$ and words $p\in D^{n}$ with $w\sqsubset p$
  and $r(p)\notin F$. The countable union of all such $\compRuns(r)$ and
  $\cyl(p)$ is collected in $C_{n}$ and we claim that
  $R_{u} = \cap_{\size{u}<n}C_{n}$ by the following argument: for any
  $\tuple{r,p}\in R_{u}$ every prolongation of $u$ that stays on $p$ does not
  visit $F$ and hence for every these prolongations $v$ we know that
  $\tuple{r,p}$ is part of $C_{\size{v}}$ and hence
  $\tuple{r,p}\in\cap_{\size{u}<n}C_{n}$. On the other hand assume
  $\tuple{r,p}\in\cap_{\size{u}<n}C_{n}$, thus for every viable prolongation
  of length $k$ for $u$ $C_{k}$ witnesses the absence of an occurence of $F$.
  
  This renders $f_{\mathcal{A}}^{-1}$ measureable and as indicator function for
  a measureable set integrateable in
  $\tuple{\mathcal{R}_{\mathcal{A}},\mu_{t}}\otimes
  \tuple{\mathcal{F}_{\mathcal{A}},\mu_{B}}$.
\end{proof}
Given the function $f_{\mathcal{A}}$ and its measureability allows to define
a function
\begin{equation*}
  g_{\mathcal{A}}:\Runs\rightarrow\interval{0,1}
  \text{ with }
  g(r) = \mu_{t}(\set{p\in\Paths\mid r(p)\in T}).
\end{equation*}
This function $g_{\mathcal{A}}$ is measureable in $\mathcal{R}_{\mathcal{A}}$
since $g_{\mathcal{A}}^{-1}(\set{1})$ is measureable by \cite[Lemma 23.1]{Bauer}
and again as an indicator function of a measureable set integrable in
$\tuple{\mathcal{R}_{\mathcal{A}},\mu_{t}}$ which renders the measure
$\mu_{t}(\set{r\in\Runs\mid\mu_{B}(\Acc(r)) = 1 })$ well-defined and hence the
acceptance condition of $\mathcal{A}$. This, concludingly, allows us to deduce
\begin{proposition}
  For a \ac{PWA} $\mathcal{A}$ and a tree $t$ holds
  \begin{equation*}
    \mathcal{A}\text{ accepts }t
    \text{ iff }
    \int f_{\mathcal{A}} d\mu_{t}\otimes\mu_{B} = 1.
  \end{equation*}
  \label{prop:pwabyf}
\end{proposition}
\begin{proof}
  The proof consist of a series of equivalences but first of all we introduce
  a helpful lemma:
  \begin{lemma}
    \cite[Lemma 40]{RandAutoInfTrees} Let $\tuple{\Omega,\mathcal{F},\mu}$ be
    a probability space and $f$ a measureable function from $\Omega$ to
    $\interval{0,1}$, then $\int_{\Omega}f d\mu = 1$ if and only if 
    $\mu(f^{-1}(\set{1})) = 1$.
    \label{lem:almosteverywhere}
  \end{lemma}
  Thus, we can derive the following equivalences:
  \begin{align*}
    \mathcal{A}\text{ accepts }t
     &\text{ iff }\mu_{t}(g_{\mathcal{A}}^{-1}(\set{1}) = 1&\\
    &\text{ iff }\int_{\Runs} g d\mu_{t} = 1
     &\text{Lemma \ref{lem:almosteverywhere}}\\
    &\text{ iff }\int_{\Runs}\int_{\Paths} f_{\mathcal{A}} d\mu_{B} d\mu_{t}
     &\text{Definition of }g\\
    &\text{ iff }\int_{\Runs\times\Paths}f_{\mathcal{A}}d\mu_{B}\otimes\mu_{t}.
     &\text{Tonelli's Theorem \cite[Theorem 23.6]{Bauer}}
  \end{align*}
\end{proof}
We proceed by defining an equivalent \ac{WDTA} for a given \ac{PWA}, yielding
\begin{theorem}[PWA inclusion]
  For any \ac{PWA} $\mathcal{A}$ exists a choiceless \ac{WDTA} $\mathcal{B}$
  such that the languages of $\mathcal{A}$ and $\mathcal{B}$ are equivalent.
\end{theorem}
\begin{proof}
  First of all, we define
  $P_{r} = \set{p\in\Paths\mid r(p) \text{ satisfies } T_{\mathcal{A}}}$ which
  is measureable by \cite[Lemma 23.1]{Bauer} because it is the $\Paths$-cut of
  $f_{\mathcal{A}}^{-1}(\set{1})$ and this yields the following equivalence
  \begin{equation*}
    \int f_{\mathcal{A}} d\mu_{t}\otimes\mu_{B} =
    \int\limits_{r\in\Runs}\int\limits_{p\in\Paths}\chi_{P_{r}}(p)
    d\mu_{B} d\mu_{t}.
  \end{equation*}
  We provide the definition of $\mathcal{B}$ as
  \begin{definition}
    For a given \ac{PWA}
    $\mathcal{A} = \tuple{Q, q_{0}, D, \Sigma, B, \Delta, T}$ define a \ac{WDTA}
    $\mathcal{B} = \tuple{Q, q_{0}, D, \Sigma, \Delta', T}$ with
    \begin{equation*}
      \Delta'(q,\sigma) = \set{G_{\sigma}^{q}}
      \text{ and }
      G_{\sigma}^{q}(p, d) = B(d)\cdot\sum^{\delta\in\supp(\Delta(q,\sigma))}_{
      \delta(d) = p}\Delta(\delta).
    \end{equation*} 
  \end{definition}
  And note that indeed $\mathcal{B}$ is choiceless. Further, we examine the
  unique run $r$ of $\mathcal{B}$ on $t$ and the measure $\mu_{r}$ it defines
  for the $\sigma$-algebra which is generated by all cylinders over words in
  $\tuple{Q\times D}$. This measure can be expressed as follows:
  \begin{align*}
    \mu_{r}(\cyl(\tuple{q_{1},d_{1}}\dots\tuple{q_{n},d_{n}}))
    = &\prod\limits_{1\leq i<n} B(d_{i})\sum\limits^{
      \delta\in\supp(\Delta(q_{i},t(d_{1}\dots d_{i})))}_{\delta(d_{i+1}) =
      q_{i+1}}\Delta(\delta)\\
    = &\mu_{B}(\cyl(d_{1}\dots d_{n}))\cdot\prod\limits_{1\leq i<n}\sum\limits^{
      \delta\in\supp(\Delta(q_{i},t(d_{1}\dots d_{i})))}_{\delta(d_{i+1}) =
      q_{i+1}}\Delta(\delta)\\
    = &\mu_{B}(\cyl(d_{1}\dots d_{n}))\\
      &\cdot\mu_{t}(\set{r\in\Runs\mid\text{ for }1\leq i\leq n:r(d_{1}\dots d_{i}) = q_{i}})\\
    = &\int\limits_{r\in\Runs}\int\limits_{p\in\cyl(d_{1}\dots d_{n})}
        \chi_{\cyl(q_{1}\dots q_{n})}(r(p))d\mu_{B} d\mu_{t}.
  \end{align*}
  Using this expression it follows that
  \begin{equation*}
    \mu_{r}(\set{\tuple{q_{1},d_{1}}\tuple{q_{2},d_{2}}\dots\in\tuple{Q\times D}^{\omega}\mid q_{0}q_{1}\dots\in T})
    =\int\limits_{r\in\Runs}\int\limits_{p\in\Paths}\chi_{P_{r}}(p)d\mu_{B} d\mu_{t}.
  \end{equation*}
  Which concludes that $\mathcal{A}$ accepts $t$ if and only if $\mathcal{B}$
  accepts $t$.
\end{proof}
