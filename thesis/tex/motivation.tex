Relying on finite structures, called automata, to parse and classify finite and
infinite objects, e.g. words and trees, is well established \cite{LangAutoLog,%
AutoInfObj}. Automata on words and trees usually operate in a absolute fashion,
i.e. from the current internal state of the automaton and the input the
automaton proceeds into another state by some rule. These rules can be
categorized to be deterministic or non-deterministic where deterministic rules
uniquely identify one successor state from the current internal state and the
read input while non-deterministic rules allow to choose from multiple possible
successor states exactly one. The source of these choices is conceived from
some omniscient oracle which provides the correct choice if one exists. Some
research focuses on making the participation of the oracle more transparent.
Either by finding equivalent deterministic automata
\cite{NonDetBuechiToDetParity} or by considering probabilistic behavior of the
oracle \cite{RandAutoInfTrees,QualTreeLang,RecOmeLangProbAuto} introducing
probabilistic automata.

Moreover, probabilistic behavior has proven invaluable for the analysis of
various domains, e.g. decision making \cite{ActingOptimallyInPOSD} or modelling
environments systems reside in \cite{PrinciplesOfMC}. Given such a model of a
probabilistic environment it is a common task to determine how well a certain
actor behaves \cite{PrinciplesOfMC} or construct a behavior which meets certain
demands \cite{SynProbEnv,QuanStochParityGames}. Additionally, an actor may show
probabilistic behavior as well \cite{RandomnessForFree}. Connections between
models of probabilistic environments and probabilistic automata are already
established \cite{DecProblemsForProbAuto}.

Thirdly, depending on the environment it is reasonable to restrict the
information an actor may observe. This concept of \emph{incomplete or partial
information} is incorporated in models for environments \cite{QualAnaPOMDP} and
connected to certain structural properties of automata 
\cite{ChurchsProblemRevisited}, namely alternation.

In this thesis we discuss approaches around the theory of automata to tackle
the \emph{synthesis problem}. The synthesis problem is generally understood as
the task to derive from a given specification a behavior which satisfy this
specification \cite{Church}. Automata on infinite words and trees are
well-understood tools for reasoning about the synthesis problem. The common
approach for solving synthesis problems by means of automata theory follows
three steps: expressing the specification as language of an automata,
determinising this automata and exectuing it in parallel on all paths of a tree
\cite{ChurchsProblemRevisited,SynProbEnv,AutoInfObj,ParityGamesPosDet}.
However, probabilistic environments allow for a relaxed demand on the
synthesised behavior. It suffices to act according to a specification in
almost-all or some situations \cite{SynProbEnv,PrinciplesOfMC,QualAnaPOMDP}. We
try to translate these approaches to probabilistic automata for words and
trees. Therefore, we incorporate alternation into tree automata with
probabilistic behavior and examine the resulting model and its implications on
the synthesis problem for probabilistic environments with special consideration
of specifications in form of probabilistic word automata.
