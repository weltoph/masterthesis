\chapter{Synthesis}
\label{chapter:synthesis}
\fxfatal{explain synthesis problem}
\section{Probabilitic Büchi Automata}
The synthesis for \acp{PBA} is - in our current context - understood as the
following question:
\begin{definition}[PBA-Synthesis Question]
  For a given \ac{MDP} $\mathcal{M}$ where the actions are a finite set $O$ and
  a \ac{PBA} $\mathcal{A}$ over the alphabet $S_{\mathcal{M}}\times O$ we want
  to find a strategy that generates for every history
  $\tuple{s_{0}, o_{0}}\dots\tuple{s_{n-1},o_{n-1}}s_{n}$ an
  $o_{n}$ such that the measure induced by the probability of projection to the
  first component induced by $\mathcal{M}$ $\mu$ of those words that lay within
  $L(\mathcal{A})$ is 1.
\end{definition}
Considering a qualitative acceptance condition on $\mathcal{A}$ we can answer
this question positively with
\begin{theorem}[Qualitative PBA Synthesis]
  The synthesis question of an environment $\mathcal{M}$ and a specification
  provided as \ac{PBA} $\mathcal{A}$ with a qualitative acceptance condition
  can be decided.
\end{theorem}
We substantiate this claim by providing a construction of a \ac{WDTA} from the
given environment and the specification \ac{PBA}. The obtained \ac{WDTA} - with
a qualitative acceptance measure - accepts precisely those trees that encode a
strategy as required by the synthesis question.
\begin{definition}[Synthesis WDTA]
  For a \ac{MDP} $\mathcal{M} = \tuple{S, O, \tuple{\tau_{o}}_{o\in O},
  \iota_{0}}$ and a \ac{PBA} $\mathcal{P} = \tuple{Q, S\times O, \delta, q_{0},
  F}$ we construct the \ac{WDTA}
  $\mathcal{A} = \tuple{S\times Q\uplus\set{s_{0}}, s_{0}, S, O, \Delta,
  T}$ with $\Delta(s_{0}, \sigma) = \set{G_{0}}$ where
  \begin{equation*}
    G_{0}(\tuple{s,q}, s') =
    \begin{cases}
      \iota_{0}(s)&\text{ if }s = s'\text{ and }q = q_{0}\\
      0 &\text{ otherwise}
    \end{cases}
  \end{equation*}
  and $\Delta(\tuple{s, q}, \sigma) = \set{G^{s}_{\sigma}}$ where
  \begin{equation*}
    G^{s}_{\sigma}(\tuple{z,p}, z') =
    \begin{cases}
      \delta(q, \tuple{s,\sigma}, p)\cdot\tau_{\sigma}(s,z)&\text{ if }z = z'\\
      0 &\text{ otherwise}
    \end{cases}
  \end{equation*}
  Furthermore we define $T$ as a Büchi condition with $T = S\times F$, i.e. the
  original Büchi condition projected to the second component of the
  states in $\mathcal{A}$.
\end{definition}
The proof mainly follows ideas from \cite[Proposition 43]{RandAutoInfTrees}.
Starting with a tree $t$ we consider two measure spaces, namely the space of
executions in the \ac{MDP} modelling the environment induced by the encoded
strategy of $t$ and the space of the executions of the \ac{PBA} modelling the
specification moving along one execution and using the strategy encoded by $t$.
Thus, we get $\mu^{\mathcal{M}}_{t}$ as a measure on the $\sigma$-algebra which
is induced by the set of cylinders in $S_{\mathcal{M}}$. And, we get
$\mu^{\mathcal{P}}_{t}$ as a measure on the $\sigma$-algebra which is induced
by the set of cylinders in $S_{\mathcal{M}}\times Q_{\mathcal{P}}$. The
semantic of $\mu^{\mathcal{P}}_{t}$ is an accumulation of runs of
$\mathcal{P}$ on all executions of the environment, i.e. we set
$\mu^{\mathcal{P}}_{t}(\cyl(\tuple{s_{1}, q_{1}}\dots\tuple{s_{n}, q_{n}}))$
as the probability that for the prefix $s_{1}\dots s_{n}$ $\mathcal{P}$
moves along $q_{1}\dots q_{n}$.
