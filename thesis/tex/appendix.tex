\chapter{Proof of Theorem \ref{thm:StratPOSG}}
In the following we consider one \ac{POSG}
\begin{equation*}
  \mathcal{G} = \tuple{S, s_{0}, E, A, \tuple{\tau_{e,a}}_{e\in E, A\in A}, 
    \sim, \Acc}
\end{equation*}
where $\sim$ defines the equivalence classes for both players. We 
additionally, introduce the notion of games with multiple initial locations 
as $\mathcal{G}_{B}$ for $B\subseteq S$ such that in an initial step \adam{}
may choose any $b\in B$ and subsequently \eve{} and \adam{} compete in the
game $\mathcal{G}_{b} = \tuple{S, b, E, A, 
\tuple{\tau_{e,a}}_{e\in E, A\in A}, \sim, \Acc}$. Naturally, \eve{} has an 
almost-surely (positively) winning strategy in $\mathcal{G}_{B}$ if and only 
if \eve{} has an almost-surely (positively) winning strategy in 
$\mathcal{G}_{b}$.

We consider for a moment Reachability-conditions, i.e. for a set
$R\subseteq S$ its suffices to visit one element in $R$ to win hence the 
associated Reachability-condition is defined as 
\begin{equation*}
  \Acc(R) = \tuple{S\setminus R}^{*}RS^{\omega}.
\end{equation*}
We introduce an observation about winning games with associated 
Reachability-conditions:
\begin{proposition}
  Given $B\subseteq S$ and $R\subseteq S$ such that \eve{} has a positively
  winning strategy $f$ in $\mathcal{G}_{B}$ equipped with the 
  Reachability-condition that $R$ induces. Then, there is an $N\in\mathbb{N}$
  and some probability $\epsilon_{B} > 0$ that bounds from below the 
  probability of the event that a state in $R$ is visited in 
  $\mathcal{G}_{B}$ if \eve{} plays $f$.
\end{proposition}
\begin{proof}
  We fix the notion $p_{N}^{g_{N}, s}$ as the probability of reaching $R$ in 
  less than $N$ steps if \adam{} initially chose $s$ and plays according to 
  $g_{N}$. From this we derive $p_{N}^{g_{N}} = 
  \min_{s\in B}\set{p_{N}^{g_{N}, s}}$ and claim that there is one 
  $N_{0} > 0$ that renders $p_{N}^{g_{N}} > 0$ for all possible $g_{N}$. For 
  the sake of contradiction we assume this is not the case, i.e. for all $N$ 
  there is one strategy $g_{N}$ rendering $p_{N}^{g_{N}} = 0$. Since there 
  are infinitely many considered $N$ but only finitely many $s\in B$ 
  infinitely often $p_{N}^{g_{N}}$ becomes $0$ from starting in one 
  particular $b\in B$. Thus, we fix such a $b$ and obtain that for more and 
  more increasing values for $N$ there is always a strategy $g_{N}$ to obtain 
  $p_{N}^{g_{N}, b} = 0$. Moreover, we can claim this for every $N$ (since 
  the Reachability-condition can never be undone once it is achieved and 
  there is always a higher $N' > N$ for which \adam{} can avoid $R$ which 
  also holds in turn for $N$).

  Since there are only finitely possible decisions in $A$ for the game 
  $\mathcal{G}_{b}$ but we consider infinitely many strategies (for all $N>0$ 
  there is $g_{N})$) infinitely many must agree what to do in the initial 
  situation; say all these strategies make decision $c\in A$. For any 
  possible subsequent game situation, i.e. the game moved from $b$ to 
  one $s_{1},\dots, s_{n}\in S$ with $\tau_{f(\epsilon), c}(b, s_{i}) > 0$ 
  for all $1\leq i\leq n$, we know that $s_{1},\dots, s_{n}\not\in R$ since
  $p_{N}^{g_{N}, b} = 0$ for all these $g_{N}$. For every $1\leq i\leq n$ we 
  consider again infinitely many strategies (all $g_{N}$ with $N>1$ which 
  initially chose $c$; by choice of $c$ this is indeed an infinite collection
  of strategies). Since there are again only finitely many possible choices 
  we obtain by the same argument one choice $c'$ on which infinitely many 
  $g_{N}$ agree on. This argument can be iterated countably many times and we 
  may construct with the choices the strategies agree upon one particular 
  strategy, say $g$, which avoids $R$ all together. But for this one $g$ we
  can argue that $p_{N, b}^{g} = 0$ for all $N>0$. Naturally, playing $g$ 
  with an additional initial choice of $b$ renders $g$ winning against $f$ in
  the positive Reachability-game $\mathcal{G}_{B}$ contradicting the 
  prerequisite that $f$ is positively winning.

  Hence, we can assured the existence of $N_{0} > 0$ which bounds the number
  rounds any counter strategy may avoid $R$ in $\mathcal{G}_{B}$. Since there
  are only finitely many choices in a game up to $N_{0}$ rounds starting in
  any $s\in B$ we may fix these choices as partial strategies of \adam{} as
  $G_{s,N_{0}}$. Setting 
  $\epsilon_{B} = \min\set{p^{g}_{N_{0}}:g\in G_{s, N_{0}}}$ yields the desired 
  result.
\end{proof}

We introduce the notion of \emph{automaton-compatible} strategies. The 
automaton such a strategy is ought to be compatible with is a structure
\begin{equation*}
  \mathcal{T} = \tuple{Q, \Sigma_{E}\times \interval{S}_{\sim_{E}}, q_{0}, 
  q_{s}, \delta, \lambda}
\end{equation*}
where $\Sigma_{E}\times \interval{S}_{\sim_{E}}$ is the input alphabet, 
$q_{0}, q_{s}\in Q$ a start and a sink state respectively, 
$\delta:Q\times \Sigma_{E}\times \interval{S}_{\sim_{E}}\rightarrow Q$ a 
deterministic transition function and $\lambda:Q\rightarrow\Pot(\Sigma_{E})$ a
labelling of the states of $\mathcal{T}$. Additionally, we enforce
\begin{itemize}
  \item $\lambda(q) = \emptyset$ if and only if $q = q_{s}$,
  \item $\delta(q, \tuple{\sigma, x}) = q_{s}$ if and only if 
    $\sigma\not\in\delta(q)$ for all $q\in Q$, $\sigma\in\Sigma_{E}$ and 
    $x\in\interval{S}_{\sim_{E}}$.
\end{itemize}
This automaton $\mathcal{T}$ associates with any play $u\in S^{*}$ in 
$\mathcal{G}$ precisely one $q\in Q$ and moreover, a labelling $\lambda(q)$. 
Any strategy $f$ of \eve{} is considered compatible with $\mathcal{T}$ if for 
any play $u\in S^{*}$ \eve{} plays an action from the associated labelling.

We use these automata to document \eve{}'s knowledge, i.e. the states she 
consideres possible given her observations. In general, this knowledge expands
beyond the observed equivalence class since the history might render certain 
states within the current equivalence class impossible to be the current state.
This induces one special automaton structure: for an initial knowledge we fix
$K_{0}\subseteq S$ such that $K_{0}\subseteq\interval{s}_{\sim_{E}}$ for one 
$s\in S$. For any sequence $u\in \interval{S}_{\sim}^{*}$ we may define the
current knowledge given \eve{} initially considered $K_{0}$ and observed a 
sequence of equivalence classes $u$ as $\currK^{K_{0}}_{E}(u)$. Inductively, we
can define $\currK$ by setting 
$\currK^{K_{0}}_{E}(\interval{s_{0}}_{\sim}) = K_{0}$ and update at every step
the knowledge by $\upK_{E}$ which computes the following knowledge given all
available information: 
\begin{align*}
  \upK_{E}&(K, e, \interval{s}_{\sim_{E}})\\
  &= \set{
    t\in\interval{s}_{\sim_{E}}\mid \text{there is }o\in K\text{ and }a\in A
    \text{ s. t. }\tau_{e,a}(o, t) > 0
  }
\end{align*}
for the current knowledge $K$, the chosen action $e\in E$ of \eve{} and the 
following observation $\interval{s}_{\sim_{E}}$. Hence, we obtain inductively
\begin{equation*}
  \currK(u\cdot s) = \upK(\currK(u), e, \interval{s}_{\sim_{E}})
\end{equation*} where $e$ is the choice of \eve{}'s strategy, i.e. $f(u) = e$ 
for $f$ being her strategy.
