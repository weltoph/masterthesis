\begin{proof}
  In the following we consider one \ac{POSG}
  \begin{equation*}
    \mathcal{G} = \tuple{S, s_{0}, E, A, \tuple{\tau_{e,a}}_{e\in E, A\in A}, 
      \sim, \Acc}
  \end{equation*}
  where $\sim$ defines the equivalence classes for both players. We 
  additionally, introduce the notion of games with multiple initial locations 
  as $\mathcal{G}_{B}$ for $B\subseteq S$ such that in an initial step \adam{}
  may choose any $b\in B$ and subsequently \eve{} and \adam{} compete in the
  game $\mathcal{G}_{b} = \tuple{S, b, E, A, 
  \tuple{\tau_{e,a}}_{e\in E, A\in A}, \sim, \Acc}$. Naturally, \eve{} has an 
  almost-surely (positively) winning strategy in $\mathcal{G}_{B}$ if and only 
  if \eve{} has an almost-surely (positively) winning strategy in 
  $\mathcal{G}_{b}$.

  We consider for a moment Reachability-conditions, i.e. for a set
  $R\subseteq S$ its suffices to visit one element in $R$ to win hence the 
  associated Reachability-condition is defined as 
  \begin{equation*}
    \Acc(R) = \tuple{S\setminus R}^{*}RS^{\omega}.
  \end{equation*}
  We introduce an observation about winning games with associated 
  Reachability-conditions:
  \begin{proposition}
    Given $B\subseteq S$ and $R\subseteq S$ such that \eve{} has a positively
    winning strategy $f$ in $\mathcal{G}_{B}$ equipped with the 
    Reachability-condition that $R$ induces. Then, there is an $N\in\mathbb{N}$
    and some probability $\epsilon_{B} > 0$ that bounds from below the 
    probability of the event that a state in $R$ is visited in 
    $\mathcal{G}_{B}$ if \eve{} plays $f$.
  \end{proposition}
  \begin{proof}
    We fix the notion $p_{N}^{g_{N}, s}$ as the probability of reaching $R$ in 
    less than $N$ steps if \adam{} initially chose $s$ and plays according to 
    $g_{N}$. From this we derive $p_{N}^{g_{N}} = 
    \min_{s\in B}\set{p_{N}^{g_{N}, s}}$ and claim that there is one 
    $N_{0} > 0$ that renders $p_{N}^{g_{N}} > 0$ for all possible $g_{N}$. For 
    the sake of contradiction we assume this is not the case, i.e. for all $N$ 
    there is one strategy $g_{N}$ rendering $p_{N}^{g_{N}} = 0$. Since there 
    are infinitely many considered $N$ but only finitely many $s\in B$ 
    infinitely often $p_{N}^{g_{N}}$ becomes $0$ from starting in one 
    particular $b\in B$. Thus, we fix such a $b$ and obtain that for more and 
    more increasing values for $N$ there is always a strategy $g_{N}$ to obtain 
    $p_{N}^{g_{N}, b} = 0$. Moreover, we can claim this for every $N$ (since 
    the Reachability-condition can never be undone once it is achieved and 
    there is always a higher $N' > N$ for which \adam{} can avoid $R$ which 
    also holds in turn for $N$).

    Since there are only finitely possible decisions in $A$ for the game 
    $\mathcal{G}_{b}$ but we consider infinitely many strategies (for all $N>0$ 
    there is $g_{N})$) infinitely many must agree what to do in the initial 
    situation; say all these strategies make decision $c\in A$. For any 
    possible subsequent game situation, i.e. the game moved from $b$ to 
    one $s_{1},\dots, s_{n}\in S$ with $\tau_{f(\epsilon), c}(b, s_{i}) > 0$ 
    for all $1\leq i\leq n$, we know that $s_{1},\dots, s_{n}\not\in R$ since
    $p_{N}^{g_{N}, b} = 0$ for all these $g_{N}$. For every $1\leq i\leq n$ we 
    consider again infinitely many strategies (all $g_{N}$ with $N>1$ which 
    initially chose $c$; by choice of $c$ this is indeed an infinite collection
    of strategies). Since there are again only finitely many possible choices 
    we obtain by the same argument one choice $c'$ on which infinitely many 
    $g_{N}$ agree on. This argument can be iterated countably many times and we 
    may construct with the choices the strategies agree upon one particular 
    strategy, say $g$, which avoids $R$ all together. But for this one $g$ we
    can argue that $p_{N, b}^{g} = 0$ for all $N>0$. Naturally, playing $g$ 
    with an additional initial choice of $b$ renders $g$ winning against $f$ in
    the positive Reachability-game $\mathcal{G}_{B}$ contradicting the 
    prerequisite that $f$ is positively winning.

    Hence, we can assured the existence of $N_{0} > 0$ which bounds the number
    rounds any counter strategy may avoid $R$ in $\mathcal{G}_{B}$. Since there
    are only finitely many choices in a game up to $N_{0}$ rounds starting in
    any $s\in B$ we may fix these choices as partial strategies of \adam{} as
    $G_{s,N_{0}}$. Setting 
    $\epsilon_{B} = \min\set{p^{g}_{N_{0}}:g\in G_{s, N_{0}}}$ yields the 
    desired result.
  \end{proof}
\end{proof}

