\label{sec:wordautomata}
Following \cite{LangAutoLog} (respectively \cite[Chapter 1]{AutoLogInfGames}
for the Parity-condition below) we introduce word automata with the following
structural definition
\begin{definition}[Word Automaton]
  We define a word automaton as
  $\mathcal{A} = \tuple{Q, \Sigma, q_{0}, \Delta}$ where
  $Q$ is a set of states, $\Sigma$ a finite alphabet, $q_{0}\in Q$ is the
  initial state and $\Delta\subseteq Q\times\Sigma\times Q$ is the transition
  relation. We define some associated notions as follows:
  \begin{description}
    \item [Run]
      For a word $\alpha = \alpha_{0}\alpha_{1}\dots\in\Sigma^{\omega}$ we call
      a sequence $\pi = \pi_{0}\pi_{1}\dots \in Q^{\omega}$ a run of
      $\mathcal{A}$ on $\alpha$ if $\pi_{0} = q_{0}$ and for every
      $i\in\mathbb{N}$ holds that
      $\tuple{\pi_{i}, \alpha_{i}, \pi_{i+1}}\in\Delta$
    \item [Determinism]
      We call $\mathcal{A}$ deterministic if for every pair $q\in Q$ and
      $\sigma\in\Sigma$ the set
      $\set{p\in Q\mid\tuple{q,\sigma,p}\in\Delta}$ has at most one
      element. In that case we occasionally substitute the relation $\Delta$ 
      with a function $\delta:Q\times\Sigma\rightarrow Q$.
  \end{description}
\end{definition}
This allows for automata to yield for an input word a corresponding run, i.e.
attach onto the input word a word of states that is consistent with the
structure of the automaton. These attached words of states are runs of word
automata on words and are categorized as accepting or non-accepting by the
following conditions:
\begin{definition}[Acceptance Conditions]
  For a word automaton $\mathcal{A} = \tuple{Q, \Sigma, q_{0}, \Delta}$ we
  define different acceptance conditions. For this we define for a run
  $\pi = \pi_{0}\pi_{1}\dots\in Q^{\omega}$ the operator
  $\Inf$ as the set of states that occur infinitely in a run, i.e.
  \begin{equation*}
    \Inf(\pi) = \set{q\in Q\middle| \text{ there are infinitely many }
    i\in\mathbb{N} \text{ s.t. } \pi_{i} = q}.
  \end{equation*}
  With this notion we define the following acceptance conditions:
  \begin{description}
    \item [Büchi] A Büchi-condition is defined by a set of final states
      $F\subseteq Q$ and we call a run $\pi$ accepting if
      $\Inf(\pi)\cap F\neq\emptyset$.
    \item [Muller] Muller-conditions are given as a family of state sets, i.e.
      $\mathcal{F}\subseteq 2^{F}$. A run $\pi$ is called accepting if
      $\Inf(\pi)\in\mathcal{F}$.
    \item [Rabin] This acceptance condition is represented by a set of pairs\\
      $\Omega = \set{\tuple{E_{0}, F_{0}}, \dots, \tuple{F_{n}, E_{n}}}$ and we
      call a run $\pi$ accepting if there is an $i$ s.t.
      $\Inf(\pi)\cap F_{i}\neq\emptyset$ but $\Inf(\pi)\cap E_{i} = \emptyset$.
    \item [Parity] This condition is defined by a function
      $\parity:Q\rightarrow \mathbb{N}$. We call a run $\pi$ accepting if the
      maximum of the set $\parity(\Inf(\pi))$ is even (note that due to the 
      finiteness of $Q$, $\Inf(\pi)$ is finite as well and thus the maximum of 
      $\parity(\Inf(\pi))$ exists).
  \end{description}
  For any of these objects $\Gamma\in\set{F, \mathcal{F}, \set{\tuple{E_{0},
  F_{0}},\dots,\tuple{E_{n}, F_{n}}}, \parity}$ we use $\Acc(\Gamma)$ to refer
  to the accepted language. If necessary we use subscripts to $\Acc$ to make 
  the interpretation of its argument clear.
\end{definition}\fxfatal{find fitting reference for prefix-independence}
Note that for these acceptance conditions acceptance solely depends on those
states that occur infinitely often. Hence, the acceptance is at any moment
determined by the \enquote{future}. We call this \emph{prefix-independence} and
formally capture this by stating that every
$\alpha = \alpha_{1}\alpha_{2}\dots\in\Acc(\Gamma)$ implies that for any $i>0$
also $\alpha_{i}\alpha_{i+1}\dots\in\Acc(\Gamma)$ (again we let $\Gamma$ be a
meta variable, ranging over the described finite representations of acceptance
conditions). We categorize word automata by their acceptance condition and if
they are deterministic. For example we call a word automaton $\mathcal{A}$
equipped with a Büchi-condition a \ac{NBA} or respectively \ac{DBA} if
$\mathcal{A}$ is deterministic. And analogously we obtain word automata for
Muller-, Rabin- and Parity-conditions as
\ac{NMA}, \ac{DMA}, \ac{NRA}, \ac{DRA}, \ac{NPA} and \ac{DPA}.

The theory to these word automata, which we call $\omega$-regular automata, is
well established. In the following we recall some of the central results and
selectively provide proofs. Our arguments follow, in some condensed and
occasionally less formal form, \cite[Chapter 1]{AutoLogInfGames}. We start with
\begin{example}
  We consider the language $\mathcal{L}$ as the set of all infinite
  words over the alphabet $\set{a,b}$ such that $a$ occurs only finitely often.
  This language can be accepted using e.g. an \ac{NBA} $\mathcal{A}$ as defined 
  in Figure \ref{fig:finaautomata}. The argument that this \ac{NBA} accepts
  $\mathcal{L}$ is as follows: The non-determinism allows the automaton to
  \enquote{guess} one moment from which on no more $a$ is read. For any word in 
  $\mathcal{L}$ such a moment exists and can therefore be correctly guessed
  while every word that is accepted by $\mathcal{A}$ needs to move to its
  $q_{F}$ state and it is straightforward to argue that no more $a$ can occur
  afterwards.

  Alternatively, we can accept $\mathcal{L}$ by a \ac{DPA} $\mathcal{B}_{P}$,
  \ac{DRA} $\mathcal{B}_{R}$, \ac{DMA}
  $\mathcal{B}_{M}$ which all share the same structure as defined in Figure
  \ref{fig:finaautomata}. We use the different acceptance conditions to model
  the same restriction, namely that $q_{a}$ only occurs finitely often while
  $q_{b}$ occurs infinitely often.
  \begin{description}
    \item [$\mathcal{B}_{P}$] We define $\parity(q_{a}) = 1$ and
      $\parity(q_{b}) = 0$, thus - since the parity of $q_{a}$ trumps the
      parity of $q_{b}$ but is odd - $q_{a}$ must only occur finitely often.
    \item [$\mathcal{B}_{R}$] Here we define $\set{\tuple{E = \set{q_{a}},
      F = \set{q_{b}}}}$, which models the restriction very clearly by
      explicitly stating which states are \enquote{good} respectively
      \enquote{bad} to visit infinitely often.
    \item [$\mathcal{B}_{M}$] The acceptance family is fixed with
      $\mathcal{F} = \set{\set{q_{b}}}$, thus the only way to achieve an
      accepting run is to eventually stay in $q_{b}$.
  \end{description}
  It is noteworthy that all these conditions are essentially modelling the
  abscence of a state. This is occasionally done by defining a set of states
  that must only be visited finitely often. Such an acceptance condition is
  called a \emph{co-Büchi-condition} with $F = \set{q_{a}}$ and describes the
  inverse of a Büchi-condition; that is the lack of occurences of elements in
  $F$ from one point onwards.

  Note that on the other hand we used a structural way to model this negative
  restriction for $\mathcal{A}$.
  \label{ex:fina}
\end{example}
\begin{drawing}
  \caption{In (a) an \ac{NBA} is illustrated which accepts the language of
  words with finitely many $a$. The states of the Büchi-condition $F$ are
  marked by doubling the outline, i.e. $F = \set{q_{F}}$.
  In (b) a deterministic automaton is defined which starting in $q_{a}$ simply
  moves to $q_{a}$ ($q_{b}$) if an $a$ ($b$) is read.}
  \label{fig:finaautomata}
  \resizebox{\textwidth}{!}{\includegraphics{tikz/fina-automaton.pdf}}
\end{drawing}
We radically compress the rich theory of $\omega$-automata by stating the
following
\begin{theorem}
  \cite[Theorem 1.19, Theorem 1.24, Section 1.3.2,Theorem 3.2]{AutoLogInfGames}
  The class of recognizable languages coincides for \acp{NBA}, \acp{NMA},
  \acp{DMA}, \acp{NRA}, \acp{DRA}, \acp{NPA} and \acp{DPA} and is called
  $\omega$-regular languages. \acp{DBA} are strictly less expressive.
  \label{thm:omegaregularexp}
\end{theorem}
In the following we present parts of the proof of this theorem. Mainly, we
introduce and discuss common concepts and arguments for automata which are also 
valuable in more involved contexts.

Firstly, we provide an argument to show that \acp{DBA} are less expressive.
Consider again $\mathcal{L}$ from Example \ref{ex:fina} and the following
argument: the deterministic nature of any \ac{DBA} $\mathcal{A}$ yields one
possible run for every word of $\mathcal{A}$. This means that words with
common finite prefixes share common finite prefixes of their individual runs.
Assume to have one \ac{DBA} $\mathcal{A}$ which accepts $\mathcal{L}$ and
consider the following family of words with common prefixes of increasing 
length
\begin{equation*}
  b^{\omega}, b^{n_{0}}ab^{\omega}, b^{n_{0}}ab^{n_{1}}ab^{\omega}, \dots
\end{equation*}
where every $n_{i}$ is chosen such that the unique run in $\mathcal{A}$
just visited a state in $F$ (which needs to happen since every element of
this word-family is in $\mathcal{L}$ and therefore accepted by
$\mathcal{A}$). Thus, by iterating \enquote{sliding in} $a$ after a visit in
$F$ we construct a word which is accepted but contains infinitely many $a$
contradicting that $\mathcal{A}$ actually recognizes $\mathcal{L}$.

Secondly, we argue that Muller-, Rabin- and Parity-conditions are
equally expressive. Therefore, we initially consider that all deterministic
automata are non-deterministic automata which do not use their
non-determinism. Furthermore it is easy to see, that Rabin- and
Parity-conditions are captured by Muller-conditions by enumerating all
$\Inf$-sets that satisfy the corresponding conditions. In a second step we
show that we can translate \acp{NMA} to \acp{NPA} as well as \acp{NRA}. This
translation makes use of the following construct, called \ac{LAR}:
\begin{definition}[Latest Appearance Record]
  For a finite state set $Q$ define the set of permutations of elements in
  $Q$ as $\perm(Q)\subseteq Q^{\size{Q}}$ and
  \begin{equation*}
    \lar(Q) = \set{\interval{w,h}\mid w\in\perm(Q)\text{ and }1\leq
    h\leq\size{Q}}
  \end{equation*}
  Additionally we define an update function
  $\up:\lar(Q)\times Q\rightarrow\lar(Q)$ with
  \begin{equation*}
    \up(\interval{q_{1}\dots q_{n},h}, q)
    = \interval{qq_{1}\dots q_{i-1}q_{i+1}\dots q_{n},i}
    \text{ for the unique }i\text{ s.t. }q=q_{i}
  \end{equation*}
  and conclude with auxillary definitions of
  \begin{description}
    \item [Hit-set] for $\ell = \interval{q_{1}\dots q_{n},h}$ we call
      $\set{q_{1},\dots,q_{h}}$ the hit-set of $\ell$,
    \item [State partition] the states of $\lar(Q)$ can be partitioned into
      those elements that resulted from an update by $\up(\cdot, q)$ for
      every $q\in Q$. We define
      $\lar(Q)\interval{q} = \set{\interval{q_{1}\dots q_{n}}\in\lar(Q)\mid
        q_{1} = q} = \set{\up(\ell, q):\ell\in\lar(Q)}$.
  \end{description}
\end{definition}
For a word $q_{0}q_{1}q_{2}\dots\in Q^{\omega}$ we can obtain one associated
sequence of elements in $\lar(Q)$ by picking one arbitrary starting value
$\ell_{0}\in\lar(Q)\interval{q_{0}}$ and successively applying the update
operation, i.e. $\ell_{0}\ell_{1}\ell_{2}\dots\in\lar(Q)^{\omega}$ with
$\ell_{i+1} = \up(\ell_{i},q_{i+1})$ for all $i\geq 0$. We can now state that
$\Inf(q_{0}q_{1}q_{2}\dots)$ moves to the beginning of the $\lar(Q)$ elements 
in the associated sequence, formalized in
\begin{lemma}
  \cite[Lemma 1.21]{AutoLogInfGames}
  $\Inf(q_{0}q_{1}q_{2}\dots) = F$ if and only if from one $k>0$ the hit-set 
  of all $\ell_{i}$ for $i>k$ is a subset of $F$ and there are infinitely 
  many $i>k$ s.t. the hit-set of $\ell_{i}$ coincides with $F$.
  \label{lem:larhitset}
\end{lemma}
This Lemma induces that the sequence of hit-sets stabilises to the set
$\Inf(q_{1}q_{2}\dots)$. Thus, $q_{1}q_{2}\dots$ satisfies a Muller-condition
$\mathcal{F}$ if and only if the hit-sets stabilises to an element of
$\mathcal{F}$. Since this implies that the $\Inf$-set is the biggest occuring 
hit-set from one point onwards, this can be captured within a Parity-condition
by associating to $\ell = \interval{p_{1}\dots p_{n},h}$ with hit-set 
$H = \set{p_{1},\dots,p_{h}}$ a Parity-condition
\begin{equation*}
  \parity(\ell) = \begin{cases}
    2\cdot\size{H} &\text{if }H\in\mathcal{F},\\
    2\cdot\size{H}-1 &\text{if }H\notin\mathcal{F}.
  \end{cases}
\end{equation*}
Thus, winning (losing) hit-sets do have an even (odd) parity and notably the
highest parity value is even if and only if
$\Inf(q_{1}q_{2}\dots)\in\mathcal{F}$. Therefore, we can define a \ac{NPA}
that has as state set the elements $\lar(Q)$ and mirrors the transitions of
the original \ac{NMA} through the application of the update function, i.e.
for every $\tuple{q, a, p}$ there is a corresponding
$\tuple{\ell_{q}, a, \up(\ell_{q}, p)}$ for every $\ell_{q}\in\lar(Q)$ that
is produced by $\up(\cdot,q)$. Every run in the \ac{NPA} can be translated
back to a run in the original \ac{NMA} and vice versa and by Lemma
\ref{lem:larhitset} and the definition of $\parity$ we obtain the acceptance
equivalence of both runs. We close this argument by stating that
any Parity-condition can be expressed by a Rabin-condition as follows:
we define for every occuring even parity $k\in\parity(Q)$ a pair
$\tuple{E_{k}, F_{k}}$ where $F_{k} = \parity^{-1}(k)$ and
$E_{k} = \cup_{p>k}\parity^{-1}(p)$; hence, an accepting run regarding the
Parity-condition satisfies the pair associated with the highest parity and
on the other hand, any run that is accepting by the Rabin-condition, namely
by the pair $\tuple{E_{k}, F_{k}}$, has  $k$ as the highest parity which is
the even witness rendering the run accepting.

Thirdly, we conclude by stating (without proof) two results. Firstly, that
there is for every \acp{NMA} an equivalent \ac{NBA} and the other way around
(this is actually easy since Muller-conditions as well as Rabin- and
Parity-conditions can emulate Büchi-conditions):
\begin{theorem}
  \cite[Theorem 1.10]{AutoLogInfGames}
  For every \acp{NBA} exists an equivalent \ac{NMA} and for every \acp{NMA}
  exists an equivalent \ac{NBA}.
\end{theorem}
Secondly, we also avoid the lengthy technicalities of determinization of an
\ac{NBA} by simply stating
\begin{theorem}
  \cite[Theorem 3.6]{AutoLogInfGames}
  \cite[Section 3.2]{NonDetBuechiToDetParity}
  \cite[Theorem 6]{OptBoundsAutoTrans}
  For every \acp{NBA} $\mathcal{A}$ exists an equivalent \ac{DPA}
  $\mathcal{P}$.

  Given that $\mathcal{A}$ has $n$ states then $\mathcal{P}$ can be
  constructed with $n^{2\cdot n + 2}$ states. There are $\mathcal{A}$
  such that $\mathcal{P}$ necessarily has states in the magnitude of $n!$.
\end{theorem}
Intuitively, we observe that determinism is only a real restriction for
Büchi-conditions, but that constructing deterministic automata can be very
costly.

Additionally, we state regarding the closure properties of $\omega$-regular
languages the following
\begin{theorem}
  \cite[Consequence from Theorem 1.5 and Theorem 1.24]{AutoLogInfGames}
  The $\omega$-regular languages form a Boolean-algebra, i.e. are closed under
  union, intersection and negation. The transformations can be effectively
  constructed.
  \label{thm:omegaregboolean}
\end{theorem}
\begin{proof}
  For this proof we give operations that take two (respectively one) word
  automata and construct a word automaton that accepts the union or
  intersection (or respectively the negation). Beginning with the union, we
  make use of non-determinism such that the resulting automaton guesses at the
  beginning in which language the word is and simulates the \enquote{correct}
  automaton. The closure under intersection is obtained by a natural product
  construction, namely
  \begin{definition}
    For $\mathcal{A}_{1} = \tuple{Q, \Sigma, q_{0}, \Delta}$ and
    $\mathcal{A}_{2} = \tuple{P, \Sigma, p_{0}, \nabla}$ we define
    \begin{equation*}
      \mathcal{A}_{1}\otimes\mathcal{A}_{2} = \tuple{Q\times P, \Sigma, 
      \tuple{q_{0}, p_{0}}, \Delta\otimes\nabla}
    \end{equation*}
    with
    \begin{equation*}
      \Delta\otimes\nabla = 
      \set{\tuple{\tuple{q^{1}, p^{1}}, \sigma, 
      \tuple{q^{2}, p^{2}}}: \tuple{q^{1},\sigma, q^{2}}\in\Delta, 
      \tuple{p^{1},\sigma, p^{2}}\in\nabla}.
    \end{equation*}
  \end{definition}
  Every run in this product can be separated into runs in $\mathcal{A}_{1}$ and
  $\mathcal{A}_{2}$. W.l.o.g. we can assume the original two automata to be
  equipped with Muller-conditions $\mathcal{F}_{1},\mathcal{F}_{2}$ and define
  for the product automaton the Muller-condition 
  \begin{equation*}
    \mathcal{F} = \set{\set{\tuple{q^{1}, p^{1}},\dots,\tuple{q^{n}, p^{n}}}:
    \set{q^{1},\dots,q^{n}}\in\mathcal{F}_{1},
    \set{p^{1},\dots,p^{n}}\in\mathcal{F}_{2}}.
  \end{equation*}
  Hence the run in the product automaton is accepted if and only if the
  associated runs in the original automata are accepted.

  The closure under negation can be expressed very elegantly (cp.
  \cite[Transformation 1.25.]{AutoLogInfGames}). W.l.o.g. we consider the
  original automaton to be a \ac{DPA}. By setting a new Parity-condition with
  $\parity'(q) = \parity(q) + 1$ for all $q\in Q$ (denoted by
  $\parity' = \parity + 1$) and keeping the original structure we exchange
  non-accepting and accepting runs and therefore obtain a structural equivalent
  \ac{DPA} that precisely accepts the complement of the original automaton.
  These two operations can be chained to obtain an intersection operator.
\end{proof}

At last, we want to shortly present that deciding emptiness for
$\omega$-regular languages can be decided by means of graph algorithms. The
central idea is to observe that for any \ac{NBA}
$\mathcal{A} = \tuple{Q, q_{0}, \Sigma, \Delta, F}$ and an accepting run $\pi$
at least one $f\in F$ has to occur infinitely often in $\pi$. It is therefore
sufficient to examine if the graph
\begin{equation*}
  \tuple{Q, E_{\Delta}}\text{ with }
  E_{\Delta} = \set{
    \tuple{q, p}\in Q\times Q\mid\text{exists }\sigma\in\Sigma
      \text{ such that }\tuple{q,\sigma,p}\in\Delta
  },
\end{equation*}
contains an $f\in F$ which is reachable from $q_{0}$ and $f$ itself. Moving
from $q_{0}$ to $f$ and then loop in $f$ yields one accepting word (see Figure
\ref{fig:nbaemptiness}). This allows to state
\begin{theorem}
  \cite{DecMethRestArith,EmptinessNBA}
  Emptiness for a given \ac{NBA}
  \begin{equation*}
    \mathcal{A} = \tuple{Q, q_{0}, \Sigma, \Delta, F}
  \end{equation*}
  can be decided. 

  It is possible to compute in time linear to the size of the automaton, i.e.
  in $\mathcal{O}(\size{Q}+\size{\Delta})$, if an \ac{NBA} accepts a word. If
  so $\mathcal{A}$ also accepts an \emph{ultimatively periodic word}, i.e. a
  word of the form $u\cdot v^{\omega}$ for $u, v\in\Sigma^{+}$.
  \label{thm:nbaemptiness}
\end{theorem}

\begin{drawing}
    \caption{Illustration of an accepting run of an \ac{NBA}.}
    \label{fig:nbaemptiness}
    \begin{center}
      \includegraphics{tikz/nbaemptiness.pdf}
    \end{center}
\end{drawing}%
