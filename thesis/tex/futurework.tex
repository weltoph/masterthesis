Naturally, the complex theory of \acp{WDTA} cannot be thoroughly explored in
all its facets in the limited scope of this thesis. For various restricted and
the unrestricted classes there are open questions about closure properties and
expressiveness.

Beside this, we want to motivate research around the non-deterministic choices
of \acp{WDTA}. Considering a given tree $t$, we can adapt Definition
\ref{def:acceptancegame} to design an acceptance game as \ac{MDP} with states
$Q\times D^{*}$ where \eve{} decides which transitions (compatible with $t$) to 
take. Unfortunately, it is not immediately clear how to obtain an emptiness
game from this construction as it is explained in Example
\ref{ex:alternatingtreeemptiness}.  Giving \eve{} control of the tree and the
run entails that she must construct a consistent tree but depending on the
current state might choose different transitions. This induces requirements on
the observations of \eve{} which are not fitting the concept of partial
observability as discussed in this thesis.  Therefore, we propose to consider
co-operative \acp{POSG} where e.g. \eve{} constructs the tree while \adam{}
complementarily builds an associated tree.  The game is won for both players if
almost-surely or positively the acceptance condition of the associated
\ac{WDTA} is satisfied. This approach allows to use the different observations
for both players to obtain a consistent tree and a run for this tree which
utilizes the complete expressiveness of \acp{WDTA}.  This is especially
interesting in the context of the synthesis problem for logics that reason
about the unrollment of transition systems, e.g. \textsc{CTL}. In
\cite{ChurchsProblemRevisited} alternating tree automata are used to formulate
synthesis results for \textsc{CTL} which even incorporating unobservable
information. Additionally, this approach might reveal how the Simulation
Theorem (Theorem \ref{thm:treesimulation}) for alternating tree automata
translates into the weighted case.

It is also possible to use non-deterministic choices in other probabilistic
automata, e.g. \acp{PBA}. This yields an automaton which upon reading a
letter chooses non-deterministically a probability distribution. This
distribution weights the move into its next state. This model subsumes
$\omega$-regular languages by using the non-determinism to choose from Dirac
distributions (for both positive and almost-sure semantics) and includes the
properties of \acp{PBA} with positive or almost-sure semantics respectively.
Initially, since non-determinism is sufficient to use Büchi-conditions for all
$\omega$-regular languages this raises the questions if this holds in the
probabilistic case as well. Analogously to \cite[Theorem 1.10]{AutoLogInfGames}
we obtain an equivalent \ac{NBA} for a given \ac{NPA} by introducing for every
winning parity a copy of the automaton which lacks transitions to higher odd
parities and marking in this copy the states with the associated winning partiy
as Büchi-states. Hence, the \ac{NBA} guesses the moment from which on no higher
odd parity is visited anymore and moves into the \enquote{correct} copy. An
immediate translation of this construction to the probabilistic case does not
work for the almost-sure case: the probability of loosing runs in a
probabilistic word automaton with Parity-condition and almost-sure acceptance
might diminish constantly for a winning run. Nevertheless, a movement into one
of the copies for a winning parity occurs after finitely many steps. Therefore,
there might be no situation after finitely many steps for such a movement which
does not allow for a measurable set of \enquote{failing} runs. In
\cite{RandAutoInfTrees} the \emph{value} of a tree $t$ in an automaton
$\mathcal{A}$ is defined as the supremum of the measure of all accepting paths
over all possible runs of $\mathcal{A}$ for $t$. This notion can be used here
as well and our construction suggests that the value of $\alpha$ is $1$ in the
Büchi-automaton if and only if it is accepted by the original probabilistic
automaton with Parity-condition.

Hence, we expect interesting results on the topic of \acp{WDTA} from
further investigation of co-operative \ac{POSG} which might be used as
emptiness games for \acp{WDTA}. Also, the further study of \acp{WDTA} might
reveal new connections to \acp{POSG} as well. Additionally, we propose to
consider automata with non-determistic as well as probabilistic choices.

Moreover, considering the presented synthesis results it is a natural next step
to develop a logic to express the specification which can be translated to
almost-surely accepting \acp{PBA}. This is an important step for carrying the
formulated synthesis results into the real world. Initial work on this topic
is done in \cite{ProbAutoProbLog} where \acp{PMA} are connected with weighted
monadic second order logic. Also, the complexity for the synthesis problem
associated with a specification given as almost-surely accepting \ac{PBA}
against an antagonistic environment can be further explored. We rely on results
which consider \adam{} more informed than \eve{}. In this case, for both
\adam{} and \eve{} a subset-construction is used to determine winning
strategies. In our specific setting we hope to save one exponential with an
algorithm that makes use of the fact that \eve{} and \adam{} are equally
informed.
