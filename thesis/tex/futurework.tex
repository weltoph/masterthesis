Naturally, the complex theory of \acp{WDTA} cannot be thoroughly explored in
all its facets in the limited scope of this thesis. For various restricted and
the unrestricted classes there are open questions about closure properties and
expressiveness.

We want to raise the question here what happens if the non-deterministic
choices for \acp{WDTA} are considered in the context of simulating \acp{PBA}.
We propose to consider \acp{PBA} which may choose (depending on the current
input letter) a probability distribution which is used to determine their next
move. This model subsumes $\omega$-regular languages by using the
non-determinism to choose from Dirac distributions and includes the properties
of \acp{PBA} as they are. Designing a game to capture acceptance can be done
similar to the games in Definition \ref{def:acceptancegame} as \ac{MDP} where
the decisions of \eve{} constructs the run on the given input word $\alpha$.
Unfortunately, it is not immediate to obtain an emptiness game from this
construction as it is for tree automata because of the problems of information
management explained in Example \ref{ex:alternatingtreeemptiness}: \eve{} needs
to construct a consistent word but depending on the current state might choose
different transitions. Regarding this question we refer to
\cite{AlgorithmsForPOSG} where it is noted that once a strategy for one player
is fixed the other player might consider a \ac{POSG} as \ac{POMDP} allowing to
only consider non-randomised strategies \cite{RandomnessForFree}. But in
general randomised strategies in \ac{POSG} are stronger than non-randomised
strategies.

Also, since non-determinism is sufficient to use
Büchi-conditions for all $\omega$-regular languages this raises the questions
if this holds in the probabilistic case as well.
Analogously to \cite[Theorem 1.10]{AutoLogInfGames} we obtain an equivalent
\ac{NBA} for a given \ac{NPA} by introducing for every winning parity a copy of
the automaton which lacks transitions to higher odd parities and marking in
this copy the states with the associated winning partiy as Büchi-states. Hence,
the \ac{NBA} guesses the moment from which on no higher odd parity is visited
anymore and moves into the \enquote{correct} copy. An immediate translation of
this construction to the probabilistic case does not work. Since the
probability of loosing runs in a probabilistic word automaton with
Parity-condition and almost-sure acceptance might diminish constantly for a
winning run but every non-deterministic choice in the Büchi-automaton has to
occur after finitely many steps still leaving a measurable set of non-accepting
runs behind. The supremum of accepting-runs over all these decisions is $1$ but
there is not necessarily one single run with this almost-sure measure. In
\cite{RandAutoInfTrees} the \emph{value} of a tree is defined in terms of this
supremum notion.

