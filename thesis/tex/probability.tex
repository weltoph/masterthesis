In the following we introduce the basic notions of probability theory on which
we rely in this thesis. We introduce these concepts in a concise fashion and
refer the interested reader to \cite{Klenke} and \cite{Bauer}.
For a given ground-set $\Omega$ we introduce a $\sigma$-algebra as a collection
of sets closed under countable union and negation. Formally, we fix
\begin{definition}[$\sigma$-Algebra]
  For a ground-set $\Omega$ we call $\mathcal{F}\subseteq\Pot(\Omega)$ for
  $\Omega$ a $\sigma$-algebra if
  \begin{enumerate}
    \item $\Omega\in\mathcal{F}$,
    \item for every $A\in\mathcal{F}$ we have $\tuple{\Omega\setminus A}
      \in\mathcal{F}$,
    \item for a countable collection $\tuple{A_{i}}_{i\in\mathbb{N}}$ with
      $A_{i}\in\mathcal{F}$ we also have $\tuple{\cup_{i\in\mathbb{N}}A_{i}}\in
      \mathcal{F}$.
  \end{enumerate}
\end{definition}
Additionally, we define
\begin{definition}[Trace]
  For a $\sigma$-algebra $\mathcal{F}$ for a ground-set $\Omega$ and a
  non-empty set $A\subseteq\Omega$ we define the $A$-trace of $\mathcal{F}$
  as
  \begin{equation*}
    \restrictTo{\mathcal{F}}{A} = \set{B\cap A:B\in\mathcal{F}}.
  \end{equation*}
\end{definition}
The trace of a $\sigma$-algebra is again a $\sigma$-algebra:
\begin{lemma}
  \cite[Theorem 1.26]{Klenke}
  For a $\sigma$-algebra $\mathcal{F}$ of a ground-set $\Omega$ and any
  non-empty set $A\subseteq\Omega$ holds that $\restrictTo{\mathcal{F}}{A}$ is
  a $\sigma$-algebra of the ground-set $A$.
  \label{lem:trace}
\end{lemma}

A probability space is defined by a set of possible results, a $\sigma$-algebra
which describes those collection of events we can observe and a function which
describes how probable a chosen observation is:
\begin{definition}[Probability Space]
  For a set $\Omega$ and a $\sigma$-algebra $\mathcal{F}$ for $\Omega$ we
  define a probability function as $\mu:\mathcal{F}\rightarrow\interval{0,1}$
  such that
  \begin{enumerate}
    \item $\mu(\Omega) = 1$,
    \item for a countable collection $\tuple{A_{i}}_{i\in\mathbb{N}}$ with
      $A_{i}\in\mathcal{F}$ for all $i\in\mathbb{N}$,
      $A_{i}\cap A_{j} = \emptyset$ for all $i, j\in\mathbb{N}$ with $i\neq j$
      and $\mathbb{A} = \cup_{i\in\mathbb{N}}A_{i}$ we have
      \begin{equation*}
        \mu(\mathbb{A}) = \sum_{i\in\mathbb{N}}\mu(A_{i}).
      \end{equation*}
      We call this property $\sigma$-additivity.
  \end{enumerate}
  We call for such a $\mu$ the triple $\tuple{\Omega,\mathcal{F},\mu}$ a 
  probability space and $\tuple{\Omega, \mathcal{F}}$ a measurable space. For
  any finite $A$ we can additionally define
  \begin{description}
    \item [Probability Distribution] as a function
      $P:A\rightarrow\interval{0,1}$ such that
      $\sum_{\omega\in A}P(\omega) = 1$.
    \item [Support] of a probability distribution $P$ as the set of possible
      outcomes, i.e.
      \begin{equation*}
        \supp(P) = \set{\omega\in A\mid P(\omega) > 0}.
      \end{equation*}
    \item [Distribution Set] the set of probability distributions over $A$ as
      $\mathcal{D}(A)$.
    \item [Dirac distribution / measure] as a special probability distribution 
      $P_{a}\in\mathcal{D}(A)$ or probability measure $\mu_{a}$ for every 
      $a\in A$ such that (cp. \cite[Example 1.30]{Klenke} or \cite{POSG})
      \begin{equation*}
        P(\omega) = \begin{cases}
          1&\text{if }\omega = a,\\
          0&\text{otherwise},
        \end{cases}
        \text{ or }
        \mu(B) = \begin{cases}
          1&\text{if }a\in B,\\
          0&\text{otherwise}.
        \end{cases}
      \end{equation*}
  \end{description}
\end{definition}
In order to obtain a probability measure on a $\sigma$-algebra it suffices to
declare the measure for a sufficiently rich family of elements which induce the
measure on the complete $\sigma$-algebra. We quote for this
\begin{theorem}[Carath\'{e}odory's Extension Theorem]
  \cite[Theorem 2.4, Theorem 5.6]{Bauer}
  For a ground-set $\Omega$ and every collection
  $\mathcal{E}\subseteq\Pot(\Omega)$ that is closed under intersection and
  which contains $\emptyset$ and a sequence $\tuple{E_{i}}_{i\in\mathbb{N}}$ 
  such that $\cup_{i\in\mathbb{N}}E_{i} = \Omega$ any $\sigma$-additiv function
  $\mu':\mathcal{E}\rightarrow\interval{0, 1}$
  with $\mu'(\emptyset) = 0$ and $\mu'(\Omega) = 1$ can be \emph{uniquely}
  extended to a probability measure $\mu$ on the smallest $\sigma$-algebra
  containing $\mathcal{E}$ (denoted by $\sigma(\mathcal{E})$). This entails a
  probability space
  \begin{equation*}
    \tuple{\Omega, \sigma(\mathcal{E}), \mu}.
  \end{equation*}
  \label{thm:measureext}
\end{theorem}

The following theorem is used to formalize the intuition that sufficiently
probable events happen again and again if an infinite amount of time passes.
\begin{theorem}[Borel-Cantelli Lemma]
  \cite[Theorem 2.7]{Klenke}\cite{BorelCantelliPairwise}
  For a probability space $\tuple{\Omega, \mathcal{F}, \mu}$ and a sequence 
  $\tuple{A_{i}}_{i>0}$ with $A_{i}\in\mathcal{F}$ and
  \begin{equation*}
    \mathcal{A} = \bigcap_{i > 0}\bigcup_{j > m}A_{j}
  \end{equation*}
  such that $\mu(A_{i}\cap A_{j}) = \mu(A_{i})\cdot\mu(A_{j})$ for all 
  $i \neq j$ and $i,j > 0$ it holds that $\mu(\mathcal{A}) = 1$ if
  \begin{equation*}
    \sum\limits_{i>0}\mu(A_{i}) = \infty.
  \end{equation*}
  The property that $\mu(A_{i}\cap A_{j}) = \mu(A_{i})\cdot\mu(A_{j})$ for 
  all $i \neq j$ and $i,j > 0$ is called \emph{pairwise independence} of 
  all $A_{i}$ for $i > 0$.
  \label{thm:BorelCantelli}
\end{theorem}

The following notions are standard for systems that show probabilistic
movements through a finite state space (cp.
\cite{RandomnessForFree,AlgorithmsForPOSG}). We introduce cylindric sets over
words. Consider a finite alphabet $A$ and for any $u\in A^{*}$ we introduce the 
cylinder of $u$ by all infinite words in $A^{\omega}$ that respect $u$ as
prefix. Formally, we get
\begin{equation*}
  \cyl(u) = u\cdot A^{\omega}.
\end{equation*}
Based upon these cylindric sets we construct the smallest $\sigma$-algebra
containing all these sets  as \emph{Borel}-algebra (denoted by 
$\mathcal{B}(A)$). We can obtain $\mathcal{B}(A)$ by including all cylindric 
sets and consider the transitive closure under negation and countably union of 
this set. This is defined as
\begin{definition}[Borel-algebra]
  For a finite set $A$ we call $\mathcal{B}(A)\subseteq\Pot(A^{\omega})$ the 
  smallest $\sigma$-algebra containing $\cyl(w)$ for all $w\in A^{*}$. We call
  a set $C\subseteq A^{\omega}$ \emph{Borel} if $C\in\mathcal{B}(A)$.
  \label{def:borelalgebra}
\end{definition}
Note that the collection of all cylindric sets $\mathcal{E}$ over a finite set
$A$ together with $\emptyset$ satisfy the conditions for the family in Theorem
\ref{thm:measureext}.

Additionally, we discuss measurable and integrable functions with helpful 
related results \cite[Chapter 7]{Bauer}.
\begin{definition}[Measurable Function]
  For two measurable spaces $\tuple{\Omega_{1}, \mathcal{F}_{1}}$, 
  $\tuple{\Omega_{2}, \mathcal{F}_{2}}$ we call a function 
  $f:\Omega_{1}\rightarrow\Omega_{2}$ $\mathcal{F}_{1}$-$\mathcal{F}_{2}$ 
  measurable if
  \begin{equation*}
    f^{-1}(A) = \set{\omega\in\Omega_{1}\mid f(\omega)\in A}\in\mathcal{F}_{1}
    \text{ for all }A\in\mathcal{F}_{2}.
  \end{equation*}
\end{definition}
We consider particularly measurable function from a measurement space 
$\tuple{\Omega, \mathcal{F}}$ to the measurment space
$\tuple{\interval{0,1}, \mathcal{A}}$ where $\mathcal{A}$ is the smallest
$\sigma$-algebra which contains all intervals $[0, a)$ for any 
$a\in\interval{0,1}$ (cp. \cite[Chapter 4, Chapter 6]{Bauer}). These functions
are called numerical functions and we quote the following helpful
\begin{theorem}
  \cite[Theorem 9.2]{Bauer}
  For a measurable space $\tuple{\Omega, \mathcal{F}}$ and a numerical 
  function $f:\Omega\rightarrow\interval{0,1}$ the 
  $\mathcal{F}$-measurability of $f$ is equivalent to one of the following 
  conditions
  \begin{enumerate}
    \item for all $a\in\interval{0,1}$ holds 
      $\set{p\in\Omega\mid f(p) < a}\in\mathcal{F}$,
    \item for all $a\in\interval{0,1}$ holds 
      $\set{p\in\Omega\mid f(p) \leq a}\in\mathcal{F}$,
    \item for all $a\in\interval{0,1}$ holds 
      $\set{p\in\Omega\mid f(p) > a}\in\mathcal{F}$,
    \item for all $a\in\interval{0,1}$ holds 
      $\set{p\in\Omega\mid f(p) \geq a}\in\mathcal{F}$.
  \end{enumerate}
  \label{thm:measurabilitybyintervals}
\end{theorem}
Moreover, we obtain for all numerical functions the following
\begin{theorem}
  \cite[Theorem 11.6]{Bauer}
  For a probability space $\tuple{\Omega, \mathcal{F}, \mu}$ and a numerical 
  $\mathcal{F}$-measurable function $f:\Omega\rightarrow\interval{0,1}$ the
  integral
  \begin{equation*}
    \int_{\omega\in\Omega}f(\omega)d\mu(\omega)
  \end{equation*}
  is well-defined in the usual sense.
  \label{thm:measurableintegrable}
\end{theorem}
And additionally
\begin{lemma}
  \cite[Lemma 40]{RandAutoInfTrees} Let $\tuple{\Omega,\mathcal{F},\mu}$ be
  a probability space and $f$ a measureable function from $\Omega$ to
  $\interval{0,1}$, then $\int_{\Omega}f d\mu = 1$ if and only if 
  $\mu(f^{-1}(1)) = 1$,
  \label{lem:almosteverywhere}
\end{lemma}

We occasionally use product spaces which are defined as follows 
\cite[Chapter 22]{Bauer}:
\begin{definition}[Product Space]
  For two measurable spaces $\tuple{\Omega_{1}, \mathcal{F}_{1}}$ and 
  $\tuple{\Omega_{2}, \mathcal{F}_{2}}$ we fix
  \begin{equation*}
    \Omega = \Omega_{1}\times\Omega_{2}
  \end{equation*}
  and consider
  \begin{equation*}
    p_{i}:\Omega\rightarrow\Omega_{i}\text{ with }
      p_{i}(\tuple{\omega_{1}, \omega_{2}}) = \omega_{i}\text{ for }i = 1,2.
  \end{equation*}
  The smalles $\sigma$-algebra $\mathcal{F}$ such that $p_{1}, p_{2}$ are 
  $\mathcal{F}$-$\mathcal{F}_{1}$, $\mathcal{F}$-$\mathcal{F}_{2}$ measurable 
  respectively is called product of the $\sigma$-algebras $\mathcal{F}_{1}$ and
  $\mathcal{F}_{2}$ denoted by
  \begin{equation*}
    \mathcal{F} = \mathcal{F}_{1}\otimes\mathcal{F}_{2}.
  \end{equation*}
\end{definition}
We obtain for product algebras that it is sufficient to examine generating
families of sets, namely
\begin{theorem}
  \cite[Theorem 22.1]{Bauer}
  Given two collections $\mathcal{E}_{1}$ and $\mathcal{E}_{2}$ which 
  generate $\sigma$-algebras $\mathcal{F}_{1}$ and $\mathcal{F}_{2}$ for 
  ground-sets $\Omega_{1}$ and $\Omega_{2}$ respectively such that there are
  sequences $\tuple{E^{i}_{j}}_{j\in\mathbb{N}}$ respectively with 
  $E^{i}_{j}\in\mathcal{E}_{i}$ for $i = 1,2$ and $j\in\mathbb{N}$ and 
  $\Omega_{i} = \cup_{j\in\mathbb{N}}E_{j}^{i}$. Then the product 
  $\sigma$-algebra $\mathcal{F}_{1}\otimes\mathcal{F}_{2}$ is generated by 
  $E_{1}\times E_{2}$ for all pairs $E_{1}\in\mathcal{E}_{1}$ and 
  $E_{2}\in\mathcal{E}_{2}$. 
  \label{thm:productgen}
\end{theorem}
Analogously to \cite[Remark 35]{RandAutoInfTrees}, we consider product algebras
of two Borel-algebras and observe the following
\begin{lemma}
  For two finite sets $A, B$ the product algebra
  $\mathcal{A}\otimes\mathcal{B}$ is generated by the sets
  \begin{equation*}
    \cyl(u)\times\cyl(v)\text{ for }u\in A^{n}, v\in B^{n}\text{ for all }n>0.
  \end{equation*}
  \label{lem:productborelalgebras}
\end{lemma}
\begin{proof}
  From Theorem \ref{thm:productgen} we know that
  $\mathcal{B}(A)\otimes\mathcal{B}(B)$ is generated by $\cyl(u)\times\cyl(v)$
  for all $u\in A^{*}$ and $v\in B^{*}$.
  W.l.o.g. we assume $\size{u} < \size{v}$. From $m = \size{v} - \size{u}$ we
  generate the set $E = \set{u\cdot y:y\in A^{m}}$. Since $A$ is finite so is
  $E$ and we may use the generating sets $\cyl(u')\times\cyl(v)$ for all
  $u'\in E$ to obtain the same set as $\cyl(u)\times\cyl(v)$. Therefore, all
  sets $\cyl(u)\times\cyl(v)$ are part of algebra which is generated by
  balanced cylinders. By construction and minimality of Borel-algebras the
  claim follows.
\end{proof}

Moreover regarding probability measures in product spaces we have
\begin{lemma}
  \cite[Lemma 23.2, Theorem 23.3]{Bauer}
  For two probability spaces 
  \begin{equation*}
    \tuple{\Omega_{1}, \mathcal{F}_{1}, \mu_{1}}\text{ and }
    \tuple{\Omega_{2}, \mathcal{F}_{2}, \mu_{2}}
  \end{equation*}
  holds that for every $Q\in\mathcal{F}_{1}\otimes\mathcal{F}_{2}$ the 
  functions
  \begin{equation*}
    \omega_{i}\mapsto\mu_{3-i}(\set{\omega_{3-i}\in\Omega_{3-i}\mid
      \tuple{\omega_{i}, \omega_{3-i}}\in Q})
    \text{ for }i = 1,2
  \end{equation*}
  defined on $\Omega_{i}$ and measurable in $\mathcal{F}_{i}$ for $i = 1,2$
  respectively. Moreover, there is a unique measure $\pi$ (denoted by 
  $\mu_{1}\otimes\mu_{2}$) for the measurable space 
  $\tuple{\Omega_{1}\times\Omega_{2}, \mathcal{F}_{1}\otimes\mathcal{F}_{2}}$ 
  with 
  \begin{equation*}
    \pi(A_{1}\times A_{2}) = \mu_{1}(A_{1})\cdot\mu_{2}(A_{2})\text{ for all }
      A_{1}\in\mathcal{F}_{1}, A_{2}\in\mathcal{F}_{2}.
  \end{equation*}
  \label{lem:productmeasure}
\end{lemma} 
Also, for these product spaces and measurable functions we have
\begin{theorem}[Tonelli's Theorem]
  \cite[Theorem 23.6]{Bauer}
  For two probability spaces $\tuple{\Sigma_{i}, \mathcal{F}_{i}, \mu_{i}}$ 
  ($i = 1,2$) and a $\mathcal{F}_{1}\otimes\mathcal{F}_{2}$-measureable
  numerical function $f:\Sigma_{1}\times\Sigma_{2}\rightarrow\interval{0,1}$
  then we obtain measurability in $\mathcal{A}_{1}$ and $\mathcal{A}_{2}$ 
  respectively of
  \begin{equation*}
    \omega_{1}\mapsto\int_{\Omega_{2}} f(\omega_{1},\cdot)d\mu_{2}
    \text{ and }
    \omega_{2}\mapsto\int_{\Omega_{1}} f(\cdot,\omega_{2})d\mu_{1}.
  \end{equation*}
  Additionally, it holds that
  \begin{equation*}
    \int f d(\mu_{1}\otimes\mu_{2}) 
    = \int_{\omega_{1}\in\Omega_{1}}(\int_{\Omega_{2}} f(\omega_{1},\cdot)
      d\mu_{2})d\mu_{1}
    = \int_{\omega_{2}\in\Omega_{2}}(\int_{\Omega_{1}} f(\cdot,\omega_{2})
      d\mu_{1})d\mu_{2}.
  \end{equation*}
  \label{thm:tonelli}
\end{theorem}

Moving away from measurable functions we consider a more involved way to relay
two measurable spaces in the means of \cite[Definition 8.25]{Klenke}
\begin{definition}[Markov-Kernel]
  For two measurable spaces $\tuple{\Omega_{1},\mathcal{F}_{1}}$ and 
  $\tuple{\Omega_{2},\mathcal{F}_{2}}$ a function
  \begin{equation*}
    K:\Omega_{1}\times\mathcal{F}_{2}\rightarrow\interval{0,1}
  \end{equation*}
  is called a Markov-kernel if
  \begin{enumerate}
    \item $K(\cdot, A)$ is measurable in $\mathcal{F}_{1}$ for all 
      $A\in\mathcal{F}_{2}$,
    \item $K(\omega, \cdot)$ is a probability measure on 
      $\tuple{\Omega_{2},\mathcal{F}_{2}}$ for every $\omega\in\Omega_{1}$.
  \end{enumerate}
\end{definition}
Moreover, we have
\begin{lemma}
  \cite[Remark 8.26]{Klenke} 
  Given two measurable spaces $\tuple{\Omega_{1},\mathcal{F}_{1}}$,
  $\tuple{\Omega_{2},\mathcal{F}_{2}}$ and a collection $\mathcal{E}$ that
  is closed under intersection and contains a sequence of sets
  $\tuple{E_{i}}_{i\in\mathbb{N}}$ such that
  $\cup_{i\in\mathbb{N}}E_{i} = \Omega_{2}$.
  A function
  \begin{equation*}
    K:\Omega_{1}\times\mathcal{F}_{2}\rightarrow\interval{0,1}
  \end{equation*}
  is a Markov-kernel if
  \begin{enumerate}
    \item $K(\cdot, E)$ is measurable in $\mathcal{F}_{1}$ for all 
      $E\in\mathcal{E}$,
    \item $K(\omega, \cdot)$ is a probability measure on
      $\tuple{\Omega_{2},\mathcal{F}_{2}}$ for every $\omega\in\Omega_{1}$.
  \end{enumerate}
  \label{lem:markovkernelgeneratingsets}
\end{lemma}
Similar to Lemma \ref{lem:productmeasure} we have
\begin{theorem}
  \cite[Corollary 14.23]{Klenke}
  For a probability space $\tuple{\Omega_{1}, \mathcal{F}_{1}, \mu}$, a
  measurable space $\tuple{\Omega_{2}, \mathcal{F}_{2}}$ and a Markov-kernel
  $K:\Omega_{1}\times\mathcal{F}_{2}$ there exists a unique probability 
  measure $\mu\otimes K$ on $\tuple{\Omega_{1}\times \Omega_{2}, 
  \mathcal{F}_{1}\otimes\mathcal{F}_{2}}$ with
  \begin{equation*}
    \mu\otimes K(A_{1}, A_{2}) = \int_{\omega\in A_{1}} K(\omega, A_{2})d\mu
    \text{ for all }A_{1}\in\mathcal{F}_{1}, A_{2}\in\mathcal{F}_{2}.
  \end{equation*}
  \label{thm:kernelmeasure}
\end{theorem}

A stronger connection between probability spaces are isomorphisms. We define
analogously to \cite[Definition 8.34]{Klenke}
\begin{definition}
  We call two probability spaces $\tuple{\Omega_{1}, \mathcal{F}_{1}, \mu_{1}}$
  and $\tuple{\Omega_{2}, \mathcal{F}_{2}, \mu_{2}}$ isomorphic if there is a
  bijective function $\rho:\Omega_{1}\rightarrow\Omega_{2}$ such that
  $\rho$ is $\mathcal{F}_{1}$-$\mathcal{F}_{2}$ measurable and $\rho^{-1}$ is
  $\mathcal{F}_{2}$-$\mathcal{F}_{1}$ measurable and 
  $\mu_{2} = \mu_{1}\circ\rho^{-1}$.
\end{definition}
We introduce a certain class of isomorphic probability spaces which we
occasionally make use of: for two finite non-empty sets $A$, $B$ and
a function $f:A\rightarrow B$, we introduce the operator
\begin{equation*}
  \lift_{f}:A^{\omega}\rightarrow\tuple{A\times B}^{\omega}
  \text{ with }\lift_{f}(\alpha_{1}\alpha_{2}\dots) = 
    \tuple{\alpha_{1}, f(\alpha_{1})}\tuple{\alpha_{2}, f(\alpha_{2})}\dots.
\end{equation*}
We infer
\begin{lemma}
  Given finite and non-empty $A, B$ and a function $f:A\rightarrow B$ then the
  probability space $\tuple{A^{\omega}, \mathcal{B}(A), \mu}$ is isomorphic to
  \begin{equation*}
    \tuple{\lift_{f}(A^{\omega}), 
      \restrictTo{\mathcal{B}(A\times B)}{\lift_{f}(A^{\omega})}, \mu'
    } \text{ with } \mu' = \mu\circ\lift_{f}^{-1},
  \end{equation*}
  for the isomorphism $\lift_{f}$.

  Moreover, $\tuple{\tuple{A\times B}^{\omega}, \mathcal{B}(A\times B),
  \mu'}$ forms a probability space itself if we set
  \begin{equation*}
    \mu'(A) = 0\text{ for all }A\in
    \tuple{A\times B}^{\omega}\setminus\lift_{f}(A^{\omega}).
  \end{equation*}
  \label{lem:liftisomorphism}
\end{lemma}
\begin{proof}
  At first, we show that $\lift_{f}$ and $\lift_{f}^{-1}$ are 
  $\mathcal{B}(A)$-$\restrictTo{\mathcal{B}(A\times B)}{\lift_{f}(A^{\omega})}$
  measurable and
  $\restrictTo{\mathcal{B}(A\times B)}{\lift_{f}(A^{\omega})}$-$\mathcal{B}(A)$
  measurable respectively. However, this is an immediate consequence of the
  observation that for every $a_{1}\dots a_{n}\in A^{*}$ holds
  \begin{equation*}
    \lift_{f}(\cyl(a_{1}\dots a_{n})) = 
      \cyl(\tuple{a_{1}, f(a_{1})}\dots\tuple{a_{n}, f(a_{n})})\cap
        \lift_{f}(A^{\omega})
  \end{equation*}
  and that $\restrictTo{\mathcal{B}(A\times B)}{\lift_{f}(A^{\omega})}$ is
  generated by
  \begin{equation*}
    \cyl(\tuple{a_{1}, f(a_{1})}\dots\tuple{a_{n}, f(a_{n})})\cap
      \lift_{f}(A^{\omega})\text{ for all }a_{1}\dots a_{n}\in A^{*}.
  \end{equation*}
  Hence the generating sets of both algebras are bijected by $\lift_{f}$. This
  implies the measurability of $\lift_{f}$ and $\lift_{f}^{-1}$ by construction
  of the respective $\sigma$-algebras. Hence, the isomorphism follows by
  definition of $\mu'$.

  By the measurability of $\lift_{f}$ we obtain that $\lift_{f}(A^{\omega})\in
  \restrictTo{\mathcal{B}(A\times B)}{\lift_{f}(A^{\omega})}$. Therefore, by
  the closure properties of the $\sigma$-algebra we have $\restrictTo{
  \mathcal{B}(A\times B)}{\lift_{f}(A^{\omega})}
  \subseteq\mathcal{B}(A\times B)$. Hence, extending $\mu'$ as proposed in the
  claim indeed yields a probability measure on 
  $\tuple{\tuple{A\times B}^{\omega}, \mathcal{B}(A\times B), \mu'}$.
\end{proof}
