For a finite set $A = \set{a_{1},\dots, a_{n}}$ we consider a finite word over
$A$ as $a_{i_{1}}a_{i_{2}}\dots a_{i_{m}}$ with $1\leq i_{j}\leq n$ for all 
$1\leq j\leq m$. $\epsilon$ denotes the unique word of length $0$ and $A^{*}$
the set of all finite words over $A$.
We call $\mathcal{L}\subseteq A^{*}$ a language 
over finite words. For any two two words $a_{i_{1}}\dots a_{i_{m}}$ and
$a_{j_{1}}\dots a_{j_{k}}$ we denote with 
$a_{i_{1}}\dots a_{i_{m}}\cdot a_{j_{1}}\dots a_{j_{k}}$ the finite word
$a_{i_{1}}\dots a_{i_{m}}a_{j_{1}}\dots a_{j_{k}}$. For two languages 
$\mathcal{L}_{1}$ and $\mathcal{L}_{2}$ we introduce
\begin{equation*}
  \mathcal{L}_{1}\cdot\mathcal{L}_{2} = \set{
    u\cdot v: u\in\mathcal{L}_{1}, v\in\mathcal{L}_{2}}\subseteq A^{*}
\end{equation*}
and for a word $u$ and a language $\mathcal{L}$ we consider 
$u\cdot\mathcal{L}$ to be equivalent to $\set{u}\cdot\mathcal{L}$. If the 
semantics is still clear we might omit $\cdot$ in these expressions. Moreover,
we introduce for all $i\geq 0$
\begin{equation*}
  \mathcal{L}^{0} = \set{\epsilon}
  \text{ and }\mathcal{L}^{i+1} = \mathcal{L}\cdot \mathcal{L}^{i}.
\end{equation*}

Infinite words over $A$ are similarly defined as finite words but requiring a
countable index sequence $\tuple{i_{k}}_{k\in\mathbb{N}}$. We denote with 
$A^{\omega}$ the set of infinite words over $A$ and languages of infinite 
words $\mathcal{L}\subseteq A^{\omega}$. For a language of \emph{finite} words
$\mathcal{L}_{1}\subseteq A^{*}$
and a language of \emph{infinite} words $\mathcal{L}_{2}\subseteq A^{\omega}$
we define
\begin{equation*}
  \mathcal{L}_{1}\cdot\mathcal{L}_{2} = \set{
    u\cdot\alpha:u\in\mathcal{L}_{1}, \alpha\in\mathcal{L}_{2}}
  \subseteq A^{\omega}
\end{equation*}
and for a finite word $u\in A^{*}$ we allow the notation
$u\cdot\mathcal{L}_{2}$ to express $\set{u}\cdot\mathcal{L}_{2}$. In general we
use the convention to use latin letters to label finite words while greek 
letters are used for infinite words.

For two finite words $u, v\in A^{*}$ we denote with $\sqsubseteq$ the prefix 
relation, i.e.
\begin{equation*}
  u\sqsubseteq v \text{ if and only if there is }u'\in A^{*}\text{ such that }
    u\cdot u' = v.
\end{equation*}
We also use $\sqsubseteq$ for the notion that a finite word is a prefix of an
infinite word. Hence, for $u\in A^{*}$ and $\alpha\in A^{\omega}$ we define
\begin{equation*}
  u\sqsubseteq\alpha\text{ if and only if there is }\beta\in A^{\omega}
    \text{ such that }u\beta = \alpha.
\end{equation*}

Considering a function $f:A\rightarrow B$ we allow to implicitly lift $f$ to
sets and finite and infinite words by elementwise application, namely
\begin{center}
  \begin{tabular}{lp{0.5cm}l}
    for $C\subseteq A$ & & $f(C) = \set{f(c) : c\in C}\subseteq B$ \\
    for $u = a_{1}\dots a_{n}\in A^{*}$ & & $f(u) = f(a_{1})
    \dots f(a_{n}) \in B^{*}$ \\
  for $\alpha\in\alpha_{1}\alpha_{2}\dots\in A^{\omega}$
    & & $f(\alpha) = f(\alpha_{1})f(\alpha_{2})\dots\in B^{\omega}$.\\
  \end{tabular}
\end{center}
Additionally, we allow to lift a function $f:A^{*}\rightarrow B$ to infinite
words such that
\begin{equation*}
  f(\alpha_{1}\alpha_{2}\alpha_{3}\dots) = f(\alpha_{1})f(
    \alpha_{1}\alpha_{2})f(\alpha_{1}\alpha_{2}\alpha_{3})\dots.
\end{equation*}

Furthermore, we introduce for a function $f:A\rightarrow B$ the following
notion
\begin{equation*}
  f\interval{A'\mapsto b}\text{ with }f\interval{A'\mapsto b}(a) = 
  \begin{cases}
    b&\text{if }a\in A',\\
    f(a)&\text{otherwise}.
  \end{cases}
\end{equation*}
Additionally, we define $\uplus$ as an operator of disjoint union. Hence, for
$A\uplus B$ we assume $A\cap B = \emptyset$. If this is not the case, we
substitute the elements of $B$ with newly defined elements such that
$A\cap B = \emptyset$ holds. Also, we make use the operator $\Pot$ to construct
the set of all subsets of the argument.
