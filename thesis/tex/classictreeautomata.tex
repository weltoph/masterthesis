\section{Automata on Infinite Trees}
In Section \ref{sec:wordautomata} it is noted that word automata attach to a
word a sequence of states. The idea of classic tree automata is to attach
states to more complex structures, namely trees. For a finite set of
directions $D$ and a finite alphabet $\Sigma$ we define a $D$-ary $\Sigma$-tree
$t:D^{*}\rightarrow\Sigma$. A tree automaton produces a tree of states
which is attached to the input tree, hence a $D$-ary $Q$-tree where $Q$ is the
set of states. The following introduction conceptually follows 
\cite[Chapter 8]{AutoLogInfGames}.
\begin{definition}[Tree automaton]
  Let $Q$ be a finite set of states and $q_{0}\in Q$ be one initial state.
  Furthermore, let $\Delta$ denote a set of transitions of the form
  $\tuple{q,\sigma, \tuple{q_{d}}_{d\in D}}$ where $q\in Q, q_{d}\in Q$ for
  all $d\in D$ and $\sigma\in\Sigma$. Thus, define a tree automaton
  \begin{equation*}
    \mathcal{A} = \tuple{Q, q_{0}, D, \Sigma, \Delta, \Acc\subseteq Q^{\omega}}.
  \end{equation*}
  And its semantics with the following auxillary notions
  \begin{description}
    \item[Run] a run of $\mathcal{A}$ on a $D$-ary $\Sigma$-tree $t$ is a
      $D$-ary $Q$-tree $r$ such that $r(\epsilon) = q_{0}$ and for every 
      $u\in D^{*}$ there is one 
      $\tuple{q,\sigma, \tuple{q_{d}}_{d\in D}}\in\Delta$ with $r(ud) = q_{d}$,
    \item[Acceptance Condition] the acceptance condition $\Acc$ is considered
      to be represented as Büchi-, Parity-, Muller- or Rabin-condition,
    \item[Projection of a tree] for any $\alpha=\alpha_{1}\alpha_{2}\dots\in
      D^{\omega}$ and any $D$-ary $\Sigma$-tree $t$ let $t(\alpha)$ denote the 
      word $\beta = \beta_{0}\beta_{1}\dots\in\Sigma^{\omega}$ such that for 
      every $i\in\mathbb{N}_{>0}$ holds 
      $\beta_{i} = t(\alpha_{1}\dots\alpha_{i})$ and $\beta_{0} = t(\epsilon)$.
  \end{description}
  $\mathcal{A}$ accepts a tree $t$ if there exists a run $r$ of $\mathcal{A}$
  on $t$ such that for every $\alpha\in D^{\omega}$ $r(\alpha)\in\Acc$.
\end{definition}
Again, we introduce acronyms for the considered tree automata, i.e. \ac{PTA},
\ac{RTA}, \ac{MTA} and \ac{BTA}, and present an easy example to familiarize
ourselves with the notions of tree automata
\begin{example}
  Consider $\mathcal{L}_{\exists a}$ as the language of those binary trees over 
  $\set{a,b}$, i.e. $\set{0,1}$-ary $\set{a,b}$-trees, that contain at least 
  one $a$. This language can be accepted by a tree automaton by 
  non-deterministically choosing a path to an occurence of an $a$ and rendering 
  every other path accepted. From the point onwards where an $a$ occured the
  chosen path is rendered accepted. The corresponding tree automaton can be 
  described by two states $q_{s}, q_{a}$, where $q_{s}$ is a state that 
  searches for an $a$ while $q_{a}$ is a state that renders all its subtrees 
  accepting. $q_{a}$ reproduces itself into all its subtrees, while $q_{s}$ 
  reproduces itself only in one of its subtrees and \enquote{dismissing} the 
  other one by sending $q_{a}$ in it. The acceptance condition is described as 
  a Büchi condition by the set $F = \set{q_{a}}$, thus, since $q_{a}$ only 
  produces $q_{a}$ in all following subtrees, it can be also understood as a 
  Reachability-condition, i.e. a condition which is satisfied as soon as one
  desired element occurs. This yields the tree automaton
  \begin{equation*}
    \mathcal{A}_{\exists a} = \tuple{Q = \set{q_{s}, q_{a}}, q_{s},
    \Delta ,F = \set{q_{a}}}
  \end{equation*}
  with
  \begin{equation*}
    \Delta = \set{\tuple{q_{a}, a, q_{a}, q_{a}},\tuple{q_{a}, b, q_{a}, q_{a}},
      \tuple{q_{s}, b, q_{s}, q_{a}}, \tuple{q_{s}, b, q_{a}, q_{s}},
      \tuple{q_{s}, a, q_{a}, q_{a}}}.
  \end{equation*}
  For all trees in $\mathcal{L}_{\exists a}$ the automaton
  $\mathcal{A}_{\exists a}$ can produce an accepting run by \enquote{guessing}
  a path to the occurence of $a$ while every tree accepted by
  $\mathcal{A}_{\exists a}$ does need to contain an $a$ since otherwise $q_{s}$
  is never transformed into $q_{a}$ which yields one non-accepting path in any
  run. Enforcing, that $q_{s}$ always reproduces into both subtrees yields an
  automaton which recognizes the language of those trees which do have an $a$
  on every branch. We denote this language as $\mathcal{L}_{\forall a}$.
  \label{ex:treeexa}
\end{example}
It is noteworthy that $\mathcal{A}_{\exists a}$ uses a very simple acceptance
condition, namely a Reachability-condition which is emulated by a 
Büchi-condition. The following example on the other hand uses a more elaborate
acceptance condition:
\begin{example}
  This example is similiar to e.g.
  \cite[Exercise 8.3 and the proof of Theorem 8.6]{AutoLogInfGames}.
  Let $\mathcal{L}_{\not\infty a}$ be the language of all binary trees over the
  set $\set{a,b}$ that contain on every path only finitely many $a$. Note the
  similarity to Example \ref{ex:fina} for word automata. This
  language can be recognized by tree automata with a very easy structure,
  namely there are two states $q_{a}, q_{b}$ which only are visited if the
  corresponding letter is read. Thus, we define
  \begin{equation*}
    \Delta = \set{
      \tuple{q_{a}, a, q_{a}, q_{a}},
      \tuple{q_{b}, a, q_{a}, q_{a}},
      \tuple{q_{a}, b, q_{b}, q_{b}},
      \tuple{q_{b}, b, q_{b}, q_{b}}
    }.
  \end{equation*}
  Any Rabin-, Muller- or Parity-condition as described in Example \ref{ex:fina}
  accepts the correct language since these precisely define the abscence of
  infinitely many occurences of $q_{a}$.

  For the corresponding \ac{NBA} in Example \ref{ex:fina} we used a structural
  way to model the abscence of further occurences of $a$ from one point 
  onwards.  This argument of guessing the moment when no further $a$ appears 
  cannot be used in the case of trees since every such guess involves the 
  complete subtree. Considering the tree $t:\set{0,1}^{*}\rightarrow\set{a,b}$ 
  with $t(w) = a$ if and only if $w = 1^{n}0$ for any $n\in\mathbb{N}_{>0}$ we 
  see that $t\in\mathcal{L}_{\not\infty a}$, but for all $k\in\mathbb{N}_{>0}$ 
  the subtree rooted in $1^{k}$ contains an $a$. Thus, there is no point from
  which this subtree is $a$-free, therefore this idea of guessing such a point 
  does not translate to tree automata.
\end{example}
The difficulties to define a \ac{BTA} to accept $\mathcal{L}_{\not\infty a}$ 
are inherent to the Büchi-condition. In fact it is possible to show that 
$\mathcal{L}_{\not\infty a}$ cannot be accepted by any \ac{BTA}, thus 
separating the expressiveness of tree automata as in
\begin{theorem}
  \cite[Corollary 8]{WeakDefRel} cited after
  \cite[Theorem 8.6]{AutoLogInfGames}.
  \acp{BTA} are strictly weaker than \ac{MTA} in the sense that there exists 
  a language recognizable by a \ac{MTA} but not recognizable by a \ac{BTA}.
\end{theorem}
This is in contrast to Theorem \ref{thm:omegaregularexp} where we could use the
non-determinism of an \ac{NBA} to emulate a more expressive acceptance 
condition, e.g. a Muller-condition. But by the same techniques as for the proof 
of Theorem \ref{thm:omegaregularexp} for the equivalence of the 
\enquote{strong} acceptance conditions, namely \acp{LAR}, we can ensure
\begin{theorem}
  \cite[Theorem 8.7]{AutoLogInfGames}
  \acp{MTA}, \acp{RTA} and \acp{PTA} all recognize the same class of tree
  languages.
  \label{thm:treeautoequiv}
\end{theorem}
Regarding the closure properties of this class of tree languages it is again
possible to show its closure under union, intersection and negation. This gives
\begin{theorem}
  \cite[Theorem 1.3]{SOTheoAndTrees}
  \cite[Theorem 1.5]{SOTheoAndTrees}
  The class of languages that can be accepted by \acp{MTA}, \acp{RTA} and 
  \acp{PTA} forms a Boolean-algebra. The transformations can be effectively
  constructed.
  \label{thm:treeautoboolean}
\end{theorem}
\begin{proof}[Proof-Sketch]
  The concepts follow the proof of Theorem \ref{thm:omegaregboolean}: given two
  \acp{PTA} $\mathcal{A}_{1}, \mathcal{A}_{2}$ we construct one \ac{PTA}
  $\mathcal{A}$ which performs an initial \emph{guess} deciding to execute
  either $\mathcal{A}_{1}$ or $\mathcal{A}_{2}$. Hence, if there is an
  accepting run for either $\mathcal{A}_{1}$ or $\mathcal{A}_{2}$ then there is
  an accepting run for $\mathcal{A}$ and vice versa. Therefore, $\mathcal{A}$
  accepts the union of the languages of $\mathcal{A}_{1}$ and
  $\mathcal{A}_{2}$.

  An operator for intersection is achieved by relying on a product construction
  which simulates both input automata. Using a Muller-condition which
  formulates the acceptance of the individual components by their original
  acceptance conditions completes the construction. 

  The closure for complementation is proven using \emph{alternating} tree
  automata and therefore postponed until we formally introduced alternating
  tree automata in the following section.
\end{proof}

\subsection{Alternating Tree Automata}
\label{subsec:ata}
Up to this point the used automata mirrored the structure of their inputs
very closely. Word automata generated words of states and tree automata 
generated trees of states of the same arity of their input trees. This notion 
can be relaxed for tree automata to obtain a new class of tree
automata, called \emph{Alternating Tree Automata}. Following 
\cite{AltTreeAuto} tree automata operate by sending down \enquote{copies} 
through the different paths. In the classical setting at every branching point 
every successor is awarded with one copy of a state of the automaton. The 
aforementioned relaxation lies in the idea to allow at a branching point to 
award some successor multiple states while others may not even receive 
a single one. This concept is also used to define automata that run over 
transition systems, e.g. \cite[Chapter 9]{AutoLogInfGames} and 
\cite{SynProbEnv}, where the input tree is the unrollment of the transition 
system.

Up to this point, transitions for tree automata are defined as tuples of the 
form $\tuple{q, \sigma, \tuple{q_{d}}_{d\in D}}$. This specifically defines for
every direction one single state to move along this direction. We introduce
transitions of the form $\tuple{q,\sigma,A}$ where $A$ is a non-empty subset of 
$A\subseteq D\times Q$ called a \emph{clause}.

Semantically, $A$ describes that a copy of $q$ is sent down the direction $d$ 
for every $\tuple{d,q}\in A$. This notion allows us to refine the tree 
automaton of Example \ref{ex:treeexa}:
\begin{example}
  Recall $\mathcal{A}_{\exists a}$ from Example \ref{ex:treeexa} which
  determines if a tree contains an $a$ with 
  \begin{equation*}
    \Delta = \set{
      \tuple{q_{a}, a, q_{a}, q_{a}},
      \tuple{q_{a}, b, q_{a}, q_{a}},
      \tuple{q_{s}, a, q_{a}, q_{a}},
      \tuple{q_{s}, b, q_{s}, q_{a}},
      \tuple{q_{s}, b, q_{a}, q_{s}}
    }.
  \end{equation*}
  The concept of alternating tree automata allows to formulate the
  \enquote{searching} part of the run more concisely\footnote{
    Occasionally, e.g. in \cite[Chapter 9]{AutoLogInfGames}, alternating tree
    automata are equipped with the possibility to render a subtree accepted 
    regardless of its structure. We explicitly do not include this notion, but 
    it is easy to see that this would allow to equivalently define transitions
    $\tuple{q_{a}, a, \delta_{\top}},\tuple{q_{a}, b, \top},
    \tuple{q_{s}, a, \top}$ where $\top$ renders the subtree accepted.
  }
  by dropping the necessity to send a $q_{a}$ along the
  path that is not chosen, thus obtaining
  \begin{equation*}
    \tuple{q_{s}, b, \set{\tuple{0, q_{s}}}},
    \tuple{q_{s}, b, \set{\tuple{1, q_{s}}}}.
  \end{equation*}
\end{example}
We formally define similar to \cite{SynProbEnv}
\begin{definition}[Alternating Tree Automata]
  For a finite set of states $Q$, one distinct intial state $q_{0}\in Q$, a
  finite set of directions $D$, a finite alphabet $\Sigma$ and a transition
  relation $\Delta$ we define 
  \begin{equation*}
    \mathcal{A} = \tuple{Q, q_{0}, D, \Sigma, \Delta, \Acc\subseteq Q^{\omega}}.
  \end{equation*}
  The semantics of $\mathcal{A}$ is given by a run $r$ of $\mathcal{A}$ on a
  $D$-ary $\Sigma$-tree $t$. Here $r$ is a $D\times Q$-ary $\set{0,1}$-tree 
  with the following neccessities:
  \begin{enumerate}
    \item $r(\epsilon) = 1$,
    \item for every $u = \tuple{d_{1}, q_{1}}\dots\tuple{d_{n}, q_{n}}\in
      \tuple{D\times Q}^{*}$ with $r(u) = 1$ there exists
      $\tuple{q_{n}, t(d_{1}\dots d_{n}), A}\in\Delta$ such that
      \begin{equation*}
        r(u\tuple{d, q}) = 1\text{ if and only if }\tuple{d, q}\in A.
      \end{equation*}
  \end{enumerate}
  We consider such a run $r$ accepting if the $Q$-component of all paths that 
  are consistently marked are part of $\Acc$. Formally, we fix for any run $r$ 
  the set of its marked path as
  \begin{equation*}
    M_{r} = \set{
      \rho\in\tuple{D\times Q}^{\omega}\mid r(v) = 1\text{ for all finite }
        v\sqsubseteq\rho
    }.
  \end{equation*}
  The set of accepted paths is given by
  \begin{equation*}
    \Acc\interval{Q} = \set{
      \tuple{d_{1}, q_{1}}\tuple{d_{2}, q_{2}}\dots\in
      \tuple{D\times Q}^{\omega}\mid q_{1}q_{2}\dots\in\Acc
    }.
  \end{equation*}
  A run $r$ is considered accepting if $M_{r}\subseteq\Acc\interval{Q}$.
  Again, we consider for $\Acc$ acceptance conditions which are representable
  as Büchi-, Rabin-, Muller- or Parity-conditions. We introduce according
  acronyms, namely \ac{ABTA}, \ac{ARTA}, \ac{AMTA} and \ac{APTA} respectively.
\end{definition}
By using the same argument of using \acp{LAR} as for Theorem 
\ref{thm:omegaregularexp} and Theorem \ref{thm:treeautoequiv} we obtain
\begin{theorem}
  \ac{ARTA}, \ac{AMTA} and \ac{APTA} recognize the same class of tree
  languages.
  \label{thm:atreeautoequiv}
\end{theorem}
Although \emph{alternation} is an extension of the \enquote{classic} tree 
automata it turns out that it does not increase the expressiveness.
Specifically, for every alternating tree automta there is a classic tree
automata which accepts the same language of trees.
\begin{theorem}[The Simulation Theorem]
  \cite[Theorem 1.2]{SimAltTreeAuto}
  There is an effective construction which, when given an \ac{AMTA}, produces 
  an equivalent \ac{MTA}. Furthermore, given an \ac{ABTA}, there is a way to 
  effectively construct an equivalent \ac{BTA}.
  \label{thm:treesimulation}
\end{theorem}
This result and the apparent observation that alternation does not weaken the 
formalism to accept trees allows to translate the closure properties stated in 
Theorem \ref{thm:treeautoboolean} to alternating tree automata as well. Thus, 
we obtain that there are effectively constructable transformations for the 
union, intersection and complement of tree languages accepted by \acp{AMTA}. 
Nevertheless, we want to present one construction for the closure under 
intersection because it allows us to relate concepts of \acp{PBA} and of
alternating tree automata.
\begin{corollary}
  The class of languages of infinite trees that can be recognized by \acp{AMTA}
  is closed under intersection.
  \label{cor:treeintersection}
\end{corollary}
\begin{drawing}
  \caption{Beginning from $q_{0}$ two run-trees unfold independently for an
  alternating tree automata. If we consider this as the runs of a union or
  intersection \ac{PBA} both runs are tuned down by the factor $\frac{1}{2}$.}
  \label{fig:parallelruns}
  \begin{center}
    \includegraphics{tikz/parallelruns.pdf}
  \end{center}
\end{drawing}
\begin{proof}[Alternative Proof]
  Given two \acp{AMTA}
  \begin{equation*}
    \mathcal{A}_{1} = \tuple{Q, q_{0}, D, \Sigma, \Delta_{1}, \mathcal{F}_{1}}
      \text{ and }
    \mathcal{A}_{2} = \tuple{P, p_{0}, D, \Sigma, \Delta_{2}, \mathcal{F}_{2}}
  \end{equation*}
  we construct an \ac{AMTA} for their intersection. W.l.o.g. we assume that 
  $Q\cap P = \emptyset$. The idea is to perform two independent runs on the 
  tree in parallel. We illustrate this in Figure \ref{fig:parallelruns}. 
  Formally, it is accomplished by initially dispatching one transition of each 
  automaton onto the tree. After this initial dispatch both state sets $Q$ and
  $P$ operate independently on the tree. Hence, we define
  \begin{equation*}
    \mathcal{A}_{\cap} = \tuple{
      Q\cup P\uplus\set{q^{i}}, q^{i}, D, \Sigma, 
      \Delta_{1}\cup\Delta_{2}\cup\Delta_{\cap}, 
      \mathcal{F}_{1}\cup\mathcal{F}_{2}
    }
  \end{equation*}
  with
  \begin{equation*}
    \Delta_{\cap} = \set{
      \tuple{q^{i},\sigma,A\cup B}:
        \tuple{q_{0},\sigma,A}\in\Delta_{1}
      \text{ and }
        \tuple{p_{0},\sigma,B}\in\Delta_{2}
      \text{ for every }
        \sigma\in\Sigma
    }.
  \end{equation*}
  Thus, we can construct from the resulting run $r$ in $\mathcal{A}_{\cap}$ two
  individual runs in $\mathcal{A}_{1}$ and $\mathcal{A}_{2}$ respectively. For
  $\mathcal{A}_{1}$ we obtain a run $r_{1}$ from the $D\times Q$-part of the
  domain of $r$ and the $D\times P$-part of the domain of $r$ induces a run
  $r_{2}$ for $\mathcal{A}_{2}$.
\end{proof}

While considering the semantics of \acp{PBA} we used a notion of considering 
individual runs and their probabilities. This can be captured very similar to 
the concept of alternation as follows: we can regard any word 
$\alpha\in\Sigma^{\omega}$ as a $\set{0}$-ary $\Sigma$-tree $t_{\alpha}$. Then 
one \ac{PBA} $\mathcal{P}$ induces a run in a tree-sense by unrolling the state
sequences of $\mathcal{P}$ around the unary tree $t_{\alpha}$. Hereby, we can
substitute clauses which have a very absolute semantic (either a
path is taken or it is not) by a probability distribution.
For a \ac{PBA} $\mathcal{P} = \tuple{Q, \Sigma, \delta, q_{0}, F}$ we introduce
transition functions $G_{\sigma}^{q}:\set{0}\times Q\rightarrow\interval{0,1}$
such that $G_{\sigma}^{q}(0, p) = \delta(q,\sigma, p)$ and fix
\begin{equation*}
  \mathcal{G} = \set{G_{\sigma}^{q}:\sigma\in\Sigma,q\in Q}.
\end{equation*}
We can model the run of $\mathcal{P}$ as a $\set{0}\times Q$-ary
$\mathcal{G}$-tree where the chosen $G_{\sigma}^{q}$ one the paths
$\tuple{0,q_{1}}\dots\tuple{0,q_{n}}$ are consistent with the current state
($q = q_{n}$) and input symbol ($\sigma = t(0^n)$). Therefore, every path
corresponds to one state sequence of $\mathcal{P}$ and the visited
$G_{\sigma}^{q}$ on the path induce the associated probabilities for the
peformed steps. In this sense we can use Figure \ref{fig:parallelruns} as
illustration for the union operator of the proof of Theorem
\ref{thm:pbaboolalgebra}. Thereby, the initial separating dispatch is in every
direction tuned down by a factor of $\frac{1}{2}$. This intuition is later on
re-visited and formally substantiated.

In the meantime we introduce the concept of \emph{graph games} 
to obtain a transformation that constructs the \ac{APTA} which accepts the 
negation of the language of a given \ac{APTA}.

\subsection{Graph Games}
Tree automata are closely related to graph games (cp.
\cite[Chapter 9]{AutoLogInfGames}). Such games are held between two players in
an \emph{arena}. An arena is a directed graph where the nodes are partioned
into two sets; each set is associated with one player. A game on such an arena
begins in one node and unfolds by the choices of the players. The player to
whom the current node $v$ belongs may choose the node $u$ the game proceeds to
from the set of those nodes connected to $v$. By associating a winning
condition for one player to the game we can call a game won by this player if
the played state sequence belongs to this winning condition. In the following 
we focus on zero-sum games, i.e. if the winning condition of one player is not 
met we consider the opponent victorious. This leads to the definition
\begin{definition}[Graph Games]
  \begin{description}
    \item [Arena] Let $V$ be a set of nodes which is partioned into $V_{0}$ and
      $V_{1}$ and $E\subseteq V\times V$ a set of edges, then we define an
      arena $\tuple{V, V_{0}, V_{1}, E}$.
    \item [Game] For an arena $\tuple{V, V_{0}, V_{1}, E}$ we define with a
      winning condition $\Acc$ and one distinguished initial vertex
      $v_{0}\in V$ the game
      $\tuple{\mathcal{G}, v_{0}} = \tuple{V, V_{0}, V_{1}, E, \Acc, v_{0}}$.
    \item [Play] A play for a game $\tuple{\mathcal{G}, v_{0}} = \tuple{V,
      V_{0}, V_{1}, E, \Acc}$ is any sequence $v_{0}v_{1}\dots\in
      V^{\omega}$ which starts in $v_{0}$ and where for every $i > 0$ holds
      that $\tuple{v_{i},v_{i-1}}\in E$.
    \item [Strategy] For $\sigma\in\set{0,1}$ we define a strategy
      $f_{\sigma}: V^{*}V_{\sigma}\rightarrow V$ where for any $w\in V^{*}$ and
      $v\in V_{\sigma}$ $f_{\sigma}(w\cdot v)\in vE$ (where $vE =
      \set{u\in V\mid\tuple{v,u}\in E}$). A play is considered winning for
      player $0$ if it belongs to $\Acc$. Notably it holds that two strategies
      $f_{0}, f_{1}$ induce a play $v_{0}v_{1}\dots$ by fixing for any $v\in V$
      the term
      \begin{equation*}
        \sigma(v) = \begin{cases}
          0&\text{if }v\in V_{0}\\
          1&\text{if }v\in V_{1}
        \end{cases}\text{ and enforcing }
        v_{i+1} = f_{\sigma(v_{i})}(v_{0}\dots v_{i}).
      \end{equation*}
      If a strategy for player $\sigma$ can equivalently be represented as a
      function $f:V_{\sigma}\rightarrow V$ (naturally still satisfying that
      $f(u)\in uE$ for all $u\in V_{\sigma}$) we call this strategy
      \emph{memoryless} or \emph{positional}.
    \item [Winning] We say a play $v_{0}v_{1}\dots$ is \emph{consistent} with a
      strategy $f$ for player $\sigma$ if for every $i$ with
      $v_{i}\in V_{\sigma}$ holds that $v_{i+1} = f(v_{0}\dots v_{i})$. 
      A strategy $f$ for player $\sigma$ is called a \emph{winning strategy} if
      \emph{every} play consistent with $f$ is winning for player $\sigma$.
      We denote with \emph{winning regions}
      $\mathcal{W}^{\mathcal{G}}_{\sigma}$ those nodes $v\in V$ such that
      player $\sigma$ has a winning strategy in $\tuple{\mathcal{G}, v}$.
  \end{description}
  $\Acc$ is represented in finite ways by using Büchi-, Rabin-, Muller- or
  Parity-conditions. In the following we call the player for whom the
  acceptance condition $\Acc$ is stated \eve{} (as for \acp{MDP}) and her
  opponent \adam{} (in the following we treat \adam{} grammatically as male).
\end{definition}
Given any game $\mathcal{G}$ the question arises whether for every node $v$
either \eve{} or \adam{} wins by playing the correct strategy (regardless of
the strategy of the opponent), i.e. 
$\mathcal{W}^{\mathcal{G}}_{0}\cup\mathcal{W}^{\mathcal{G}}_{1} = V$. We call
a game with this property \emph{determined}. There is a fundamental result by
Martin establishing determinacy for all Borel winning conditions for a more
general class of games\footnote{Gale-Stewart games} \cite{BorelDeterminacy}.
Fortunately, this translates to graph games and allows us to state using that
Büchi-, Rabin-, Muller- and Parity-conditions are Borel (see Lemma
\ref{lem:measureabilityAcceptance} and its Corollary \ref{cor:borelAcceptance})
\begin{theorem}
  \cite[Corollary 2.10]{AutoLogInfGames}
  All graph games with Büchi-, Rabin-, Muller- or Parity-condition are 
  determined.
  \label{thm:boreldet}
\end{theorem}
This means that given for every strategy $f$ of \eve{} \adam{} has a strategy
$g$ which induces a play not in $\Acc$ \adam{} can find \emph{one} strategy
$g'$ which is winning against \emph{all} strategies $f'$ of \eve{}.
For Parity-conditions we even obtain that the corresponding strategies are 
positional, i.e.
\begin{theorem}
  \cite[Theorem 6]{ParityGamesPosDet}
  For a graph game with a Parity-condition with finitely many parities both 
  players have positional winning strategies on their corresponding winning 
  regions.
  \label{thm:posdetparity}
\end{theorem}

As mentioned above we use graph games to model the semantics of tree automata.
Therefore, we define an acceptance game for an automaton and a tree. In this
(infinite) graph game \eve{} constructs a run of the automaton on the 
tree and \adam{} explores this run and tries to find a path which is not 
accepted. Formally, we restrict ourselves to \ac{APTA} since they are 
sufficiently expressive and define (cp. \cite{treeautosurvey,AltTreeAuto})
\begin{definition}[Tree Acceptance Game]
  \label{def:acceptancegame}
  For a \ac{APTA} $\mathcal{A} = \tuple{Q, q_{0}, D, \Sigma, \Delta, \parity}$
  and a $D$-ary $\Sigma$-tree $t:D^{*}\rightarrow\Sigma$ we define the graph 
  game $\mathcal{G}(\mathcal{A}, t) = \tuple{V, V_{0}, V_{1}, E, 
  \parity', \tuple{q_{0},\epsilon}}$ with
  \begin{description}
    \item [Arena] 
      $V_{0} = Q\times D^{*}, V_{1} = \Pot(D\times Q)\times D^{*}$ and
      $V = V_{0}\cup V_{1}$,
    \item [Edges]
      \eve{} chooses the taken transition which is captured by
      \begin{equation*}
        E_{0} = \set{
          \tuple{\tuple{q, u},\tuple{A, u}}:\text{for all }
          q\in Q,u\in D^{*}\text{ and }\tuple{q, t(u), A}\in\Delta
        }.
      \end{equation*}
      \adam{} on the other hand chooses which element in $A$ is taken, i.e.
      \begin{equation*}
        E_{1} = \set{
          \tuple{\tuple{A, u}, \tuple{q, u\cdot d}}:\text{ for all }
          A\in\Pot(D\times Q), u\in D^{*}\text{ and }\tuple{q,d}\in A
        }.
      \end{equation*}
      Naturally, we set $E = E_{0}\cup E_{1}$.
    \item [Winning Condition]
      the associated winning condition is the projection of the old 
      condition to the state component for nodes in $V_{0}$, i.e.
      \begin{equation*}
        \parity'(\tuple{q, u}) = \parity(q).
      \end{equation*}
      Nodes in $V_{1}$ should not impact the evaluation of a game directly, 
      therefore we set their parities lower than all parities used in 
      $\mathcal{A}$:
      \begin{equation*}
        \parity'(\tuple{\delta_{A}, u}) = p_{\bot}
        \text{ for one }p_{\bot} < \min\set{\parity(q):q\in Q}.
      \end{equation*}
  \end{description}
  \label{def:treemembershipgame}
\end{definition}
An illustration of such a game can be found in Figure 
\ref{fig:treeacceptancegame}. 

\begin{drawing}
  \caption{
    Illustration of $\mathcal{G}(\mathcal{A}, t)$ for $\mathcal{A}$ as
    defined in Example \ref{ex:treeexa} and a $\set{0,1}$-ary $\set{a,b}$-tree
    $t$ with $t(u) = a$ if and only if $u = 00$. Nodes of player $0$ are 
    circles while player $1$ plays on rectangles. Every triangle is 
    associated with one word $u\in\set{0,1}^{*}$ and labeled with $t(u)$. 
    Player $0$ may win by moving towards the $t(00)$ subgame, which is easily
    achievable since player $1$ only has trivial moves up to that point.
  }
  \label{fig:treeacceptancegame}
  \begin{center}
    \resizebox{\textwidth}{!}{%
      \includegraphics{tikz/treeacceptancegame.pdf}%
    }
  \end{center}
\end{drawing}

The central insight for this game is the following lemma which connects the 
acceptance of $\mathcal{A}$ of $t$ with the strategies of \eve{} in 
$\mathcal{G}(\mathcal{A}, t)$.
\begin{lemma}
  \cite[Lemma 3.1.]{AltTreeAuto}
  The automaton $\mathcal{A}$ accepts $t$ if and only if \eve{} has a 
  winning strategy in the game $\mathcal{G}(\mathcal{A}, t)$.
  \label{lem:stratacceptance}
\end{lemma}
In the following we present a proof for this lemma which reveals the 
central idea behind the construction of a dual automata for a given \ac{APTA}.
\begin{proof}
  We can connect strategies $f$ for \eve{} with runs $r$ of the automaton 
  $\mathcal{A}$ on $t$ directly. Initially, \eve{} starts in state 
  $\tuple{q_{0}, \epsilon}$ and chooses an initial clause $A$ (by her movement
  to $\tuple{A, \epsilon}$ which by definition of $E_{0}$ corresponds to a 
  transition $\tuple{q_{0}, t(\epsilon), A}$. This $A$ can be used to define 
  the first level of a run $r$ of $\mathcal{A}$ on $t$ by setting 
  $r(\tuple{q,d}) = 1$ if and only if $\tuple{q,d}\in A$. \adam{} does now 
  choose which element of $A$ to explore (by moving to $\tuple{q, d}$). This 
  can be identified by \enquote{moving} to $r(q, d)$. Inductively, \eve{}'s 
  choices correspond to chosen transitions in $r$ and \adam{} chooses the 
  precise marked paths to explore. Any combination of strategies $f$ and $g$
  for \eve{} and \adam{} respectively yields a play which corresponds to one 
  path in $M_{r}$ for one run $r$ (especially $r$ is induced by $f$ and all
  possible strategies of \adam{}). \eve{} wins if this chosen path satisfies
  the Parity-condition on the $Q$-component (notably because the definition of
  the Parity-condition $\mathcal{G}(\mathcal{A}, t)$ ignores nodes of $V_{1}$).
  Considering a winning strategy $f$ for \eve{} any opposing strategy of 
  \adam{} is loosing rendering all elements in the corresponding $M_{r}$ 
  accepted. On the other hand, if there is no winning strategy $f$ for \eve{}
  the induced run $r$ and the corresponding $M_{r}$ always contains at least 
  one not accepted element (which corresponds to the choices \adam{} may take 
  which lead to \eve{}'s loss).
\end{proof}

Notably, Theorem \ref{thm:boreldet} in combination with Lemma
\ref{lem:stratacceptance} allows to state that $\mathcal{A}$ does not accept
$t$ if and only if \adam{} has a winning strategy in
$\mathcal{G}(\mathcal{A},t)$. Constructing a dual automaton to $\mathcal{A}$
resolves around the idea to encode \adam{}'s strategies in the transitions 
rather than \eve{}'s. Thus, given the game is in state $q$ and $t$ carries at 
the current position the letter $\sigma$ \eve{} may choose according to any 
transition of $\mathcal{A}$ for this situation, namely 
$\tuple{q,\sigma,A_{1}},\dots,\tuple{q,\sigma,A_{n}}$. 
\adam{}'s strategy needs to state for every choice of \eve{} one element in the 
chosen $A_{i}$ for $1\leq i\leq n$ such that he wins the corresponding play. We
construct \adam{}'s choices as
\begin{equation*}
  R_{q,\sigma} = \set{
    \set{
      \tuple{d_{1}, q_{1}},\dots,\tuple{d_{n},q_{n}}\mid
      \tuple{d_{i}, q_{i}}\in A_{i}\text{ for }1\leq i\leq n
    }
  }.
\end{equation*}
We obtain for a set of transitions $\Delta$ the \enquote{inverse} $\nabla$ by
\begin{equation*}
  \nabla = \set{
    \tuple{q, \sigma, B}:q\in Q, \sigma\in\Sigma, B\in R_{q,\sigma}
  }.
\end{equation*}
We use the transitions in $\nabla$ to effectively describe the game 
$\mathcal{G}(\mathcal{A}, t)$ with exchanged roles (in the sense of which 
\enquote{type} of moves are executed) of \eve{} and \adam{}. \adam{} 
pro-actively chooses his answers (which is encoded as transition) and \eve{} 
choice of one direction corresponds to an implicit choice of a transition, 
namely the one to which \adam{} answers by this movement. Considering any 
transition in $\nabla$ we observe that for every possible choice of \eve{} 
in parallel one answer of \adam{} is explored. This captures the usage of
determinacy by exploring all of \adam{}'s option since a winning strategy 
produces for all choices of \eve{} a winning play, hence choosing the set of 
answers that are initial for winning plays allows to encode \adam{}'s winning
strategy. We construct for a given 
\ac{APTA}
\begin{equation*}
  \mathcal{A} = \tuple{Q, q_{0}, D, \Sigma, \Delta, \parity},
\end{equation*}
the automaton
\begin{equation*}
  \mathcal{C} = (Q, q_{0}, D, \Sigma, \nabla, \parity + 1).
\end{equation*}
Assuming \adam{} does have a winning strategy in $\mathcal{G}(\mathcal{A}, t)$, 
then \eve{}\footnote{Beware the role-exchange!} has a winning strategy in the 
game $\mathcal{G}(\mathcal{C}, t)$ since $\parity + 1$ precisely inverts the
set of accepted paths. Technically, any run explores all possible plays of 
\adam{} in $\mathcal{G}(\mathcal{A}, t)$ while the choices of $\eve{}$ induce
one path in this run. Therefore $\mathcal{C}$ is the dual automaton to 
$\mathcal{A}$ implying
\begin{proposition}
  The class of languages of trees accepted by \acp{APTA} is closed under
  negation. This holds by Theorem \ref{thm:treesimulation} also for \acp{PTA}.
  \label{prop:atanegation}
\end{proposition}

Concludingly, we want to examine the problem of deciding emptiness of tree
automata. Again, by the effective transformations between alternating and
non-alternating tree automata (Theorem \ref{thm:treesimulation}) it is
sufficient to examine one of those types. We opt for non-alternating tree
automata and discuss the reason for this after introducing a game to capture
the emptiness of tree automata. Formally, we follow 
\cite[Proof of Theorem 23]{ParityGamesPosDet} when considering for a given
\ac{PTA}
\begin{equation*}
  \mathcal{M} = \tuple{Q, q_{0}, D, \Sigma, \Delta, \parity}
\end{equation*}
the following graph game
\begin{definition}[Emptiness Game]
  \eve{} chooses transitions depending on the current state, while
  \adam{} chooses the direction of a transition. Therefore
  we define the emptiness game by the following notions.
  \begin{description}
    \item [Arena] \eve{} is assigned the states $Q = V_{0}$ of 
      $\mathcal{M}$, while \adam{} operates on the transitions 
      $\Delta = V_{1}$ of $\mathcal{M}$. The edges correlate to the choices
      both players make, i.e. $E = E_{0}\cup E_{1}$ with
      \begin{equation*}
        E_{0} = \set{\tuple{q, \delta}\in Q\times\Delta\mid\delta = 
          \tuple{q, \sigma, \tuple{p_{d}}_{d\in D}}\in\Delta}
      \end{equation*}
      and
      \begin{equation*}
        E_{1} = \set{\tuple{\tuple{q, \sigma, \tuple{p_{d}}_{d\in D}}, p}
          \in\Delta\times Q\mid\text{there is }b\in D
          \text{ such that }p_{b} = p
        }.
      \end{equation*}
      We obtain with $V = V_{0}\cup V_{1}$ an arena 
      $\tuple{V, V_{0}, V_{1}, E}$.
    \item [Winning Condition] \eve{} wins the game if the explored state 
      sequence is accepted in terms of $\parity$, hence we define for any
      $\bot < \min\set{\parity(q):q\in Q}$
      \begin{equation*}
        \parity'(\Gamma) = \begin{cases}
          \parity(q) &\text{if }\Gamma = q\in Q,\\
          \bot       &\text{otherwise}.
        \end{cases}
      \end{equation*}
  \end{description}
  We define the emptiness game as 
  $\mathcal{G} = \tuple{V, V_{0}, V_{1}, E, \parity', q_{0}}$.
  \label{def:treeemptinessgame}
\end{definition}
This game is similar to the one from Definition \ref{def:treemembershipgame}, 
but instead of choosing a transition for the corresponding letter in a tree 
\eve{} is free to choose any transition (and therefore implicitly a letter).
The central insight for the emptiness game is formulated in
\begin{lemma}
  \eve{} has a winning strategy in $\mathcal{G}$ as defined above
  if and only if the language of $\mathcal{M}$ is not empty.
  \label{lem:evewinningemptiness}
\end{lemma}
\begin{proof}
  Initially, we observe that for any play $q_{0}\delta_{0}q_{1}\delta_{1}\dots$ 
  we obtain by definition of $\parity'$ that $\max\set{\parity(q):q\in\Inf(
  q_{0}q_{1}q_{2}\dots)} = \max\set{\parity'(\Gamma):\Gamma\in\Inf(
  q_{0}\delta_{0}q_{1}\delta_{1}\dots)}$.

  Consider a winning strategy $f:V^{*}Q\rightarrow\Delta$ for \eve{},
  and an arbitrary riposte $g:V^{*}\Delta\rightarrow Q$ for \adam{}. By
  the definition of $E_{1}$ we may equivalently define the strategy of 
  \adam{} in terms of $g:V^{*}\Delta\rightarrow D$. In this form, we 
  obtain for $f$ an associated tree $t_{f}:D^{*}\rightarrow\Sigma$ by
  considering for any $u_{0}\dots u_{n}\in D^{*}$ the play 
  $q_{0}\delta_{0}q_{1}\delta_{1}\dots$ which is consistent with $f$ and the
  riposte $g$ with $g(q_{0}\delta_{0}\dots q_{i}\delta_{i}) = u_{i}$ for 
  $0\leq i\leq n$ and arbitrary otherwise. Naturally, we set 
  $t(u_{0}\dots u_{n}) = \sigma$ for one $\sigma$ such that 
  $f(q_{0}\delta_{0}\dots q_{n}) = \delta_{n} 
    = \tuple{q_{n}, \sigma, \tuple{p_{d}}_{d\in D}}\in\Delta$. By the 
  construction of $t$ we see that the choices of $f$ immediately describes a 
  possible run $r$ on $t$. Since $f$ is winning all paths \adam{} can
  describe are won by \eve{} (accepted under $\parity'$) and therefore 
  all paths of $r$ are accepting in terms of $\parity$, rendering $r$ an
  accepting run for $t$ of $\mathcal{M}$.

  If we have on the other hand, an accepted tree $t$ of $\mathcal{M}$ with
  a corresponding run $r$ then \eve{} simply plays $r$, i.e. we define
  $f_{r}$ with $f_{r}(q_{1}\delta_{0}\dots q_{n}) = \delta_{n} = r(u)$ for 
  the $u\in D^{*}$ that corresponds to the choices of \adam{} which led 
  to the play $q_{0}\delta_{0}\dots q_{n}$. Again, all paths in $r$ are 
  accepting in terms of $\parity$ rendering all plays consistent with $f_{r}$ 
  accepted in terms of $\parity'$. Hence, $f_{r}$ is a winning strategy.
\end{proof}

This lemma directly allows us to state
\begin{theorem}
  \cite[Theorem 21]{AutoInfObj} \cite[Theorem 7.25]{AutoLogInfGames}
  The problem if the language of a given \ac{PTA}
  \begin{equation*}
    \mathcal{M} = \tuple{Q, q_{0}, D, \Sigma, \Delta, \parity}
  \end{equation*}
  is empty can be decided in time in 
  $\mathcal{O}(\size{\parity(Q)}\cdot\size{Q\times Q}\cdot\tuple{
    \frac{\size{Q}}{d}}^{d})$ with 
  $d = \lfloor \frac{\size{\parity(Q)}}{2} \rfloor$.
  \label{thm:emptinessPTA}
\end{theorem}
\begin{proof}
  The stated complexity relates to the complexity of computing winning regions
  in parity games. Using Lemma \ref{lem:evewinningemptiness} directly induces
  the statement.
\end{proof}

By Theorem \ref{thm:posdetparity} we even obtain a positional winning strategy 
for \eve{} (if there is any winning strategy)\footnote{
  This means that the structure of the obtained tree $t$ indeed is - to a
  certain degree - simple. Theorem 23 in \cite{ParityGamesPosDet} calls such a
  tree $t$ \emph{regular} which means that the number of isomorphic subtrees in
  $t$ is finite. This is similar to ultimatively periodic words for
  $\omega$-regular languages.
}.

Note that we designed this emptiness game in terms of \ac{PTA} rather than 
\ac{APTA}. This is due to the fact that the choices of \eve{} must be 
consistent with an underlying tree (cp. Example \ref{ex:pbaasmdp}). We 
illustrate this problem in
\begin{example}
  Consider an \ac{APTA}
  \begin{equation*}
    \mathcal{A} = \tuple{Q, q_{0}, D, \Sigma, \Delta, \parity}
  \end{equation*}
  with $Q = \set{q_{0}, q_{a}, q_{b}, \bot}$, $D = \set{0,1}$, 
  $\Sigma = \set{a, b}$, $\parity(q_{0}) = \parity(q_{b}) = \parity(q_{a}) 
    = 0$ and $\parity(\bot) = 1$. The conceptual idea is that $q_{b}$ ($q_{a}$) 
  must move along a path that solely contains of $b$ ($a$) but any occurence of 
  $a$ ($b$) yields a move to the non-accepting sink-state $\bot$ which 
  reproduces itself towards all directions (necessarily rendering these paths
  rejecting). Hence, we add 
  \begin{equation*}
    \set{
      \tuple{q_{\sigma}, \sigma, \set{\tuple{q_{\sigma, d}}}},
      \tuple{q_{\sigma}, \overline{\sigma}, \set{\tuple{\bot, 0}, 
        \tuple{\bot, 1}}}:\text{for all }\sigma\in\set{a,b}, d\in\set{0,1}
    }
  \end{equation*}
  to $\Delta$ where $\overline{a} = b$ and $\overline{b} = a$. Initially, 
  $q_{0}$ dispatches $q_{a}$ and $q_{b}$ in the \emph{same} direction (by 
  arbitrary choice) $0$, yielding
  \begin{equation*}
    \set{
      \tuple{q_{0}, \sigma, \set{\tuple{q_{a}, 0}, \tuple{q_{b}, 0}}}:
        \text{for }\sigma\in\set{a, b}
      }
  \end{equation*}
  as part of $\Delta$. Naturally, there is \emph{no} possible accepted tree for 
  $\mathcal{M}$ since $t(0)$ is either $a$ or $b$ but each choice leaves one
  unsatisfied state $q_{b}$ or $q_{a}$ respectively. On the other hand, if we
  naively translate the concept of the emptiness game (Definition 
  \ref{def:treeemptinessgame}) \eve{} still can win by constructing 
  different trees depending on the fact if $q_{a}$ or $q_{b}$ is chosen by 
  \adam{} in the intial step. For \acp{PBA} we addressed this problem 
  with partial observability (see Example \ref{ex:pbaaspomdp}). But for 
  \acp{PBA} \eve{} only has one choice to make, namely one letter for the 
  current position in the word. Here \eve{} is entitled to two choices: which
  letter to choose and which transition to take for this letter and the current
  state. This requires for a dedicated knowledge management for \eve{} because
  it has to be enforced that \eve{} chooses letters consistent with a tree 
  structure and notably independent from the current state while the choice
  of the transition is highly dependent on the current state.
  \label{ex:alternatingtreeemptiness}
\end{example}
