Relaxing the notion of acceptance for word automata on infinite words is a
relatively recent approach \cite{PBAcoin,Groesser,RecOmeLangProbAuto,%
DecProblemsForProbAuto}. Valuation of runs of automata is also used for
weighted automata \cite{MaxSumSemanticsWeightedAutomata} and explored in the
context of qualitative satisfaction rather than absolute correctness
\cite{DecWeightedAutomata} which is also the basis for synthesis results in
stochastic environments \cite{HighQualSynStochEnv}.

The probabilistic approach is translated to trees in \cite{RandAutoInfTrees,%
QualTreeLang} in two different ways. On the one hand, a weighting in form of
probabilities on paths is considered and, on the other hand, also it is
examined to choose transitions probabilistically which is semantically closer
to the approach of probabilistic word automata. Relaxation on the cardinality
of accepting paths in runs of tree automata is also proposed in
\cite{CardinalsForTreePaths} by considering runs accepting if the cardinality
of accepted paths exceeds or deceeds a given cardinal number. The required
argumentation presents in a more logic-based and set-theoretical framework.

Furthermore, notions of imperfect information are broadly applied in different
settings. Used in graph games partial observability lifts the computation of
winning regions even for simple winning conditions, e.g. Safety-conditions,
into exponential complexity \cite{PowerOfImperfectInformation}. Also, partial
observability is very prominently used in \aclp*{MDP} \cite{RandomnessForFree,%
ActingOptimallyInPOSD, QualAnaPOMDP} but entails harsh consequences on the
analysis, e.g. computing almost-surely winning strategies for Parity-objectives
becomes undecidable for partial observability while there are polynomial
solutions in the case of complete information \cite{QualAnaPOMDP,%
QuanStochParityGames, ContrSynProbSys}. This entails interesting consequences
for our model of tree automata since we are able to pinpoint a structural
property of alternation in \aclp*{WDTA} that precisely corresponds with partial
observability and therefore translates these results into our automata model.
This is also suggested in \cite{ChurchsProblemRevisited} where alternating tree
automata are used to model information that are inaccessible for the actor, but
still are part of the specification.

Moreover, considering probabilistic automata as representation for the
specification of a synthesis problem is not yet widely adopted. Logics that
incorporate quantification for a probability, e.g. \textsc{PCTL}, state the
probability of the the considered environment to satisfy a condition and do not
use stoachastic behavior inherently to the specification \cite{PrinciplesOfMC}.
Using probabilistic automata as specification is explored for timed models
\cite{VerificationAndControl} or as concise representation of $\omega$-regular
properties \cite{PBAforLTLSafety}. To the best of our knowledge we pioneer in
the approach to use the incomparable expressibility of almost-surely accepting
\aclp*{PBA} to $\omega$-regular languages for specifications of synthesis
problems. Actually, the solution for qualitative strategy synthesis in
\aclp*{POMDP} for a specification given as almost-surely accepting \aclp*{PBA}
fits well to the approach in \cite{RandomnessForFree} where reductions are
examined that allow to remove probabilistic behavior for strategies or
transitions in a graph by deterministic behavior giving the associated
randomness for \enquote{free}. In this sense we allow for \enquote{free}
randomness in the specification considering \aclp*{POMDP}.
