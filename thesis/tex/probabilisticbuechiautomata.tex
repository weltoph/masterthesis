\section{Probabilistic Büchi Automata}
Upon the concept of \acp{NBA} there are approaches to substitute the
non-deterministic choices by probabilistic ones, e.g.
\cite{RecOmeLangProbAuto,DecProblemsForProbAuto,Groesser}. We begin with
a formal introduction of an automaton model which makes use of this idea,
called \aclp{PBA}. Then we proceed by introducing the necessary concept of
measuring theory to formally define their semantics. Additionally, these basic
concepts of measuring theory are re-used later on for more complex
probabilistic automata models.

A \ac{PBA} essentially is a \ac{NBA} where the non-determinism is solved by a 
probability distribution. Therefore, we define analogously to 
\cite{Groesser}
\begin{definition}[Probabilistic Büchi Automata]
  A \acl{PBA} $\mathcal{A}$ over a finite alphabet $\Sigma$ is defined by a
  tuple $\tuple{Q, \Sigma, \delta, q_{0}, F}$ where $Q$ is a finite state set,
  $q_{0}\in Q$ the initial state,
    $\delta:Q\times\Sigma\times Q\rightarrow \interval{0,1}$
  a transition probability function such that for all pairs $q\in Q$ and
  $\sigma\in\Sigma$ we have $\sum_{p\in Q}\delta(q,\sigma,p)\in\set{0,1}$
  and $F\subseteq Q$ is the set of final states.
\end{definition}
Note that this definition may also be altered to allow initial distributions
rather than one initial state \cite{RecOmeLangProbAuto}. Conceptually, a
\ac{PBA} $\mathcal{P}$ processes an input word similarly as other word automata
by sequentially reading the letters of the input word and moving along its
states. Provided the current state is $q$ and the next letter of the input word 
is $a$ then $\mathcal{P}$ consults $\delta$ to determine its next state,
specifically $\mathcal{P}$ chooses a state $p$ by the probability
$\delta(q,a,p)$. Naturally, any run of $\mathcal{P}$ starts in its initial
state $q_{0}$ (respectively $\mathcal{P}$ may choose its initial state by a
given probability distribution $\iota:Q\rightarrow\interval{0,1}$). Considering
one fixed input word $\alpha$ the \ac{PBA} $\mathcal{P}$ provides a process
which produces corresponding state sequences with certain probabilities. Its
acceptance is defined by the notion of how probable the produced state 
sequences are to satisfy the associated Büchi-condition $F$. The formal
definition of this intuition requires a little work beforehand.
Following \cite{Groesser,Bauer}, we fix a set $\Omega$ which is the collection 
of all possible results, in the context of \acp{PBA} all possible sequences of
states $Q^{\omega}$ to the input word $\alpha$. 
For such an $\Omega$ we introduce a \emph{$\sigma$-algebra} which is a 
collection of subsets of $\Omega$, formally:
\begin{definition}[$\sigma$-Algebra]
  For a set $\Omega$ we call $\mathcal{F}\subseteq\Pot(\Omega)$ for $\Omega$ a
  $\sigma$-algebra if
  \begin{enumerate}
    \item $\Omega\in\mathcal{F}$,
    \item for every $A\in\mathcal{F}$ we have $\tuple{\Omega\setminus A}
      \in\mathcal{F}$,
    \item for a countable collection $\tuple{A_{i}}_{i\in\mathbb{N}}$ with
      $A_{i}\in\mathcal{F}$ we also have $\tuple{\cup_{i\in\mathbb{N}}A_{i}}\in
      \mathcal{F}$.
  \end{enumerate}
\end{definition}
A probability space is defined by a set of possible results, a $\sigma$-algebra
which describes those collection of events we can observe and a function which
describes how probable a chosen observation is:
\begin{definition}[Probability Space]
  For a set $\Omega$ and a $\sigma$-algebra $\mathcal{F}$ for $\Omega$ we
  define a probability function as $\mu:\mathcal{F}\rightarrow\interval{0,1}$
  such that
  \begin{enumerate}
    \item $\mu(\Omega) = 1$,
    \item for a countable collection $\tuple{A_{i}}_{i\in\mathbb{N}}$ with
      $A_{i}\in\mathcal{F}$ we have $\mu(\cup_{i\in\mathbb{N}}A_{i}) =
      \sum_{i\in\mathbb{N}}\mu(A_{i})$.
  \end{enumerate}
  And we call a triple $\tuple{\Omega,\mathcal{F},\mu}$ a probability space.
  For any finite or at most countably large set $\Omega$ we can additionally
  define a
  \begin{description}
    \item [Probability Distribution] as a function
      $P:\Omega\rightarrow\interval{0,1}$ such that
      $\sum_{\omega\in\Omega}P(\omega) = 1$. 
  \end{description}
\end{definition}
Thus, we are interested in a probability space that is induced by the \ac{PBA}
$\mathcal{P}$ on the base set $Q^{\omega}$ while reading the input word 
$\alpha$. Fix for every $u\in Q^{*}$ the set 
$\cyl(u) = \set{u\beta : \beta\in Q^{\omega}}\subseteq Q^{\omega}$, a 
\emph{cylinder} as illustrated in Figure \ref{fig:cylinder}.
\begin{drawing}
  \caption{All infinite sequences of states $\set{q_{1},\dots, q_{k}}$ can be
  organised in a tree. In this tree the set $\cyl(u_{1}\dots u_{n})$ are all 
  possible prolongations of the initial sequence $u_{1}\dots u_{n}$ as 
  illustrated by the blue path and attached cylinder.}
  \label{fig:cylinder}
  \begin{center}
  % \resizebox{\textwidth}{!}{
    \includegraphics{tikz/cylinder.pdf}
  % }
  \end{center}
\end{drawing}
By using the transition probabilities of $\mathcal{P}$ we can formulate the 
probability to stay within a certain cylinder while reading a finite prefix of 
the input word. Let $v = u_{0}\dots u_{n}$ be such a finite prefix of $\alpha$.
First we observe, that all cylinders that are not rooted in $q_{0}$ can be 
discarded. For any sequence $q_{0}q_{1}\dots q_{n}$ the run of $\mathcal{P}$
on $\alpha$ stays within $\cyl(q_{0}q_{1}\dots q_{n})$ with the probability to
move along the path $q_{0}q_{1}\dots q_{n}$ while reading $u_{1}\dots u_{n}$.
This is the product of the probabilities to move from $q_{i-1}$ to $q_{i}$
while reading $u_{i}$ for $1\leq i\leq n$; thus 
$\prod_{1\leq i\leq n}\delta(q_{i-1},u_{i},q_{i})$. In order to associate with
the state sequences of $\mathcal{P}$ a probability space we rely on the 
$\sigma$-algebra which is created by including the sets $\cyl(w)$ for all 
$w\in Q^{*}$ and closing the set under negation and countably union. This is
\fxfatal{find reference for borel algebra}
commonly referred to as \emph{Borel}-algebra, cp. \cite{}, and formally 
defined as
\begin{definition}[Borel-algebra on words]
  For a finite set $A$ we call $\mathcal{B}(A)\subseteq\Pot(A^{\omega})$ the 
  smallest $\sigma$-algebra containing $\cyl(w)$ for all $w\in A^{*}$. We call
  a set $C\subset A^{\omega}$ \emph{Borel} if $C\in\mathcal{B}(A)$.
\end{definition}
Given an input word $\alpha$ and having fixed a $\sigma$-algebra, namely the
Borel-algebra over $Q$ which is induced by all cylinders, and the probability 
for staying in a cylinder for $\alpha$.
This allows with 
\cite[Theorem 5.6]{Bauer}\footnote{This theorem is called Carath\'{e}odory's
extension theorem, cp. \cite[Chapter 2.3.]{RandAutoInfTrees}.} to obtain a 
probability space
\begin{equation*}
  \tuple{Q^{\omega}, \mathcal{B}(Q), \mu_{\alpha}}.
\end{equation*}
Note that by \cite[Theorem 5.4]{Bauer} this $\mu_{\alpha}$ is unique.

Since we want to define the acceptance of an \ac{PBA} $\mathcal{P}$ of a word
$\alpha$ in terms of the induced probability of the set 
$\set{\alpha\in Q^{\omega}\mid\alpha\text{ satisfies the Büchi-condition }F}$,
it is necessary to show that this set is measureable, i.e. Borel. This is known
from e.g. \cite[Chapter 4.1.1.]{Groesser} but we present a proof here which
follows \cite[Proposition 6]{RandAutoInfTrees} since this argument can be
easily presented and is re-used later in a more involved context.
\begin{lemma}[Measurability]
  The set $\Acc(F)$ for a Büchi-condition $F\subseteq Q$ is measurable in 
  $\mathcal{B}(Q)$.
  \label{lem:measureabilityAcceptance}
\end{lemma}
\begin{proof}
  In order to show the measureability of $\Acc(F)$ we formulate it at as
  Boolean combination of cylinders inducing its membership in the 
  Borel-algebra by the closure properties of a $\sigma$-algebra.
  Initially, we start with a co-Büchi condition, i.e. those words which 
  eventually do not visit a target set $T$ anymore. More precisely, such a 
  set $T$ defines the accepted language as
  \begin{equation*}
    \Acc_{\cobuechi}(T) = \set{\alpha_{0}\alpha_{1}\dots\in Q^{\omega}\mid
    \text{ there is } i \text{ such that for all } j>i \text{ holds }
    \alpha_{j}\notin T}.
  \end{equation*}
  We claim that
  \begin{equation} 
    \Acc(T) = \bigcup\limits_{u\in Q^{*}}(
    \cyl(u)\setminus\bigcup\limits^{v_{0}\dots v_{n}\in\prolong(u)}_{v_{n}\in T}
    \cyl(v))
    \label{eq:measureability}
  \end{equation}
  which renders $\Acc(T)$ measurable as countable union of measurable sets.
  For $\beta\in\Acc(T)$ we know that there is eventually no occurence of $T$
  anymore. This induces that for every finite prefix $u\sqsubseteq\beta$ after 
  the last occurence of an element in $T$ holds
  $\beta\notin\bigcup\limits^{v_{0}\dots v_{n}\in\prolong(u)}_{v_{n}\in E}
  \cyl(v)$ and thus $\beta$ is element of the right hand side of Equation 
  \ref{eq:measureability}. On the other hand if $\beta\notin\Acc(T)$ we know 
  that for every finite prefix $u\sqsubseteq\beta$ there is a (finite) 
  prolongation $u\sqsubseteq v_{0}\dots v_{n}\sqsubseteq\beta$ such that 
  $v_{n}\in T$, hence $\beta$ is for every such $u$ removed by the cylinder 
  associated with $v_{0}\dots v_{n}$ since $v_{0}\dots v_{n}\in\prolong(u)$.
  This yields that $\beta$ is not element of the right hand side of Equation
  \ref{eq:measureability} which proves the announced equality. 
  For any Büchi-condition $F\subseteq Q$ we use its duality to the 
  co-Büchi-condition $T = Q\setminus F$, i.e.
  \begin{equation*}
    \Acc_{\buechi}(F) = Q^{\omega}\setminus\Acc_{\cobuechi}(T).
  \end{equation*}
  The measureability of $\Acc(F)$ follows by the closure of $\mathcal{B}(Q)$
  under complementation and the explored measureability of $\Acc(T)$.
\end{proof}
Since we explicitly defined \acp{PBA} with Büchi-conditions this suffices to
show their definition to be useful. But it is a natural escalation to consider
more elaborate acceptance conditions, e.g. Muller-, Parity- or 
Rabin-conditions. We show the measureability of those sets here as well, since
it is a easy consequence of Lemma \ref{lem:measureabilityAcceptance}. Also, we
note that we introduce, analogously as for $\omega$-regular word automata, 
corresponding acronyms. \acuse{PRA}\acuse{PMA}\acuse{PPA}
\begin{corollary}
  For a Muller-condition, a Parity-condition and a Rabin-condition
  defined by $\mathcal{F}\subseteq\Pot(Q)$, $\parity$, 
  $R = \set{\tuple{E_{0}, F_{0}},\dots,\tuple{E_{n}, F_{n}}}$ respectively, the 
  sets $\Acc(\mathcal{F})$, $\Acc(\parity)$ and $\Acc(R)$ are measureable in 
  $\mathcal{B}(Q)$.
  \label{cor:borelAcceptance}
\end{corollary}
\begin{proof}
  Since the Rabin- and Parity-condition can be expressed by fitting 
  Muller-conditions it suffices to show the measureability of any 
  $\Acc(\mathcal{F})$ for a Muller-condition $\mathcal{F}\subseteq\Pot(Q)$.
  Fix one Muller-condition $\mathcal{F} = \set{F_{1},\dots,F_{n}}$. We claim
  that for one $F = \set{f_{1},\dots, f_{k}}\in\mathcal{F}$
  \begin{equation*}
    \Acc_{\muller}(\set{F}) = \tuple{
      \bigcap\limits_{1\leq i\leq k}\Acc_{\buechi}(\set{f_{i}})}\setminus
      \Acc_{\cobuechi}(Q\setminus F).
  \end{equation*}
  The individual Büchi-conditions on $f_{i}$ for $1\leq i\leq k$ ensure that 
  every of these states is visited infinitely often while the 
  co-Büchi-condition on all other states ensure that they are only visited
  finitely often rendering the $\Inf$-set to be $F$. By simply setting
  \begin{equation*}
    \Acc_{\muller}(\mathcal{F}) = \bigcup\limits_{F\in\mathcal{F}}(
    \Acc_{\muller}(\set{F}))
  \end{equation*}
  we obtain the measureability of $\Acc_{\muller}(\mathcal{F})$ by the closure
  properties of the $\sigma$-algebra $\mathcal{B}(Q)$.
\end{proof}

This finally allows us to define the semantics of a \ac{PBA} $\mathcal{P}$
formally with
\begin{definition}[Probabilistic Büchi Automaton - Acceptance]
  A \ac{PBA} $\mathcal{P} = \tuple{Q, \Sigma, \delta, q_{0}, F}$ positively 
  (resp. almost-surely) accepts a word $\alpha\in\Sigma^{\omega}$ if
  $\mu_{\alpha}(\Acc(F)) > 0$ (resp. $\mu_{\alpha}(\Acc(F))$).
\end{definition}
This yields two kinds of \acp{PBA} separated by the measure of their accepted 
runs. We want to provide a small discussion of the theory regarding \acp{PBA}.
Therefore we discuss an example of a \ac{PBA} from \cite{RecOmeLangProbAuto} 
which allows us to get a feel for the behavior while also providing a witness 
to separate the expressiveness of $\omega$-regular languages and languages 
positively accepted by \acp{PBA}.
\begin{drawing}
  \caption{A \ac{PBA} accepting under positive acceptance a 
  non-$\omega$-regular language depicted by the same notions as finite word 
  automata before.}
  \label{fig:posacceptingpba}
  \begin{center}
    \resizebox{0.35\textwidth}{!}{%
      \includegraphics{tikz/posacceptingpba.pdf}
    }
  \end{center}
\end{drawing}
\begin{example}
  We examine the \ac{PBA} $\mathcal{P}$ as defined in Figure 
  \ref{fig:posacceptingpba}. The accepted language is by 
  \cite{RecOmeLangProbAuto}
  \begin{equation*}
    \mathcal{L}_{>0} = \set{a^{k_{1}}ba^{k_{2}}b\dots:k_{i}>0\text{ for all }
    i>0\text{ and } \prod_{i>0}(1-\tuple{\frac{1}{2}}^{k_{i}}) > 0}.
  \end{equation*}
  This can be explained the following way: intially $\mathcal{P}$ starts in 
  $q_{0}$, thus the whole \emph{probability mass} lays within $q_{0}$. Every
  read $a$ dissipates half of the mass in $q_{0}$ to $q_{1}$ while the mass in
  $q_{1}$ stays in $q_{1}$. Every read $b$ discards all mass in $q_{0}$ while
  moving all mass in $q_{1}$ to $q_{0}$. Thus, reading $a$ gradually saves away
  probability mass into $q_{1}$ and $b$ restarts this process. Note that any 
  word ending in $a^{\omega}$ is not accepted since an infinite dissipatation 
  leaves no mass in $q_{0}$ behind but only the mass visiting $q_{0}$ 
  infinitely often is considered accepting. Thus, we need to have some 
  probability mass which moves infinitely often through the dissipatation step 
  but gradually, due to the loss of mass, steps need to \enquote{save} more 
  mass in order to not let it diminish too fast.
  \label{ex:posacceptingpba}
\end{example}
As mentioned before the \ac{PBA} $\mathcal{P}$ from Example 
\ref{ex:posacceptingpba} is not $\omega$-regular giving
\begin{lemma}
  Follows from the proof of \cite[Theorem 4]{RecOmeLangProbAuto}.
  The language
  \begin{equation*}
    \set{a^{k_{1}}ba^{k_{2}}b\dots:k_{i}>0\text{ for all }i>0\text{ and }
    \prod_{i>0}(1-\tuple{\frac{1}{2}}^{k_{i}}) > 0}
  \end{equation*}
  can be positively accepted by a \ac{PBA} but is not $\omega$-regular.
  \label{lem:posaccpba>omegareg}
\end{lemma}
\begin{proof}
  It is well known that any $\omega$-regular language contains an ultimately
  periodic word, i.e. a word of the form $u\cdot (v)^{\omega}$ where 
  $\epsilon\neq v$. We argue that no ultimately periodic word is part of the
  language. Since any word ending in $a^{\omega}$ is not part of the language
  (as argued in Example \ref{ex:posacceptingpba}) any ultimatively periodic
  word that might be part of the language contains at least one $b$ in its
  period $v$. But this induces that the sequences of $a$ are bound by a natural
  number $k_{0}$ yielding that for this word
  \begin{equation*}
    \prod_{i>0}(1-\tuple{\frac{1}{2}}^{k_{i}}) \leq
    \underbrace{\prod_{i>0}(1-\tuple{\frac{1}{2}}^{k_{0}})}_{
      \text{infinite product of a constant }c<0} = 0
  \end{equation*}
  rendering it not accepted.
\end{proof}
Furthermore, we state here without proof
\begin{lemma}
  \cite[Lemma 5]{RecOmeLangProbAuto} For any \ac{NBA} $\mathcal{A}$ exists a
  \ac{PBA} $\mathcal{P}$ which accepts the same language.
\end{lemma}
Which combined yields
\begin{theorem}
  \cite[Theorem 4]{RecOmeLangProbAuto}
  The class of languages that can be accepted by a \ac{PBA} strictly contains
  the class of $\omega$-regular languages.
\end{theorem}
Additionally, we know that the class of languages that are positively accepted 
by a \ac{PBA} forms a Boolean algebra. This is captured in
\begin{theorem}
  \cite[Chapter 4.3.]{Groesser}
  The class of languages that can be accepted by a \ac{PBA} form a Boolean 
  algebra. The transformations can be effectively constructed.
  \label{thm:pbaboolalgebra}
\end{theorem}
\begin{proof}[Proof-sketch]
  A union operator is very easily obtained by the following idea: given two
  \acp{PBA} $\mathcal{P}_{1},\mathcal{P}_{2}$ with initial states $q^{1}_{0},
  q^{2}_{0}$ respectively, then we can construct a \ac{PBA} $\mathcal{P}$ which
  effectively uses a initial distribution 
  $\iota:Q^{1}\cup Q^{2}\rightarrow\interval{0,1}$ such that 
  $\iota(q^{1}_{0}) = \iota(q^{2}_{0}) = \frac{1}{2}$ while $\iota(p) = 0$ for
  any other state $p$. By introducing a new initial state $q_{0}$ and 
  \enquote{skipping over} $q^{1}_{0},q^{2}_{0}$ in the first step, i.e.
  setting
  \begin{equation*}
    \delta(q_{0},\sigma, p) = \frac{1}{2}\cdot\delta^{i}(q^{i}_{0},\sigma, p)
    \text{ with }
    i = \begin{cases}
      1&\text{if }p\in Q^{1},\\
      2&\text{if }p\in Q^{2}
    \end{cases}
  \end{equation*}
  we can emulate such an initial distribution $\iota$. This yields that the 
  resulting $\mathcal{P}$ accepts $\alpha$ with a probability
  $\frac{1}{2}\cdot\mu^{1}_{\alpha}(\Acc(F^{1})) + 
  \frac{1}{2}\cdot\mu^{2}_{\alpha}(\Acc(F^{2}))$ which is greater than $0$ if
  and only if one of the probabilities is greater than $0$.

  Complementation requires a rather involved operation which we do not present 
  here; but we mention an intermediate construction, namely a \ac{PRA} 
  $\mathcal{R}$ associated with a \ac{PBA} $\mathcal{P}$. The 
  interesting aspect of $\mathcal{R}$ is that it induces a binary probability 
  measure on its acceptance condition, i.e. for its Rabin-condition $R$ we have
  $\mu_{\alpha}(\Acc(R))\in\set{0,1}$, but it accepts the same language as
  $\mathcal{P}$. The main idea of its construction is to sample partial runs of 
  the original \ac{PBA} and keeping track of those that are accepting. These 
  partial runs can be then \enquote{sticked together}\footnote{The idea of 
  sticking partial runs together is conceptually similar to the proof of 
  Theorem \ref{thm:omegaregularexp}.}. Therefore we state here
  \begin{theorem}
    \cite[Theorem 4.3.2]{Groesser}
    For each \ac{PBA} $\mathcal{P}$ there exists a \ac{PRA} $\mathcal{R}$ that 
    accepts the same language, but for every word the acceptance measure for
    $\mathcal{R}$ is either $0$ or $1$.
    \label{thm:pbatopra}
  \end{theorem}
  Since we soon see that \acp{PBA} with almost-sure acceptance are strictly
  less expressive than \acp{PBA} with positive acceptance we note the 
  correspondance to Theorem \ref{thm:omegaregularexp} in the
  sense that more expressive acceptance conditions allow for a 
  \enquote{stuctural} difference, which in this case is the acceptance measure
  (rather than determinsm before). Also, we note that all \enquote{strong} 
  conditions again coincide but we postpone the proof since it is a corollary 
  from a later result\footnote{
    Nevertheless, it makes use of the same concept, namely \acp{LAR}.
  }.
  \begin{theorem}
    The expressiveness for \ac{PRA}, \ac{PMA} and \ac{PPA} coincides under
    positive as well as almost-sure acceptance.
    \label{thm:probautoequiv}
  \end{theorem}
\end{proof}
The gain in expressiveness for \ac{PBA} with positive acceptance over 
$\omega$-regular languages is unfortunately accompanied by some undesired 
effects. Notably, while emptiness for a \ac{NBA} can be decided by means of 
graph algorithms we have that given a \ac{PBA} $\mathcal{P}$ it is undecideable 
to find if there is any word positively accepted by $\mathcal{P}$ as stated in
\begin{theorem}
  \cite[Theorem 2]{DecProblemsForProbAuto}
  The emptiness problem for \ac{PBA} is undecideable.
  \label{thm:emptinesspospba}
\end{theorem}
And by Theorem \ref{thm:pbatopra} and Theorem \ref{thm:probautoequiv} we obtain
the undecideability of emptiness for probabilistic word automata with 
almost-sure accpetance of more expressiv acceptance conditions and since 
Büchi-conditions can be emulated also positive acceptance of more expressice
acceptance conditions:
\begin{corollary}
  The emptiness problem for probabilistic word automata with Rabin-, Parity- or
  Muller-conditions is undecideable for positive and almost-sure acceptance.
  \label{cor:emptinessstrongalmostsureprob}
\end{corollary}

In order to re-gain some desired algorithmic properties of e.g. the emptiness
problem \cite{DecProblemsForProbAuto} considers almost-sure acceptance. By
Theorem \ref{thm:pbatopra} we conclude easily that the emptiness problem for
almost-sure accepting \ac{PRA} (and by Theorem \ref{thm:probautoequiv} also for
\ac{PPA} and \ac{PMA}) is undecideable. But for Büchi-conditions this is not
the case as stated in
\begin{theorem}
  \cite[Theorem 6]{DecProblemsForProbAuto}
  The emptiness problem for \ac{PBA} with almost-sure acceptance is decideable.
  \label{thm:emptinessalmostsurepba}
\end{theorem}
Again, we consider an example for a \ac{PBA} with almost-sure acceptance (taken
from \cite{DecProblemsForProbAuto}.
\begin{example}
  Let $\mathcal{P}$ be a \ac{PBA} as defined in Figure 
  \ref{fig:almostsureacceptingpba}. The language accepted by $\mathcal{P}$ is
  \begin{equation*}
    L_{\lambda} = \set{a^{k_{1}}ba^{k_{2}}b\dots\mid k_{i}>0\text{ for }i>0
    \text{ and }\prod\limits_{i>0}\tuple{1-\lambda^{k_{i}}} = 0}.
  \end{equation*}
  Every $a$-sequence $w\in\set{a}^{*}$ models an experiment which determines 
  how probable it is to stay in $q_{2}$, namely $\lambda^{\size{w}}$. Every 
  experiment is concluded with the occurence of a single $b$.  Naturally, the 
  probability to \enquote{fail} such an experiment is $1-\lambda^{\size{w}}$. 
  Due to the equivalent behaviour of $q_{3}$ and $q_{0}$ all these experiments 
  are independent and therefore, consistenly failing these experiments with 
  $a$-sequences of length $k_{1}, k_{2},\dots$ starting from the $m$-th 
  experiment happens with probability
  \begin{equation*}
    p_{m} = \prod\limits_{i>m}(1-\lambda^{k_{i}})
    \text{ and thus }
    \prod\limits_{i>0}(1-\lambda^{k_{i}}) = 
    \underbrace{\prod\limits_{m\geq i>0}(1-\lambda^{k_{i}})}_{=c_{m}}
    \cdot p_{m}.
  \end{equation*}
  $c_{m} > 0$ implies $c_{m}\cdot p_{m} = 0$ if and only if 
  $p_{m} = 0$.
  Every non-accepting run $\pi$ consistently fails from one experiment onwards.
  This allows to conclude with asymptotic equvialence of $p_{m}$ and 
  $c_{m}\cdot p_{m}$ and $\sigma$-additivity of a probability measure that the 
  probability of non-accepting runs is
  \begin{equation*}
    \sum_{i > 0} p_{m}\text{, and therefore }
    \sum_{i > 0} p_{m} = 0\text{ iff }\prod\limits_{i>0}(1-\lambda^{k_{i}}) =0.
  \end{equation*}
  In order to accept almost-surely the set of non-accepting runs needs to be
  \emph{negligible}, i.e. carrying a probability of $0$.  Thus, indeed 
  $\mathcal{P}$ accepts $L_{\lambda}$.

  If $\lambda$ is set to $\frac{1}{2}$, the resulting language 
  $\mathcal{L}_{=1}$ is - to a certain degree - dual to the language in Example 
  \ref{ex:posacceptingpba} (without considering all words ending in 
  $a^{\omega}$ or those words starting in $b$ or ommitting an 
  \enquote{$a$-phase}). This duality is rooted in structural similarities. 
  Examining the \emph{Separation} part, it mirrors the $a$-transitions of the 
  \ac{PBA} in Figure \ref{fig:posacceptingpba}. But an occurences of $b$ does 
  not discard the probability mass in $q_{2}$ ($q_0$, respectively) but saves
  it in an accepting state and re-introduces into the circulation (as seen in 
  the \emph{Disposal} part).
  \label{ex:almostsureacceptingpba}
\end{example}
\begin{drawing}
  \caption{A \ac{PBA} which accepts a non-$\omega$-regular language under
  almost-sure acceptance. We conceptually separate it into two regions, namely
  the \enquote{Separation-} and \enquote{Disposal-}region (marked by the green
  or red box respectively).}
  \label{fig:almostsureacceptingpba}
  \begin{center}
    \resizebox{0.4\textwidth}{!}{%
      \includegraphics{tikz/almostsureacceptingpba.pdf}
    }
  \end{center}
\end{drawing}
Considering Example \ref{ex:almostsureacceptingpba} it is a natural question if
the choice of $\lambda$ matters. Quoting the following
\begin{lemma}
  \cite[Lemma 1]{DecProblemsForProbAuto}
  For each $n\in\mathbb{N}_{>1}$ there exists a sequence $\tuple{k_{i}}_{i>0}$
  such that
  \begin{equation*}
    \prod_{i>0}(1-\lambda^{k_{i}}) > 0 \text{ if and only if }
    \lambda<\frac{1}{n},
  \end{equation*}
  \label{lem:measureinpba}
\end{lemma}
allows us to answer this question positively. Choosing $\lambda_{1}$ and
$\lambda_{2}$ differently, we can find $n$ such that 
$\lambda_{1} < \frac{1}{n} < \lambda_{2}$ and then the languages of 
$\mathcal{P}$ with $\lambda_{1}$ and $\lambda_{2}$ do not agree on the sequence
given by Lemma \ref{lem:measureinpba} for $n$ since $\lambda_{1}$ renders
this sequence non-accepting in $\mathcal{P}$ while $\lambda_{2}$ renders it
accepting (as described in \cite{DecProblemsForProbAuto}).

Regarding the relationship of $\omega$-regularity to languages which are 
accepted by a \ac{PBA} with almost-sure acceptance we find that
\begin{theorem}
  \cite[Theorem 4, (b), (c)]{DecProblemsForProbAuto}
  The classes of $\omega$-regular languages and languages accepted by a 
  \ac{PBA} with almost-sure acceptance are incomparable.
\end{theorem}
\begin{proof}
  The proof separates in both directions of possible inclusions. As noted in
  Figure \ref{fig:almostsureacceptingpba} the defined \ac{PBA} $\mathcal{P}$
  with almost-surely accepts a non-$\omega$-regular language, hence
  \begin{lemma}
    There is a non-$\omega$-regular language that is accepted by any \ac{PBA} 
    with almost-sure acceptance.
  \end{lemma}
  \begin{proof}
    As expected, we use the language $\mathcal{L}_{=1}$ from Example 
    \ref{ex:almostsureacceptingpba} to witness this claim. In Example 
    \ref{ex:almostsureacceptingpba} we already mentioned the 
    \enquote{almost duality} of this language to $\mathcal{L}_{>0}$ from 
    Example \ref{ex:posacceptingpba}. By defining
    \begin{equation*}
      \mathcal{L}_{\omega} = \set{
        \alpha_{1}\alpha_{2}\dots\in\set{a,b}^{\omega}\middle|
        \text{s.t. }
        \begin{aligned}
          &\alpha_{1} = b&\text{or }\\
          &\alpha_{i} = \alpha_{i+1} = b&\text{for one }i>0\text{ or }\\
          &\alpha_{i} = \alpha_{i+1} = \dots = b&\text{for one }i>0\text{ holds}
        \end{aligned}
      },
    \end{equation*}
    we can state that 
    $\set{a,b}^{\omega}\setminus\mathcal{L}_{>0} = 
      \mathcal{L}_{=1}\cup\mathcal{L}_{\omega}$.
    $\mathcal{L}_{\omega}$ can be separated into its individual parts where 
    each part is $\omega$-regular and so is $\mathcal{L}_{\omega}$ by closure 
    under union for $\omega$-regular languages. Thus, assuming 
    $\mathcal{L}_{=1}$ $\omega$-regularity renders 
    $\mathcal{L}_{=1}\cup\mathcal{L}_{\omega}$ $\omega$-regular and by closure 
    under complement of $\omega$-regular languages also $\mathcal{L}_{>0}$ 
    which is a contradiction to Lemma \ref{lem:posaccpba>omegareg}. Therefore, 
    we state that $\mathcal{L}_{=1}$ is not $\omega$-regular.
  \end{proof}
  And on the other hand we state that
  \begin{lemma}
    There is a $\omega$-regular language that cannot be accepted by any 
    \ac{PBA} with almost-sure acceptance.
  \end{lemma}
  \begin{proof}
    We examine the $\omega$-regular language $\mathcal{L}$ over the alphabet
    $\set{a,b}$ of those words that contain only finitely many $a$ (cp. Example
    \ref{ex:fina}).
    \fxfatal{to be finished}
  \end{proof}
\end{proof}

\subsection{Markov Decision Processes}
In the following we introduce \acp{MDP}, a model of interaction between a 
probabilistic domain and an actor within this domain. The actions of the actor
impact the behavior of the domain and depending on the behavior of the domain
the actor may adapt her behavior. It shares significant structural similarities
to \acp{PBA} but allows for a rather different interpretation. Nevertheless, we
reuse various concepts from \acp{PBA} to define \acp{MDP}. The following 
definitions follow \cite{RandAutoInfTrees}:
\begin{definition}[Markov Decision Process]
  A \acl*{MDP} is modelled as tuple
  $\tuple{S, A, \tuple{\tau_{a}}_{a\in A}, s_{0}}$ where $S$ is a set of states
  and $A$ a set of actions. Given an action $a\in A$ the corresponding
  transition function $\tau_{a}:S\times S\rightarrow \interval{0,1}$ satisfies
  for every $q\in S$ that $\sum_{p\in S}\tau_{a}\tuple{q,p} = 1$.
  $s_{0}$ is the initial state. Additionally we define a few helpful
  auxilliaries:
  \begin{description}
    \item [Plays] We consider all infinite sequences of states
      (i.e. $S^{\omega}$) that start in $s_{0}$ a valid play. All plays
      are gathered in $\plays = \set{
        \alpha\in S^{\omega}\mid \alpha_{0} = s_{0}} = \cyl(s_{0})$.
    \item [Strategy] We define $\varphi:S^{+}\rightarrow A$ as strategy for
      a \ac{MDP}. Such a strategy models interaction with a \ac{MDP} by
      giving a specific $a\in A$ to which the \ac{MDP} reacts by $\tau_{a}$.
    \item [Initial Distribution] Occasionally, we substitute the initial state
      $s_{0}$ with an initial probability distribution
      $\iota_{0}:S\rightarrow[0,1]$ which allows the plays to start in any
      $s\in S$ such that $\iota_{0}(s) > 0$.
  \end{description}
  \label{def:mdp}
\end{definition}
A strategy $\varphi$ for an \ac{MDP}
$\mathcal{M} = \tuple{S, A, \tuple{\tau_{a}}_{a\in A}, s_{0}}$ induces a 
probability measure $\mu_{\varphi}$ on the Borel-algebra $\mathcal{B}(S)$. 
\acp{PBA} can be interpreted as \acp{MDP} where the decisions in every state
are induced by a word $\alpha = \alpha_{1}\alpha_{2}\dots\in\Sigma^{\omega}$
rather than a player. Any \ac{PBA} $\mathcal{P}$ can be interpreted as \ac{MDP}
and for any $\alpha$ a corresponding strategy 
$\varphi_{\alpha}:Q^{+}\rightarrow\Sigma$ with 
$\varphi_{\alpha}(w) = \alpha_{\size{w}}$ can be defined. On the other hand, do
strategies for \acp{MDP} allow for an interaction between the player and the
environment, thus the player might react differently considering which state
she stays in, e.g.
\begin{example}
  We interpret the \ac{PBA} pictured in Figure \ref{fig:posacceptingpba} as
  \ac{MDP} $\mathcal{M}$. Furthermore, we challenge the player with the task to
  visit states in $F$ infinitely often. The player trivially has a strategy 
  that allows him to satisfy this task by
  \begin{equation*}
    \varphi(p_{1}\dots p_{n}) = \begin{cases}
      b&\text{if }p_{n} = q_{1},\\
      a&\text{if }p_{n} = q_{0}.
    \end{cases}
  \end{equation*}
  The player is not restricted as words are to be \enquote{consistent} 
  regarding the length of the play. A word has exactly one letter at each 
  position while a player may provide different actions (or letters) after
  the same length of the play depending on the state she currently resides in
  (or even the current history of the play).
  \label{ex:pbaasmdp}
\end{example}
This freedom of the player can be restricted again. Therefore, we define in the
following \acp{POMDP}. A \ac{POMDP} is a \ac{MDP} $\mathcal{M}$ with an 
associated equivalence relation $\sim$. The equivalence relation models a 
restriction of the player to observe the state of $\mathcal{M}$. For every
$s\in S_{\mathcal{M}}$ we define
$[s]_{\sim} = \set{r\in S\mid \sim(r,s)}$ and
$S_{\mathcal{M}}/_{\sim}$ as set of all these equivalence classes. A
strategy for a \ac{POMDP} $\tuple{\mathcal{M}, \sim}$ is defined as
$\varphi:(S_{\mathcal{M}}/_{\sim})^{+}\rightarrow A_{\mathcal{M}}$.
\begin{example}
  Again, consider a \ac{PBA} $\mathcal{P}$ and interpret it as an \ac{MDP}
  $\mathcal{M}$ we observed that every word $\alpha$ induces a corresponding
  strategy $\varphi_{\alpha}$. But not every strategy induces a word because
  strategies on $\mathcal{M}$ may choose depending on the state of the play
  different letters for plays of the same length which creates an inconsistency
  with respect to an - hypothetical - associated word (cp. Example 
  \ref{ex:pbaasmdp}). We add to $\mathcal{M}$ an equivalence relation
  $\sim = S_{\mathcal{M}} \times S_{\mathcal{M}}$ (as suggested in
  \cite{DecProblemsForProbAuto}), hence there is exactly one equivalence class.
  Firstly, we observe that $\varphi_{\alpha}$ is still a valid strategy for the
  resulting \ac{POMDP} $\mathcal{N} = \tuple{\mathcal{M}, \sim}$. Secondly,
  since there is exactly one equivalence class every strategy for $\mathcal{N}$
  observes at any time only the length of the history. Hence we can provide an
  equivalent strategy of the form $\varphi:\mathbb{N}\rightarrow \Sigma$. This
  in turn means that every strategy in $\mathcal{N}$ induces a word in
  $\Sigma^{\omega}$ of the form $\varphi(0)\varphi(1)\dots$. Furthermore, the
  measure of one strategy and an associated word coincide, i.e. the language of
  $\mathcal{P}$ can be understood as strategy space for $\mathcal{N}$ and vice
  versa. This illustrates an idea which is used to define emptiness games for
  certain automata (e.g. \cite[Proposition 45]{RandAutoInfTrees}).
  \label{ex:pbaaspomdp}
\end{example}
The definitions of \acp{MDP} and \acp{POMDP} are for now probabilistic 
transition systems and strategies induce a behavior of such a transition 
system. To model goals or task for the player to satisfy we again use 
Büchi-, Rabin-, Parity- and Muller-conditions. We can then ask if a player has
a strategy to positively or almost-surely satisfy the associated condition in
an \ac{MDP} or \ac{POMDP}. The equality described in Example 
\ref{ex:pbaaspomdp} directly induces some undecideability results, namely
\begin{theorem}
  \cite[Corollary 3 (a)]{DecProblemsForProbAuto}
  \cite[Theorem 5]{QualAnaPOMDP}
  It is undecideable to compute a strategy for a \ac{POMDP} that either 
  positively or almost-surely satisfies a Rabin-, Parity- or Muller-condition.
  Additionally, it is undecideable to compute a strategy for a \ac{POMDP} that
  positively satisfies a Büchi-condition.
\end{theorem}
\begin{proof}
  Any algorithm to compute a strategy for a \ac{POMDP} that positively 
  satisfies a Büchi-condition immediately solves the emptiness problem for
  \acp{PBA} with positive acceptance which is undecideable by Theorem 
  \ref{thm:emptinesspospba}. Furthermore, it is easy to express 
  Büchi-conditions as Parity-, Rabin- or Muller-conditions, rendering the 
  positive acceptance of these conditions in \ac{POMDP} at least as difficult
  as for Büchi-conditions. 
  
  By Theorem \ref{thm:pbatopra} and Theorem \ref{thm:probautoequiv} there is 
  for any \ac{PBA} an equivalent \ac{PRA}, \ac{PPA} and \ac{PMA} which 
  almost-surely accepts the same language. Hence again computing a strategy for 
  the almost-sure satisfaction of Rabin-, Parity- or Muller-conditions for 
  \acp{POMDP} solves the undecideable emptiness problem for \acp{PBA} with 
  positive acceptance.
\end{proof}
On the other hand, there are some positive results regarding computation of 
strategies for \acp{MDP} and for almost-sure satisfaction of Büchi-conditions
in \acp{POMDP}:
\begin{theorem}
  \cite[Theorem 5]{QualAnaPOMDP}
  It is possible to compute a strategy for a \ac{POMDP} that almost-surely 
  satisfies an associated Büchi-condition.
  \label{thm:pomdpstratsynthesis}
\end{theorem}
and
\begin{theorem}
  \cite[Theorem 4.1.7.]{QuanStochParityGames}
  It is possible to compute a strategy for a \ac{MDP} that almost-surely
  satisfies an associated Büchi- or Parity-condition.
  \label{thm:mdpstratsynthesis}
\end{theorem} 
Additionally to the close connection of \acp{POMDP} to probabilistic word 
automata we make use of these models in the later course to define decision
procedures for even more elaborate probabilistic automata.

