% inputenc:
\usepackage[utf8]{inputenc}

\usepackage{epigraph}

% header/footer
\usepackage{fancyhdr}

% including graphics
\usepackage[final]{graphicx}

% annotations
\usepackage[inline=true, margin=true]{fixme}
\fxsetup{}


% tikz
\usepackage{tikz}
\usetikzlibrary{}

% math-symbols
\usepackage{amssymb}
\usepackage{amsmath}
\usepackage{amsthm}
\usepackage{bbm}
\newcommand{\tuple}[1]{\left(#1\right)}
\newcommand{\set}[1]{\left\{#1\right\}}
\newcommand{\size}[1]{\left\vert{#1}\right\vert}
\newcommand{\interval}[1]{\left[#1\right]}

% titlepage
\usepackage{titling}

% acronyms
\usepackage{acro}
\DeclareAcronym{MDP}{
  short = MDP ,
  long  = Markov Decision Process ,
  class = environment ,
  long-plural  = es
}
\DeclareAcronym{POMDP}{
  short = POMDP,
  long = Partially Observable Markov Decision Process ,
  class = environment ,
  long-plural = es
}
\DeclareAcronym{MC}{
  short = MC ,
  long  = Markov Chain ,
  class = environment ,
  long-plural  = es
}
\DeclareAcronym{AMDP}{
  short = AMDP ,
  long  = Annotated Markov Decision Process ,
  class = environment ,
  long-plural  = es
}

\DeclareAcronym{WDTA}{
  short = WDTA ,
  long  = Weighted Descent Tree Automaton ,
  class = automaton ,
  long-plural-form = Weighted Descent Tree Automata
}

\DeclareAcronym{PWA}{
  short = PWA ,
  long  = Probabilistic Weighted Automaton ,
  class = automaton ,
  long-plural-form = Probabilistic Weighted Automata
}

\DeclareAcronym{PBA}{
  short = PBA ,
  long  = Probabilistic Büchi Automaton ,
  class = automaton ,
  long-plural-form = Probabilistic Büchi Automata
}
\DeclareAcronym{DBA}{
  short = DBA ,
  long  = Deterministic Büchi Automaton ,
  class = automaton ,
  long-plural-form = Deterministic Büchi Automata
}
\DeclareAcronym{NBA}{
  short = NBA ,
  long  = Non-Deterministic Büchi Automaton ,
  class = automaton ,
  long-plural-form = Non-Deterministic Büchi Automata
}

\DeclareAcronym{DMA}{
  short = DMA ,
  long  = Deterministic Muller Automaton ,
  class = automaton ,
  long-plural-form = Deterministic Muller Automata
}
\DeclareAcronym{NMA}{
  short = NMA ,
  long  = Non-Deterministic Muller Automaton ,
  class = automaton ,
  long-plural-form = Non-Deterministic Muller Automata
}

\DeclareAcronym{DRA}{
  short = DRA ,
  long  = Deterministic Rabin Automaton ,
  class = automaton ,
  long-plural-form = Deterministic Rabin Automata
}
\DeclareAcronym{NRA}{
  short = NRA ,
  long  = Non-Deterministic Rabin Automaton ,
  class = automaton ,
  long-plural-form = Non-Deterministic Rabin Automata
}

\DeclareAcronym{DSA}{
  short = DSA ,
  long  = Deterministic Streett Automaton ,
  class = automaton ,
  long-plural-form = Deterministic Streett Automata
}
\DeclareAcronym{NSA}{
  short = NSA ,
  long  = Non-Deterministic Streett Automaton ,
  class = automaton ,
  long-plural-form = Non-Deterministic Streett Automata
}

\DeclareAcronym{DPA}{
  short = DPA ,
  long  = Deterministic Parity Automaton ,
  class = automaton ,
  long-plural-form = Deterministic Parity Automata
}
\DeclareAcronym{NPA}{
  short = NPA ,
  long  = Non-Deterministic Parity Automaton ,
  class = automaton ,
  long-plural-form = Non-Deterministic Parity Automata
}

\DeclareAcronym{LAR}{
  short = LAR ,
  long  = Latest Appearance Record ,
  class = auxilliary
}

% math operators
\DeclareMathOperator{\cyl}{cyl}
\DeclareMathOperator{\plays}{Plays}
\DeclareMathOperator{\follow}{follow}
\DeclareMathOperator{\parity}{par}
\DeclareMathOperator{\supp}{support}
\DeclareMathOperator{\Inf}{Inf}
\DeclareMathOperator{\prolong}{prolong}
\DeclareMathOperator{\cob}{coB}
\DeclareMathOperator{\bue}{Bue}
\DeclareMathOperator{\lar}{LAR}
\DeclareMathOperator{\perm}{Perm}
\DeclareMathOperator{\up}{update}
\DeclareMathOperator{\Syn}{Syn}
\DeclareMathOperator{\compRuns}{compRuns}
\DeclareMathOperator{\Runs}{Runs}
\DeclareMathOperator{\Paths}{Paths}
\DeclareMathOperator{\Acc}{Acc}
\DeclareMathOperator{\inner}{inner}

% environments
\usepackage{mdframed}
\newtheorem{theorem}{Theorem}[section]
\newtheorem{proposition}{Proposition}[section]
\newtheorem{lemma}{Lemma}[section]
\newtheorem{corollary}{Corollary}[section]

\newtheorem{localexample}{Example}[section]
\newenvironment{example}
{\begin{mdframed}[nobreak=true, backgroundcolor=green!40]\begin{localexample}}
{\end{localexample}\end{mdframed}}

\newtheorem{localdef}{Definition}[section]
\newenvironment{definition}[1][]
{\begin{mdframed}[nobreak=true, backgroundcolor=gray!20]\begin{localdef}\ifthenelse{\equal{#1}{}}{}{\emph{\textbf{#1}}:}\leavevmode\\}
{\end{localdef}\end{mdframed}}

% quotes
\usepackage{csquotes}

% bibliography
\usepackage[style=alphabetic,backend=biber]{biblatex}
\addbibresource{bib/synthesis.bib}
\addbibresource{bib/automata.bib}
\addbibresource{bib/probability.bib}
