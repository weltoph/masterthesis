% inputenc:
\usepackage[utf8]{inputenc}

% header/footer
\usepackage{fancyhdr}

% including graphics
\usepackage[final]{graphicx}

% annotations
\usepackage[margin=true]{fixme}
\fxsetup{}

% colors
\usepackage{xcolor}
\colorlet{bgdef}{gray!20}
\colorlet{bgex}{green!40}
\colorlet{bgfig}{yellow!40}

% math-symbols
\usepackage{amssymb}
\usepackage{amsmath}
\usepackage{amsthm}
\usepackage{bbm}
\newcommand{\tuple}[1]{\left(#1\right)}
\newcommand{\set}[1]{\left\{#1\right\}}
\newcommand{\size}[1]{\left\vert{#1}\right\vert}
\newcommand{\interval}[1]{\left[#1\right]}

% titlepage
\usepackage{titling}

% acronyms
\usepackage{acro}

\DeclareAcronym{MDP}{
  short = MDP ,
  long  = Markov Decision Process ,
  class = environment ,
  long-plural  = es
}
\DeclareAcronym{POMDP}{
  short = POMDP,
  long = Partially Observable Markov Decision Process ,
  class = environment ,
  long-plural = es
}
\DeclareAcronym{MC}{
  short = MC ,
  long  = Markov Chain ,
  class = environment ,
  long-plural  = es
}

\DeclareAcronym{BTA}{
  short = BTA ,
  long  = Büchi Tree Automaton ,
  class = automaton ,
  long-plural-form = Büchi Tree Automaton 
}
\DeclareAcronym{RTA}{
  short = RTA ,
  long  = Rabin Tree Automaton ,
  class = automaton ,
  long-plural-form = Rabin Tree Automaton 
}
\DeclareAcronym{MTA}{
  short = MTA ,
  long  = Muller Tree Automaton ,
  class = automaton ,
  long-plural-form = Muller Tree Automaton 
}
\DeclareAcronym{PTA}{
  short = PTA ,
  long  = Parity Tree Automaton ,
  class = automaton ,
  long-plural-form = Parity Tree Automaton 
}

\DeclareAcronym{ABTA}{
  short = ABTA ,
  long  = Alternating Büchi Tree Automaton ,
  class = automaton ,
  long-plural-form = Alternating Büchi Tree Automata
}
\DeclareAcronym{ARTA}{
  short = ARTA ,
  long  = Alternating Rabin Tree Automaton ,
  class = automaton ,
  long-plural-form = Alternating Rabin Tree Automata
}
\DeclareAcronym{AMTA}{
  short = AMTA ,
  long  = Alternating Muller Tree Automaton ,
  class = automaton ,
  long-plural-form = Alternating Muller Tree Automata
}
\DeclareAcronym{APTA}{
  short = APTA ,
  long  = Alternating Parity Tree Automaton ,
  class = automaton ,
  long-plural-form = Alternating Parity Tree Automata
}

\DeclareAcronym{WDTA}{
  short = WDTA ,
  long  = Weighted Descent Tree Automaton ,
  class = automaton ,
  long-plural-form = Weighted Descent Tree Automata
}
\DeclareAcronym{PWA}{
  short = PWA ,
  long  = Probabilistic Weighted Automaton ,
  class = automaton ,
  long-plural-form = Probabilistic Weighted Automata
}

\DeclareAcronym{PBA}{
  short = PBA ,
  long  = Probabilistic Büchi Automaton ,
  class = automaton ,
  long-plural-form = Probabilistic Büchi Automata
}
\DeclareAcronym{PRA}{
  short = PRA ,
  long  = Probabilistic Rabin Automaton ,
  class = automaton ,
  long-plural-form = Probabilistic Rabin Automata
}
\DeclareAcronym{PMA}{
  short = PMA ,
  long  = Probabilistic Muller Automaton ,
  class = automaton ,
  long-plural-form = Probabilistic Muller Automata
}
\DeclareAcronym{PPA}{
  short = PPA ,
  long  = Probabilistic Parity Automaton ,
  class = automaton ,
  long-plural-form = Probabilistic Parity Automata
}

\DeclareAcronym{DBA}{
  short = DBA ,
  long  = Deterministic Büchi Automaton ,
  class = automaton ,
  long-plural-form = Deterministic Büchi Automata
}
\DeclareAcronym{NBA}{
  short = NBA ,
  long  = Non-Deterministic Büchi Automaton ,
  class = automaton ,
  long-plural-form = Non-Deterministic Büchi Automata
}
\DeclareAcronym{DMA}{
  short = DMA ,
  long  = Deterministic Muller Automaton ,
  class = automaton ,
  long-plural-form = Deterministic Muller Automata
}
\DeclareAcronym{NMA}{
  short = NMA ,
  long  = Non-Deterministic Muller Automaton ,
  class = automaton ,
  long-plural-form = Non-Deterministic Muller Automata
}
\DeclareAcronym{DRA}{
  short = DRA ,
  long  = Deterministic Rabin Automaton ,
  class = automaton ,
  long-plural-form = Deterministic Rabin Automata
}
\DeclareAcronym{NRA}{
  short = NRA ,
  long  = Non-Deterministic Rabin Automaton ,
  class = automaton ,
  long-plural-form = Non-Deterministic Rabin Automata
}
\DeclareAcronym{DPA}{
  short = DPA ,
  long  = Deterministic Parity Automaton ,
  class = automaton ,
  long-plural-form = Deterministic Parity Automata
}
\DeclareAcronym{NPA}{
  short = NPA ,
  long  = Non-Deterministic Parity Automaton ,
  class = automaton ,
  long-plural-form = Non-Deterministic Parity Automata
}

\DeclareAcronym{LAR}{
  short = LAR ,
  long  = Latest Appearance Record ,
  class = auxilliary
}

% math operators
\DeclareMathOperator{\cyl}{cyl}
\DeclareMathOperator{\plays}{Plays}
\DeclareMathOperator{\follow}{follow}
\DeclareMathOperator{\parity}{par}
\DeclareMathOperator{\supp}{support}
\DeclareMathOperator{\Inf}{Inf}
\DeclareMathOperator{\prolong}{prolong}
\DeclareMathOperator{\cobuechi}{co-B\ddot{u}chi}
\DeclareMathOperator{\buechi}{B\ddot{u}chi}
\DeclareMathOperator{\muller}{Muller}
\DeclareMathOperator{\lar}{LAR}
\DeclareMathOperator{\perm}{Perm}
\DeclareMathOperator{\up}{update}
\DeclareMathOperator{\Syn}{Syn}
\DeclareMathOperator{\compRuns}{compRuns}
\DeclareMathOperator{\Runs}{Runs}
\DeclareMathOperator{\Paths}{Paths}
\DeclareMathOperator{\Acc}{Acc}
\DeclareMathOperator{\inner}{inner}
\DeclareMathOperator{\Pot}{Pot}

% environments
\usepackage{mdframed}
\newtheorem{theorem}{Theorem}[section]
\newtheorem{proposition}{Proposition}[section]
\newtheorem{lemma}{Lemma}[section]
\newtheorem{corollary}{Corollary}[section]

\AfterEndEnvironment{theorem}{\noindent\ignorespaces}
\AfterEndEnvironment{proposition}{\noindent\ignorespaces}
\AfterEndEnvironment{lemma}{\noindent\ignorespaces}
\AfterEndEnvironment{corollary}{\noindent\ignorespaces}

\newtheorem{localexample}{Example}[section]
\newenvironment{example}
{\begin{mdframed}[backgroundcolor=bgex]\begin{localexample}}
{\end{localexample}\end{mdframed}}

\AfterEndEnvironment{example}{\noindent\ignorespaces}

\newtheorem{localdef}{Definition}[section]
\newenvironment{definition}[1][]
{\begin{mdframed}[backgroundcolor=bgdef]\begin{localdef}\ifthenelse{\equal{#1}{}}{}{\emph{\textbf{#1}}:}\leavevmode\\}
{\end{localdef}\end{mdframed}}

\AfterEndEnvironment{definition}{\noindent\ignorespaces}

\newenvironment{drawing}
{\begin{figure}\begin{mdframed}[backgroundcolor=bgfig]}
{\end{mdframed}\end{figure}}

\AfterEndEnvironment{drawing}{\noindent\ignorespaces}

% quotes
\usepackage{csquotes}

% captions:
\usepackage{caption}
\captionsetup{labelfont=bf}
\captionsetup{labelsep=period}

% bibliography
\usepackage[style=alphabetic,backend=biber]{biblatex}
\addbibresource{bib/synthesis.bib}
\addbibresource{bib/automata.bib}
\addbibresource{bib/probability.bib}
