\acp{WDTA} present with a complex theory but with strong capabilities of
modelling probabilistic systems. Namely, even the restricted class of
choiceless \acp{WDTA} subsumes various stochastic models such as \acp{POMDP} or
\acp{PBA}. But the complex nature of \acp{WDTA} arguably lacks an easy
intuition. Addressing this complexity by restriction to choiceless \acp{WDTA}
seems a valuable approach by its connection to the well-established concept of
\acp{POMDP}. Neverthelss, we argue that the theory of choiceless \acp{WDTA} is
not subsumed by the theory of \acp{POMDP} since there is a subtle difference in
semantics of the considered objects. \acp{WDTA} provide a tool to argue about
languages of trees and therefore induce questions about collections of elements
where strategies for \acp{POMDP} are more considered in an existential setting,
i.e. exists a strategy that satisfy certain conditions. The view on collections
of trees or strategies and the associated questions of e.g. closure properties
raises questions about strategies that are well behaved for more than one
structure (or even for the same structure with different objectives).

Additionally, \acp{WDTA} have proven a useful tool to translate the rich theory
of word and tree automata and their connections to the probabilistic setting.
For example we see that the weighted unrollment of \acp{PBA} run-trees along
the paths in a \ac{WDTA} mirrors the parallel executions of word automata along
tree automata. Interestingly, we unvail in this way the chimeric nature of
\acp{PBA} regarding non-determinism. The usage of probabilities to resolve
non-determinism allows to use \enquote{deterministic} run-trees (which is
important for the parallel execution on paths) but still allow to model
decisions of the automaton. Notably, arguments around \acp{PBA} in
\cite{Groesser} resolve by reasoning about probabilistic decisions of the
automaton while our reasoning for \acp{WDTA} is closer connected to arguments
over run-trees.
