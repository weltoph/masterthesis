\acp{WDTA} present with a complex theory and strong capabilities to model other
probabilistic systems. Namely, even the restricted class of choiceless
\acp{WDTA} subsumes various stochastic models such as \acp{POMDP} or \acp{PBA}.
Hence reducing the complexity of \acp{WDTA} by restriction to choicelessness
seems a valuable approach. Neverthelss, we argue that the theory of choiceless
\acp{WDTA} is not itself subsumed by the theory of \acp{POMDP} although we
established strong connections between these models. Notably, there is a subtle
difference in semantics of the considered objects. \acp{WDTA} provide a tool to
argue about languages of trees and therefore induce questions about collections
of elements where strategies for \acp{POMDP} are more considered in an
existential setting, i.e. exists a strategy that satisfy certain conditions.
The view on collections of trees or strategies and the associated notions of
e.g. closure properties raises questions about strategies that are well-behaved
for more than one structure (or even for the same structure with different
objectives).

Additionally, \acp{WDTA} have proven a useful tool to translate the rich theory
of word and tree automata and their connections to the probabilistic setting.
For example we see that the weighted unrollment of \acp{PBA} run-trees along
the paths in a \ac{WDTA} mirrors the parallel executions of word automata along
tree automata. Interestingly, we unvail in this way the chimeric nature of
\acp{PBA} regarding non-determinism. The usage of probabilities to resolve
non-determinism allows to use \enquote{deterministic} run-trees (which is
important for the parallel execution on paths) but still allow to model
decisions of the automaton. Notably, this allows to skip the determinisation
step that is usually required for the specifying automaton (cp.
$\omega$-regular languages). This determinisation is inherently costly.
However, it is highly non-trivial to formulate specifications in almost-surely
accepting \acp{PBA}. Therefore, we note that the obtained synthesis results
are ought to be considered intermediate. In contrast to
$\omega$-regular languages which capture linear time logics
\cite{PrinciplesOfMC} and the monadic second-order theory of linearly ordered
sets \cite{DecMethRestArith}, languages that can be accepted by almost-surely
accepting \acp{PBA} do not have an immediate (more accessible) formulation.

On the other hand, we consider solving the synthesis question for an
antagonistic domain against a probabilistic specification an intriguing result.
It allows to contain the probabilistic aspects of the problem within the
specification. This is especially interesting since the probabilistic behavior
of the specification conceptually distinguishes it from e.g. $\omega$-regular
languages. Additionally, observing probabilistic specifications for
\acp{POMDP}, we are able to push the probabilistic behavior into the structure
of the \ac{POMDP} cleansing the specification from it. This embeds nicely into
the de-randomisation reductions of \cite{RandomnessForFree} where is also noted
that strategies for \eve{} can be cleansed from randomisation.
