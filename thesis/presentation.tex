\documentclass{beamer}

\usetheme{metropolis}

% inputenc:
\usepackage[utf8]{inputenc}

\usepackage{xcolor}

% acronyms
\usepackage{acro}

\DeclareAcronym{MDP}{
  short = MDP ,
  long  = Markov Decision Process ,
  class = environment ,
  long-plural  = es
}
\DeclareAcronym{POMDP}{
  short = POMDP,
  long = Partially Observable Markov Decision Process ,
  class = environment ,
  long-plural = es
}
\DeclareAcronym{MC}{
  short = MC ,
  long  = Markov Chain ,
  class = environment ,
  long-plural  = es
}

\DeclareAcronym{BTA}{
  short = BTA ,
  long  = Büchi Tree Automaton ,
  class = automaton ,
  long-plural-form = Büchi Tree Automaton 
}
\DeclareAcronym{RTA}{
  short = RTA ,
  long  = Rabin Tree Automaton ,
  class = automaton ,
  long-plural-form = Rabin Tree Automata 
}
\DeclareAcronym{MTA}{
  short = MTA ,
  long  = Muller Tree Automaton ,
  class = automaton ,
  long-plural-form = Muller Tree Automata 
}
\DeclareAcronym{PTA}{
  short = PTA ,
  long  = Parity Tree Automaton ,
  class = automaton ,
  long-plural-form = Parity Tree Automata
}

\DeclareAcronym{ABTA}{
  short = ABTA ,
  long  = Alternating Büchi Tree Automaton ,
  class = automaton ,
  long-plural-form = Alternating Büchi Tree Automata
}
\DeclareAcronym{ARTA}{
  short = ARTA ,
  long  = Alternating Rabin Tree Automaton ,
  class = automaton ,
  long-plural-form = Alternating Rabin Tree Automata
}
\DeclareAcronym{AMTA}{
  short = AMTA ,
  long  = Alternating Muller Tree Automaton ,
  class = automaton ,
  long-plural-form = Alternating Muller Tree Automata
}
\DeclareAcronym{APTA}{
  short = APTA ,
  long  = Alternating Parity Tree Automaton ,
  class = automaton ,
  long-plural-form = Alternating Parity Tree Automata
}

\DeclareAcronym{WDTA}{
  short = WDTA ,
  long  = Weighted Descent Tree Automaton ,
  class = automaton ,
  long-plural-form = Weighted Descent Tree Automata
}
\DeclareAcronym{PWA}{
  short = PWA ,
  long  = Probabilistic Weighted Automaton ,
  class = automaton ,
  long-plural-form = Probabilistic Weighted Automata
}

\DeclareAcronym{PBA}{
  short = PBA ,
  long  = Probabilistic Büchi Automaton ,
  class = automaton ,
  long-plural-form = Probabilistic Büchi Automata
}
\DeclareAcronym{PRA}{
  short = PRA ,
  long  = Probabilistic Rabin Automaton ,
  class = automaton ,
  long-plural-form = Probabilistic Rabin Automata
}
\DeclareAcronym{PMA}{
  short = PMA ,
  long  = Probabilistic Muller Automaton ,
  class = automaton ,
  long-plural-form = Probabilistic Muller Automata
}
\DeclareAcronym{PPA}{
  short = PPA ,
  long  = Probabilistic Parity Automaton ,
  class = automaton ,
  long-plural-form = Probabilistic Parity Automata
}

\DeclareAcronym{DBA}{
  short = DBA ,
  long  = Deterministic Büchi Automaton ,
  class = automaton ,
  long-plural-form = Deterministic Büchi Automata
}
\DeclareAcronym{NBA}{
  short = NBA ,
  long  = Non-Deterministic Büchi Automaton ,
  class = automaton ,
  long-plural-form = Non-Deterministic Büchi Automata
}
\DeclareAcronym{DMA}{
  short = DMA ,
  long  = Deterministic Muller Automaton ,
  class = automaton ,
  long-plural-form = Deterministic Muller Automata
}
\DeclareAcronym{NMA}{
  short = NMA ,
  long  = Non-Deterministic Muller Automaton ,
  class = automaton ,
  long-plural-form = Non-Deterministic Muller Automata
}
\DeclareAcronym{DRA}{
  short = DRA ,
  long  = Deterministic Rabin Automaton ,
  class = automaton ,
  long-plural-form = Deterministic Rabin Automata
}
\DeclareAcronym{NRA}{
  short = NRA ,
  long  = Non-Deterministic Rabin Automaton ,
  class = automaton ,
  long-plural-form = Non-Deterministic Rabin Automata
}
\DeclareAcronym{DPA}{
  short = DPA ,
  long  = Deterministic Parity Automaton ,
  class = automaton ,
  long-plural-form = Deterministic Parity Automata
}
\DeclareAcronym{NPA}{
  short = NPA ,
  long  = Non-Deterministic Parity Automaton ,
  class = automaton ,
  long-plural-form = Non-Deterministic Parity Automata
}

\DeclareAcronym{LAR}{
  short = LAR ,
  long  = Latest Appearance Record ,
  class = auxilliary
}

\DeclareAcronym{SCC}{
  short = SCC ,
  long  = Strongly-Connected Component ,
  class = auxilliary
}

\DeclareAcronym{POSG}{
  short = POSG ,
  long  = Partial Observation Stochastic Game ,
  class = game
}


\usepackage{amssymb}
\usepackage{pifont}
\usepackage{amsmath}
\usepackage{amsthm}
\usepackage{bbm}

\newcommand{\pathfinder}{\textsc{Pathfinder}}
\newcommand{\automaton}{\textsc{Automaton}}
\newcommand{\spoiler}{\textsc{Spoiler}}
\newcommand{\acceptor}{\textsc{Acceptor}}
\newcommand{\inputp}{\textsc{Input}}
\newcommand{\outputp}{\textsc{Output}}
\newcommand{\adam}{\textsc{Adam}}
\newcommand{\eve}{\textsc{Eve}}
\newcommand{\random}{\textsc{Random}}

\usepackage{array}
\usepackage{csquotes}

% math operators
\DeclareMathOperator{\cyl}{cyl}
\DeclareMathOperator{\plays}{Plays}
\DeclareMathOperator{\follow}{follow}
\DeclareMathOperator{\parity}{par}
\DeclareMathOperator{\supp}{support}
\DeclareMathOperator{\Inf}{Inf}
\DeclareMathOperator{\prolong}{prolong}
\DeclareMathOperator{\cobuechi}{co-B\ddot{u}chi}
\DeclareMathOperator{\buechi}{B\ddot{u}chi}
\DeclareMathOperator{\muller}{Muller}
\DeclareMathOperator{\reach}{Reachability}
\DeclareMathOperator{\safety}{Safety}
\DeclareMathOperator{\lar}{LAR}
\DeclareMathOperator{\perm}{Perm}
\DeclareMathOperator{\up}{update}
\DeclareMathOperator{\Syn}{Syn}
\DeclareMathOperator{\compRuns}{compRuns}
\DeclareMathOperator{\Runs}{Runs}
\DeclareMathOperator{\Paths}{Paths}
\DeclareMathOperator{\Acc}{Acc}
\DeclareMathOperator{\inner}{inner}
\DeclareMathOperator{\Pot}{Pot}
\DeclareMathOperator{\upK}{updateKnowledge}
\DeclareMathOperator{\currK}{currentKnowledge}
\DeclareMathOperator{\lift}{lift}
\DeclareMathOperator{\obs}{obs}
\DeclareMathOperator{\inpout}{io}
\DeclareMathOperator{\io}{io}

\newcommand{\tuple}[1]{\left(#1\right)}
\newcommand{\set}[1]{\left\{#1\right\}}
\newcommand{\size}[1]{\left\vert{#1}\right\vert}
\newcommand{\interval}[1]{\left[#1\right]}



\title[Syn. for prob. Env.]{
  Automata-theoretic Synthesis for Probabilistic Environments}

\titlegraphic{
  \resizebox{0.9\textwidth}{!}{
    \includegraphics{tikz/titlepic.pdf}
  }
}

\date{\today}
\author{Christoph Welzel}
\institute{Informatik 7, RWTH Aachen}

\metroset{
  sectionpage = simple,
  numbering = fraction
}

\begin{document}

  \acuseall

  \maketitle

  \section{Word-Automata}
  \begin{frame}
    \frametitle{$\omega$-regular Languages}
    \begin{theorem}
      The class of recognizable languages coincides for \alert<2>{\acp{NBA}},
      \alert<3>{\acp{NPA} and \acp{DPA}} and is called $\omega$-regular
      languages. \alert<4>{\acp{DBA} are strictly less expressive.}
      \uncover<3->{\alert<3>{However, \ac{NBA}-determinisation is inherently
      costly ($>n!$).}}
    \end{theorem}
    \begin{overlayarea}{\textwidth}{0.5\textheight}
        \begin{equation*}
          \mathcal{L} = \set{
            \alpha\in\set{a,b}^{\omega}\mid a\notin\Inf(\alpha)
          }
        \end{equation*}
        \begin{columns}
          \begin{column}{0.3\textwidth}<2->
            \includegraphics{tikz/finabuechi.pdf}
          \end{column}
          \begin{column}{0.3\textwidth}<3->
            \includegraphics{tikz/finaparity.pdf}
          \end{column}
          \begin{column}{0.3\textwidth}<4->
            \begin{align*}
              &b^{\omega},\\
              &b^{n_{0}}ab^{\omega},\\
              &b^{n_{0}}ab^{n_{1}}ab^{\omega},\\
              &\vdots
            \end{align*}
          \end{column}
        \end{columns}
    \end{overlayarea}
  \end{frame}

  \begin{frame}
    \frametitle{Probabilistic Büchi Automata}
    \begin{equation*}
      \mathcal{A} = \tuple{Q, \Sigma,
      \delta:Q\times\Sigma\times Q\rightarrow\interval{0,1}, q_{0}, F}
    \end{equation*}
    \begin{overlayarea}{\textwidth}{0.7\textheight}
      \begin{onlyenv}<1>
        \begin{itemize}
          \item $Q$: finite state set
          \item $\Sigma$: finite alphabet
          \item $q_{0}\in Q$: initial state
          \item $\delta:Q\times\Sigma\times Q\rightarrow \interval{0,1}$: 
            transition probability function
          \item $F\subseteq Q$: final states
        \end{itemize}
      \end{onlyenv}
      \begin{onlyenv}<2>
        \begin{center}
          \resizebox{!}{0.6\textheight}{%
            \includegraphics{tikz/pbaruntree.pdf}
          }
        \end{center}
      \end{onlyenv}
      \begin{onlyenv}<3->
        \begin{columns}
          \begin{column}{0.5\textwidth}
            \begin{itemize}
              \item<3-> cylindric sets: $\cyl(u) = 
                \set{u\cdot\alpha:\alpha\in Q^{\omega}}$
            \end{itemize}
            \begin{center}
              \resizebox{0.8\textwidth}{!}{%
                \includegraphics{tikz/verticalcylinder.pdf}
              }
            \end{center}
          \end{column}
          \begin{column}{0.5\textwidth}
            \begin{itemize}
              \item<4-> $\alpha$ induces probability space 
                \begin{equation*}
                  \tuple{Q^{\omega}, \mathcal{B}(Q), \mu_{\alpha}}
                \end{equation*}
              \item<5-> positive: $\mu_{\alpha}(\Acc_{\buechi}(F)) > 0$
              \item<6-> almost-sure: $\mu_{\alpha}(\Acc_{\buechi}(F)) = 1$
            \end{itemize}
          \end{column}
        \end{columns}
      \end{onlyenv}
    \end{overlayarea}
  \end{frame}

  \begin{frame}
    \frametitle{Probabilistic Automata - Examples}
    \begin{columns}
      \begin{column}{0.5\textwidth}
        \small{Positive Acceptance}
        \begin{center}
          \resizebox{0.6\textwidth}{!}{%
            \includegraphics{tikz/posacceptingpba.pdf}
          }
        \end{center}
        \begin{equation*}
          \underbrace{
              \set{a^{k_{1}}ba^{k_{2}}b\dots:
              \substack{k_{i}>0\text{ for all }i>0,\\
              \prod_{i>0}\tuple{1-\frac{1}{2}^{k_{i}}} > 0}
            }
          }_{\mathcal{L}_{1}}
        \end{equation*}
      \end{column}
      \begin{column}<2->{0.5\textwidth}
        \small{Almost-Sure Acceptance}
        \begin{center}
          \resizebox{0.6\textwidth}{!}{%
            \includegraphics{tikz/almostsureacceptingpbapresentation.pdf}
          }
        \end{center}
        \begin{equation*}
          \underbrace{
            \set{
              a^{k_{1}}ba^{k_{2}}b\dots:
              \substack{k_{i}>0\text{ for all }i>0\\
              \prod_{i>0}\tuple{1-\frac{1}{2}^{k_{i}}} = 0}
            }
          }_{\mathcal{L}_{2}}
        \end{equation*}
      \end{column}
    \end{columns}
    \begin{uncoverenv}<3->
      \begin{equation*}
        \overline{\mathcal{L}_{1}} = \mathcal{L}_{2}\cup\underbrace{
          b\Sigma^{\omega} + \Sigma^{*}bb\Sigma^{\omega} + \Sigma^{*}a^{\omega}
        }_{\omega-\text{regular}}
      \end{equation*}
    \end{uncoverenv}
  \end{frame}

  \begin{frame}
    \frametitle{Probabilistic Automata - Properties}
    \begin{columns}
      \begin{column}{0.5\textwidth}
        \small{Positive Acceptance}
        \begin{itemize}
          \item<2-> \alert<3>{strictly} subsumes $\omega$-regular
          \item<4-> recognizable languages form Boolean-algebra
          \item<5-> undecidable emptiness
        \end{itemize}
        \small{Almost-Sure Acceptance}
        \begin{itemize}
          \item<6-> incomparable with $\omega$-regular
          \item<9-> decidable emptiness
          \item<10-> \alert{Parity-}condition coincides with positive
            acceptance of Büchi- or Parity-condition
        \end{itemize}
      \end{column}
      \begin{column}{0.5\textwidth}
        \begin{overlayarea}{\textwidth}{\textheight}
          \begin{onlyenv}<3>
            \begin{uncoverenv}
              \vspace{3cm}
              \begin{equation*}
                \set{
                  a^{k_{1}}ba^{k_{2}}b\dots:
                  \substack{k_{i}>0\text{ for all }i>0,\\
                  \prod_{i>0}\tuple{1-\frac{1}{2}^{k_{i}}} > 0}
                }
              \end{equation*}
            \end{uncoverenv}
          \end{onlyenv}
          \begin{onlyenv}<7-8>
            \vspace{0.2\textheight}
            \begin{equation*}
              \set{\alpha\in\set{a,b}^{\omega}\mid a\notin\Inf(\alpha)}
            \end{equation*}
            \vfill
            \begin{uncoverenv}<8>
              \begin{align*}
                &\overline{\set{
                  a^{k_{1}}ba^{k_{2}}b\dots:
                  \substack{k_{i}>0\text{ for all }i>0,\\
                  \prod_{i>0}\tuple{1-\frac{1}{2}^{k_{i}}} > 0}
                }}\\
                &= \set{
                  a^{k_{1}}ba^{k_{2}}b\dots:
                  \substack{k_{i}>0\text{ for all }i>0\\
                  \prod_{i>0}\tuple{1-\frac{1}{2}^{k_{i}}} = 0}
                }\\
                &\cup b\Sigma^{\omega} + \Sigma^{*}bb\Sigma^{\omega}
                + \Sigma^{*}a^{\omega}\\
              \end{align*}
            \end{uncoverenv}
          \end{onlyenv}
        \end{overlayarea}
      \end{column}
    \end{columns}
  \end{frame}

  \section{Tree-Automata}
  \begin{frame}
    \frametitle{Definition}
    \begin{overlayarea}{\textwidth}{0.2\textheight}
      \begin{equation*}
        \mathcal{A} = \tuple{Q, q_{0}, D, \Sigma, \Delta,
          \Acc\subseteq Q^{\omega}}
      \end{equation*}
    \end{overlayarea}
    \begin{overlayarea}{\textwidth}{0.6\textheight}
      \begin{onlyenv}<1>
        \begin{itemize}
          \item $Q$: finite state set
          \item $q_{0}\in Q$: initial state
          \item $D$: finite set of directions
          \item $\Sigma$: finite alphabet
          \item $\Delta$: finite set of transitions
          \item $\Acc$: accepted language
        \end{itemize}
      \end{onlyenv}
      \begin{onlyenv}<2-5>
        $D$-ary $\Sigma$-tree: $t:D^{*}\rightarrow\Sigma$\hfill
        \uncover<3->{$D$-ary $Q$-run: $r:D^{*}\rightarrow Q$}
        \begin{center}
          \resizebox{!}{0.5\textheight}{
            \alt<2>{
              \includegraphics{tikz/tree.pdf}
            }{
              \alt<3>{
                \includegraphics{tikz/run.pdf}
              }{
                \alt<4>{
                  \includegraphics{tikz/weightedpathsruns.pdf}
                }{
                  \includegraphics{tikz/weightedpathsrunsmarked.pdf}
                }
              }
            }
          }
        \end{center}
      \end{onlyenv}
    \end{overlayarea}
    \begin{overlayarea}{\textwidth}{0.2\textheight}
      \uncover<5>{
        \begin{equation*}
          \tuple{D^{\omega}, \mathcal{B}(D), \mu_{r}}\text{ with }\mu_{r}(
          \set{\rho\in D^{\omega}\mid r(\rho)\in\Acc})
        \end{equation*}
      }
    \end{overlayarea}
  \end{frame}

  \begin{frame}
    \frametitle{Tree Automata - Example}
    \begin{overlayarea}{\textwidth}{0.1\textheight}
      \begin{equation*}
        \mathcal{A} = (Q = \set{q_{a}, q_{b}}, q_{a}, D = \set{0,1},
          \Sigma = \set{a, b}, \Delta, F = \set{q_{a}})
      \end{equation*}
    \end{overlayarea}
    \begin{overlayarea}{\textwidth}{0.9\textheight}
          \uncover<2->{
            \begin{equation*}
              \Delta = \set{
                \tuple{q, \sigma, q_{\sigma}, q_{\sigma}}:\substack{
                  q\in Q,\\
                  \sigma\in\set{a,b}
                }
              }\text{ or }
              \set{
                \tuple{q, \sigma, q_{\sigma}, \frac{1}{2}, q_{\sigma},
                \frac{1}{2}}:\substack{
                  q\in Q,\\
                  \sigma\in\set{a,b}
                }
              }
            \end{equation*}
          }
          \uncover<3->{
            \begin{equation*}
              \mathcal{L} = 
              \set{
                \begin{aligned}
                  t:&\set{0,1}^{*}\rightarrow\set{a,b}\mid\\&
                  A^{t}_{\infty} = D^{\omega}
                \end{aligned}
              }\text{ or }
              \set{
                \begin{aligned}
                  t:&\set{0,1}^{*}\rightarrow\set{a,b}\mid\\&
                    \mu_{\frac{1}{2}}(A^{t}_{\infty}) = 1\;
                  (\mu_{\frac{1}{2}}(A^{t}_{\infty}) > 0)
                \end{aligned}
              }
            \end{equation*}
            where
            \begin{equation*}
              A^{t}_{\infty} = \set{\alpha_{1}\alpha_{2}\dots\in D^{\omega}\mid
                a\in\Inf(t(\epsilon)t(\alpha_{1})t(\alpha_{1}\alpha_{2})\dots)}
            \end{equation*}
          }
    \end{overlayarea}
  \end{frame}

  % \begin{frame}
  %   \frametitle{Tree Automata - Example}
  %   \begin{equation*}
  %     \mathcal{L} = \set{
  %       t:\set{0,1}^{*}\rightarrow\set{a,b}\mid\substack{
  %         a\notin \Inf(t(\epsilon)t(\alpha_{1})t(\alpha_{1}\alpha_{2})\dots)\\
  %         \text{for all }\alpha_{1}\alpha_{2}\dots\in\set{0,1}^{\omega} 
  %       }
  %     }
  %   \end{equation*}
  %     \begin{columns}
  %       \begin{column}{0.4\textwidth}
  %         \vspace{-3cm}
  %         \begin{itemize}
  %           \item \alert<2-3>{Parity-condition} \uncover<3->{\checkmark}
  %           \item \alert<4->{Büchi-condition} \uncover<5->{\ding{55}}
  %         \end{itemize}
  %       \end{column}
  %       \begin{column}{0.6\textwidth}
  %         \begin{overlayarea}{\textwidth}{0.8\textheight}
  %           \begin{onlyenv}<3>
  %             \begin{align*}
  %               \mathcal{A} = (\set{q_{a}, q_{b}}, q_{a}, \set{0,1},
  %               \set{a, b}&,\\
  %               \Delta, \set{q_{a}\mapsto 1, q_{b}\mapsto 0}&)
  %             \end{align*}
  %             with
  %             \begin{equation*}
  %               \Delta = \set{
  %                 \tuple{q, \sigma, q_{\sigma}, q_{\sigma}}:\substack{
  %                   q\in Q,\\
  %                   \sigma\in\set{a,b}
  %                 }
  %               }
  %             \end{equation*}
  %           \end{onlyenv}
  %           \begin{onlyenv}<4->
  %             \begin{center}
  %               % \resizebox{0.6\textwidth}{!}{%
  %                 \includegraphics{tikz/finatree.pdf}
  %               % }
  %             \end{center}
  %           \end{onlyenv}
  %         \end{overlayarea}
  %       \end{column}
  %     \end{columns}
  % \end{frame}

  \begin{frame}
    \frametitle{Alternating Tree Automata - Definition}
    \begin{overlayarea}{\textwidth}{0.2\textheight}
      \begin{equation*}
        \mathcal{A} = \tuple{Q, q_{0}, D, \Sigma, \Delta,
          \Acc\subseteq Q^{\omega}}
      \end{equation*}
    \end{overlayarea}
    \begin{overlayarea}{\textwidth}{0.5\textheight}
      \begin{onlyenv}<2-4>
        \begin{center}
          \alt<2>{
            \includegraphics{tikz/bintree.pdf}
          }{
            \alt<3>{
              \includegraphics{tikz/alternatingrun.pdf}
            }{
              \includegraphics{tikz/weightedalternatingrun.pdf}
            }
          }
        \end{center}
      \end{onlyenv}
    \end{overlayarea}
    \uncover<4>{
      \begin{equation*}
        \Rightarrow\tuple{\tuple{Q\times D}^{\omega}, \mathcal{B}(Q\times D),
          \mu_{r}}
      \end{equation*}
    }
  \end{frame}

  \begin{frame}
    \frametitle{Alternating Tree Automaton - Example}
    \begin{equation*}
      \set{t:\set{0,1}^{*}\rightarrow\set{a,b}\mid\substack{
        \text{there is }u\in\set{0,1}^{*}\\
        \text{ with }t(u) = a}
      }
    \end{equation*}
    \begin{columns}
      \begin{column}{0.5\textwidth}
        \begin{itemize}
          \item search and found state: $q_{s}, q_{f}$
          \item<2-> non-deterministically move $q_{s}$
          \item<3-> transform into positive sink-state $q_{f}$ when reading $a$
          \item<4-> use $\set{q_{f}}$ as Büchi-condition
        \end{itemize}
      \end{column}
      \begin{column}{0.5\textwidth}
        \begin{center}
          \uncover<5>{\includegraphics{tikz/searcharun.pdf}}
        \end{center}
      \end{column}
    \end{columns}
  \end{frame}

  \begin{frame}
    \frametitle{Weighted Descent Tree Automata - Structural Properties}
    \begin{tabular}{llm{0.5\textwidth}}
      choiceless (c.l.): & $\substack{\size{\set{G: \tuple{q, \sigma, G}
      \in\Delta}} = 1\\\text{ for all }q\in Q, \sigma\in\Sigma}$ & \\
      uni-directional (u.d.): & $\substack{\text{for every }
      G\in\mathcal{G}(\mathcal{A}),d\in D\\\text{ exists \emph{at most} one }
      p\in Q\\\text{with } G(d, p) > 0}$ & \resizebox{0.4\textwidth}{!}{
        \includegraphics{tikz/unidirectional.pdf}}\\
    \end{tabular}
  \end{frame}

  \begin{frame}
    \frametitle{Tree Automata - Properties}
    \begin{theorem}[Simulation Theorem]
      There is an effective construction which, when given an \ac{APTA},
      produces an equivalent \ac{PTA}. Furthermore, given an \ac{ABTA}, there
      is a way to effectively construct an equivalent \ac{BTA}.
    \end{theorem}
    \begin{itemize}
      \item recognizable languages form a Boolean-algebra
      \item decidable emptiness
    \end{itemize}
  \end{frame}

  % \begin{frame}
    % \frametitle{Weighted Descent Tree Automata - Idea}
    % \begin{itemize}
    %   \item non-deterministically construct run
    %   \item<2-> weight individual paths
    %   \item<3-> acceptance by measure
    % \end{itemize}
    % \begin{overlayarea}{\textwidth}{0.6\textheight}
    %   \begin{center}
    %     \alt<1>{
    %       \includegraphics{tikz/weightedrun.pdf}
    %     }{
    %       \alt<2>{
    %         \includegraphics{tikz/weightedpathsruns.pdf}
    %       }{
    %         \begin{columns}
    %           \begin{column}{0.5\textwidth}
    %             \begin{equation*}
    %               \mu_{r}(\cyl(u_{1}\dots u_{n})) = \prod_{1\leq i\leq n}
    %                 p_{u_{1}\dots u_{i}}
    %             \end{equation*}
    %             \uncover<4->{
    %               \begin{equation*}
    %                 \Rightarrow\tuple{D^{\omega}, \mathcal{B}(D), \mu_{r}}
    %               \end{equation*}
    %               with
    %               \begin{equation*}
    %                 \mu_{r}(\set{\rho\in D^{\omega}\mid r(\rho)\in\Acc})
    %               \end{equation*}
    %             }
    %           \end{column}
    %           \begin{column}{0.5\textwidth}
    %             \resizebox{\textwidth}{!}{
    %               \includegraphics{tikz/weightedpathsrunsmarked.pdf}
    %             }
    %           \end{column}
    %         \end{columns}
    %       }
    %     }
    %   \end{center}
    % \end{overlayarea}
  % \end{frame}

  % \begin{frame}
  %   \frametitle{Weighted Descent Tree Automata - Definition}
  %   \begin{equation*}
  %     \mathcal{A} = \tuple{Q, q_{0}, D, \Sigma, \Delta, \Acc}
  %   \end{equation*}
  %   \begin{overlayarea}{\textwidth}{0.8\textheight}
  %     \begin{onlyenv}<2>
  %       \begin{itemize}
  %         \item $Q$: finite state set
  %         \item $q_{0}\in Q$: initial state
  %         \item $D$: finite set of directions
  %         \item $\Sigma$: finite alphabet
  %         \item $\Delta$: transitions of the form $\tuple{q, \sigma, G}$
  %         \item $\Acc$: accepted language
  %       \end{itemize}
  %     \end{onlyenv}
  %     \begin{onlyenv}<3>
  %       \includegraphics{tikz/weightedalternatingrun.pdf}
  %     \end{onlyenv}
  %     \begin{onlyenv}<4->
  %       \begin{columns}
  %         \begin{column}{0.5\textwidth}
  %           \begin{itemize}
  %             \item run: $Q\times D$-ary 
  %               $\mathcal{G}(\mathcal{A})$-tree
  %             \item[$\Rightarrow$] $
  %               \tuple{\tuple{Q\times D}^{\omega}, 
  %               \mathcal{B}(Q\times D), \mu_{r}}
  %               $
  %           \end{itemize}
  %         \end{column}
  %         \begin{column}{0.5\textwidth}
  %           \resizebox{\textwidth}{!}{
  %             \includegraphics{tikz/weightedalternatingrun.pdf}
  %           }
  %         \end{column}
  %       \end{columns}
  %       \begin{equation*}
  %         \Acc\interval{Q} = \set{
  %           \tuple{d_{1}, q_{1}}\tuple{d_{2}, q_{2}}\dots\in
  %           \tuple{D\times Q}^{\omega}\mid q_{1}q_{2}\dots\in\Acc
  %         }
  %       \end{equation*}
  %       and
  %       \begin{equation*}
  %         \mu_{r}(\Acc\interval{Q}) > 0\text{ or }\mu_{r}(\Acc\interval{Q}) = 1
  %       \end{equation*}
  %     \end{onlyenv}
  %   \end{overlayarea}
  % \end{frame}

  \begin{frame}
    \frametitle{Weighted Descent Tree Automata - Word-Automata}
    for unary trees:
    \begin{itemize}
      \item<2-> c.l. \acp{WDTA} \enquote{equivalent} to \acp{PPA}
      \item<3-> u.d. \acp{WDTA} \enquote{equivalent} to $\omega$-regular
      \item[$\Rightarrow$]<4-> undecidable emptiness for (c.l.) \acp{WDTA}
        with positive Büchi-acceptance and almost-sure Parity-acceptance
      \item[$\Rightarrow$]<5-> Simulation Theorem does not translate to
        \acp{WDTA}
    \end{itemize}
    \begin{overlayarea}{\textwidth}{0.5\textheight}
      \begin{center}
        \begin{onlyenv}<2>
          \resizebox{!}{0.5\textheight}{
            \includegraphics{tikz/pbaruntree.pdf}
          }
        \end{onlyenv}
        \begin{onlyenv}<3>
          \resizebox{!}{0.5\textheight}{
            \includegraphics{tikz/unaryunidirectionalrun.pdf}
          }
        \end{onlyenv}
      \end{center}
    \end{overlayarea}
    % \begin{overlayarea}{\textwidth}{0.5\textheight}
    %   \begin{onlyenv}<4-8>
    %     \begin{itemize}
    %       \item<4-> $\mathcal{P} = \tuple{Q, \Sigma, q_{0}, \delta, \parity}
    %         \Rightarrow \mathcal{A} = \tuple{Q, q_{0}, \set{0}, \Sigma, \Delta,
    %         \parity}$
    %       \item<5-> $\Delta = \set{\tuple{q, \sigma,
    %         \set{\tuple{p,0}\mapsto\delta(q,\sigma,p):p\in Q}}: q\in Q,
    %         \sigma\in\Sigma}$
    %       \item<8-> unique run of $\mathcal{A}$ and run-tree of $\mathcal{P}$
    %         coincide
    %       \item<6-> $\mathcal{A} = \tuple{Q, q_{0}, \set{0}, \Sigma, \Delta,
    %         \parity}\Rightarrow\mathcal{P} = \tuple{Q, \Sigma, q_{0}, \delta,
    %         \parity}$
    %       \item<7-> $\delta(q, \sigma, p) = G_{\sigma}^{q}(p, 0)$
    %     \end{itemize}
    %   \end{onlyenv}
    %   \begin{onlyenv}<9-12>
    %     \begin{itemize}
    %       \item<10-> uni-directional \acp{WDTA} over singleton direction set
    %         only weight one following state
    %       \item<11-> every run induces Dirac measure
    %       \item[$\Rightarrow$]<12-> acceptance only dependent on the
    %         non-deterministically chosen state sequence
    %     \end{itemize}
    %   \end{onlyenv}
    % \end{overlayarea}
  \end{frame}

  \begin{frame}
    \frametitle{Partially Observable Markov Decision Processes}
    \begin{equation*}
      \mathcal{S} = \tuple{S, s_{0}, A, \tuple{\tau_{a}}_{a\in A}, \sim}
    \end{equation*}
    \begin{overlayarea}{\textwidth}{0.8\textheight}
      \begin{onlyenv}<2>
        \begin{itemize}
          \item $S$: finite state set
          \item $s_{0}\in S$: initial state
          \item $A$: finite state of actions
          \item $\tau_{a}\in\mathcal{D}(S\times S)$: transition probabilities
          \item $\sim$: observable equivalence classes
          \item strategy $f:\interval{S}_{\sim}^{*}\rightarrow A$
          \item[$\Rightarrow$] $\tuple{S^{\omega}, \mathcal{B}(S), \mu_{f}}$
        \end{itemize}
      \end{onlyenv}
      \begin{onlyenv}<3-6>
        \begin{columns}
          \begin{column}{0.6\textwidth}
            \uncover<4->{
              \alt<4-5>{
                \begin{align*}
                  f(\epsilon) &= a\\
                  f(p_{1}\dots p_{n}) &= \begin{cases}
                    a&\text{if }p_{n} = q_{0},\\
                    b&\text{otherwise.}
                  \end{cases}
                \end{align*}
              }{
                \begin{equation*}
                  \mu_{f}(\Acc_{\buechi}(F)) > 0 \text{ if and only if }
                  f\in\mathcal{L}(\mathcal{P}).
                \end{equation*}
              }
            }
          \end{column}
          \begin{column}{0.4\textwidth}
            \begin{center}
              \alt<3-4>{
                \includegraphics{tikz/differentlycoloredpba.pdf}
              }{
                \includegraphics{tikz/uniformlycoloredpba.pdf}
              }
            \end{center}
          \end{column}
        \end{columns}
      \end{onlyenv}
      \begin{onlyenv}<7>
        \begin{center}
          \begin{tabular}{|l|ll|}
            \hline
            Computing Strategies & Fully observable & Partially observable\\
            \hline
            Positive Büchi & \textsc{PTIME} & Undecidable \\
            Almost-Sure Büchi & \textsc{PTIME} & \textsc{EXPTIME} \\
            Positive Parity & \textsc{PTIME} & Undecidable \\
            Almost-Sure Parity & \textsc{PTIME} & Undecidable \\
            \hline
          \end{tabular}
        \end{center}
      \end{onlyenv}
    \end{overlayarea}
  \end{frame}

  \begin{frame}
    \frametitle{\acp{WDTA} and \acp{POMDP}}
    \begin{equation*}
      \mathcal{A} = \tuple{Q, q_{0}, D, \Sigma, \Delta, \parity}
      \text{ and }
      \mathcal{S} = \tuple{S, s_{0}, A, \tuple{\tau_{a}}_{a\in A}, \sim}
    \end{equation*}
    \begin{overlayarea}{\textwidth}{0.8\textheight}
      \begin{columns}
        \begin{column}{0.35\textwidth}
          \begin{overlayarea}{\textwidth}{\textheight}
            \begin{itemize}
              \item<2-> strategies are
                $\interval{S}_{\sim}$-ary $A$-trees
              \item<3-> translate \acs{POMDP} to c.l. \acp{WDTA}
              \item<4-> use \acp{POMDP} as emptiness game for c.l. \acp{WDTA}
            \end{itemize}
          \end{overlayarea}
        \end{column}
        \begin{column}{0.65\textwidth}
          \vspace{-3cm}
          \begin{onlyenv}<3>
            \begin{center}
              \enquote{directions are observations}
            \end{center}
            \resizebox{\textwidth}{!}{
              \includegraphics{tikz/pomdpaswdta.pdf}
            }
            % \begin{itemize}
            %   \item<5-> $Q = S$ with $q_{0} = s_{0}$
            %   \item<6-> $D = \interval{S}_{\sim}$ and $\Sigma = A$
            %   \item<7-> $G_{a}^{s}(s', d) = \begin{cases}
            %       \tau_{a}(s, s')&\text{if }d = \interval{s'}_{\sim},\\
            %       0&\text{otherwise}.
            %   \end{cases}$
            % \item<8-> if $\sim\equiv =$ then $\mathcal{A}$ is u.d.
            % \end{itemize}
          \end{onlyenv}
          \begin{onlyenv}<4>
            \begin{center}
              \enquote{play the run}
            \end{center}
            \resizebox{\textwidth}{!}{
              \includegraphics{tikz/weightedalternatingrunreduced.pdf}
            }
            % \begin{itemize}
            %   \item<11-> $S = Q\times D\cup\set{q_{0}}$ with
            %     $s_{0} = q_{0}$
            %   \item<12-> $A = \Sigma$
            %   \item<13-> $\tau_{\sigma}(\tuple{q, d}, \tuple{p, d'}) = 
            %     G_{q}^{\sigma}(p, d')$
            %   \item<14-> $\tuple{q, d}\sim\tuple{p, d'}$ iff $d = d'$
            %   \item<15-> compute strategy for $\parity$ on $\mathcal{S}$
            % \end{itemize}
          \end{onlyenv}
        \end{column}
      \end{columns}
    \end{overlayarea}
  \end{frame}

  \begin{frame}
    \frametitle{Weighted Descent Tree Automata - Overview}
    \resizebox{\textwidth}{!}{\includegraphics{tikz/wdtamapreduced.pdf}}
  \end{frame}

  \section{Synthesis}
  \begin{frame}
    \frametitle{Setting}
    \begin{center}
      \alt<2->{
        \begin{definition}[Synthesis Problem]
          Given a logic $\mathbb{L}$. Compute for every formula
          \alert{$\phi(\cdot, \cdot)\in\mathbb{L}$} over inputs $I$ and outputs
          $J$ an algorithm \alert{$S:I^{+}\rightarrow J$} such that
          $\phi(\alpha, S(\alpha_{1})S(\alpha_{1}\alpha_{2})\dots)$ is true for
          all $\alpha_{1}\alpha_{2}\dots\in I^{\omega}$ or prove that such an
          $S$ cannot exist.
        \end{definition}
        % $\mathbb{L}$ is $\begin{aligned}
        %   &\omega\text{-regular}\\
        %   &\text{almost-surely \ac{PBA} recognizable}
        % \end{aligned}$
        % if it is possible to effectively construct a language
        % $\mathcal{L}_{\phi}$ which is
        % $\begin{aligned}
        %   &\omega\text{-regular}\\
        %   &\text{almost-surely \ac{PBA} recognizable}
        % \end{aligned}$
        % such that
        \uncover<3>{
          \begin{equation*}
            \tuple{j_{0}, i_{0}}\tuple{j_{1}, i_{1}}\dots\in\mathcal{L}_{\phi}
            \text{ iff }\phi(i_{0}i_{1}\dots, j_{1}j_{2}\dots)\text{ is true.}
          \end{equation*}
        }
        % \begin{itemize}
        %   \item $\mathbb{L}$ is $\omega$-regular if for every
        %     $\phi(\cdot,\cdot)$ it is possible to effectively construct an
        %     $\omega$-regular $\mathcal{L}_{\phi}$ such that
        %     $\tuple{j_{0}, i_{0}}\tuple{j_{1}, i_{1}}\dots\in\mathcal{L}_{\phi}
        %     \text{ iff }\phi(i_{0}i_{1}\dots, j_{1}j_{2}\dots)\text{ is true.}$
        %   \item $\mathbb{L}$ is almost-surely \ac{PBA} recognizable if for
        %     every $\phi(\cdot,\cdot)$ it is possible to effectively constuct a
        %     $\mathcal{L}_{\phi}$ which is recognizable by a \ac{PBA} with
        %     almost-sure acceptance such that
        %     $\tuple{j_{0}, i_{0}}\tuple{j_{1}, i_{1}}\dots\in\mathcal{L}_{\phi}
        %     \text{ iff }\phi(i_{0}i_{1}\dots, j_{1}j_{2}\dots)\text{ is true.}$
        % \end{itemize}
      }{
        \resizebox{0.8\textwidth}{!}{\includegraphics{tikz/synthesis.pdf}}
      }
    \end{center}
  \end{frame}

  \begin{frame}
    \frametitle{Antagonistic Environment - $\omega$-regular Specification}
    \begin{columns}
      \begin{column}{0.5\textwidth}
        \begin{itemize}
          \item $\omega$-regular specification
          \item[$\Rightarrow$]<2-> \ac{DPA} $\tuple{Q, J\times I, \delta,
            q_{0}, \parity}$
          \item[$\Rightarrow$]<3-> \ac{PTA}
            $\tuple{Q, q_{0}, I, J, \Delta, \parity}$ with
          $\Delta = \set{\substack{\tuple{q, i, \tuple{\delta(q,
            \tuple{i, j})}_{j\in J}}:\\q\in Q, i\in I}}$
        \end{itemize}
      \end{column}
      \begin{column}{0.5\textwidth}
        \uncover<4->{
          \includegraphics{tikz/inputoutputtree.pdf}
        }
      \end{column}
    \end{columns}
  \end{frame}

  \begin{frame}
    \frametitle{Antagonistic Environment - \ac{PBA} Specification}
    \begin{onlyenv}<1-7>
      \only<1-6>{
        \begin{equation*}
          \tuple{Q, J\times I, \delta, q_{0}, F}
        \end{equation*}
      }
      \only<7->{
        \begin{equation*}
          \tuple{Q, J\times I, \delta, q_{0}, F}\Rightarrow\tuple{S, s_{0},
          E, A, \tuple{\tau_{e,a}}_{e\in E, a\in A}, \sim_{E}, \sim_{A}, F'}
        \end{equation*}
      }
      \begin{itemize}
        \item<2-> Partially Observable Stochastic Game (\ac{POSG}):
        \item<3-> \ac{POMDP} with antagonistic players (\eve{} \& \adam{})
        \item<4-> $G = \tuple{S, s_{0}, E, A,
          \tuple{\tau_{e,a}}_{e\in E, a\in A}, \sim_{E}, \sim_{A}, F'}$
        \item<5-> 
          $\begin{aligned}
            \eve{}\; &f:\interval{S}_{\sim_{A}}^{*}\rightarrow A\\
            \adam{}\; &g:\interval{S}_{\sim_{E}}^{*}\rightarrow E
          \end{aligned} \Rightarrow \tuple{S^{\omega}, \mathcal{B}(S), 
          \mu_{f,g}}$
        \item<6-> $f$ almost-surely wins for \eve{} if
          $\mu_{f,g}(\Acc_{\buechi}(F)) = 1$ for all $g$
      \end{itemize}
    \end{onlyenv}
    \begin{onlyenv}<8-9>
      \begin{center}
        \enquote{input-/output-game with unobservable stochastic background}
      \end{center}
      % \begin{itemize}
      %   \item<9-> $S = I\times Q\times J\cup\set{q_{0}}$ with $s_{0} = q_{0}$
      %   \item<10-> $E = J$, $A = I$
      %   \item<11-> $\tau_{j,i}(\tuple{a, q, e}, \tuple{i', p, j'}) =
      %     \begin{cases}
      %       \delta(q, \tuple{i,j}, p)&\text{if }i' = i, j' = j,\\
      %       0&\text{otherwise}.
      %     \end{cases}$
      %   \item<12-> $\sim_{E} = \sim_{A} = \set{\substack{
      %     \tuple{\tuple{i, q, j}, \tuple{i, p, j}}:\\
      %     i\in I, j\in J, q, p\in Q}}$
      %   \item<13-> $F' = I\times F\times J$
      % \end{itemize}
      \begin{columns}
        \begin{column}{0.5\textwidth}
          \begin{equation*}
            \set{
              a^{k_{1}}ba^{k_{2}}b\dots:
              \substack{k_{i}>0\text{ for all }i>0\\
              \prod_{i>0}\tuple{1-\frac{1}{2}^{k_{i}}} = 0\\
              a:\text{last input differs from current output}\\
              b:\text{last input is equal to current output}}
            }
          \end{equation*}
          \begin{center}
            \resizebox{\textwidth}{!}{
              \includegraphics{tikz/synthesispbapresentation.pdf}
            }
          \end{center}
        \end{column}
        \begin{column}{0.5\textwidth}
          \begin{center}
            \resizebox{\textwidth}{!}{
              \uncover<9>{\includegraphics{tikz/synthesispbagame.pdf}}
            }
          \end{center}
        \end{column}
      \end{columns}
    \end{onlyenv}
  \end{frame}

  \begin{frame}
    \frametitle{Qualitative Strategy Synthesis}
    \begin{definition}
      Given a \ac{POMDP} $\mathcal{S} = \tuple{S, s_{0}, A,
      \tuple{\tau_{a}}_{a\in A}, \sim}$ and a specification $\phi\subseteq
      S^{\omega}$ such that $\phi\in\mathcal{B}(S)$. The qualitative synthesis
      problem $\tuple{\mathcal{S}, \phi}$ demands the computation of a strategy
      $s:\interval{S}_{\sim}^{*}\rightarrow A$ such that $\mu_{s}(\phi) = 1$.
    \end{definition}
  \end{frame}

  \begin{frame}
    \frametitle{\ac{POMDP} - $\omega$-regular Specification}
    \begin{equation*}
      \mathcal{S} = \tuple{S, s_{0}, A, \tuple{\tau_{a}}_{a\in A}, \sim}
    \end{equation*}
    \begin{itemize}
      \item<2-> $\omega$-regular specification
      \item[$\Rightarrow$]<3-> \ac{DPA} $\mathcal{P}$
      \item<4-> natural product construction $\mathcal{S}\otimes\mathcal{P}$
        yields \ac{POMDP} with associated Parity-condition
    \end{itemize}
    \uncover<5->{
      \begin{center}
        \begin{tabular}{|l|ll|}
          \hline
          Computing Strategies & Fully observable & Partially observable\\
          \hline
          Positive Büchi & \textsc{PTIME} & Undecidable \\
          Almost-Sure Büchi & \textsc{PTIME} & \textsc{EXPTIME} \\
          Positive Parity & \textsc{PTIME} & Undecidable \\
          Almost-Sure Parity & \textsc{PTIME} & Undecidable \\
          \hline
        \end{tabular}
      \end{center}
    }
  \end{frame}

  \begin{frame}
    \frametitle{\ac{POMDP} - \ac{PBA} Specification}
    \begin{overlayarea}{\textwidth}{0.2\textheight}
      \begin{equation*}
        \mathcal{S} = \tuple{S, s_{0}, A, \tuple{\tau_{a}}_{a\in A}, \sim}
        \text{ and }
        \mathcal{P} = \tuple{Q, S, \delta, q_{0}, F}
      \end{equation*}
    \end{overlayarea}
    \begin{overlayarea}{\textwidth}{0.8\textheight}
      \begin{onlyenv}<2-3>
        \begin{equation*}
          \alpha\in S^{\omega}\Rightarrow\tuple{Q, \mathcal{B}(Q), \mu_{s}}
        \end{equation*}
        \uncover<3>{
          \begin{definition}
            Given $\tuple{\Omega_{1},\mathcal{F}_{1}}$,
            $\tuple{\Omega_{2},\mathcal{F}_{2}}$, then
            $K:\Omega_{1}\times\mathcal{F}_{2}\rightarrow \interval{0,1}$ is a
            Markov-kernel if
            \begin{enumerate}
              \item $K(\cdot, A)$ is measurable in $\mathcal{F}_{1}$ for all
                $A\in\mathcal{F}_{2}$,
              \item $K(\omega, \cdot)$ is a probability measure on
                $\tuple{\Omega_{2},\mathcal{F}_{2}}$ for every
                $\omega\in\Omega_{1}$.
            \end{enumerate}
          \end{definition}
        }
      \end{onlyenv}
      \begin{onlyenv}<4-8>
        \begin{equation*}
          K:S^{\omega}\times\mathcal{B}(Q)\rightarrow\interval{0,1}
          \text{ with }
          K(\alpha, A) = \mu_{\alpha}(A)
        \end{equation*}
        \begin{itemize}
          \item<5-> for a strategy $s$:
            $\begin{aligned}
              \int_{\alpha\in S^{\omega}}&K(\alpha, \Acc_{\buechi}(F))d\mu_{s}
              = 1\text{ iff }\\
              &\mu_{\alpha}(\Acc_{\buechi}(F)) = 1\text{ for almost all }\alpha
            \end{aligned}$
          \item<6-> measure $\mu_{s}\otimes K$ on $\tuple{S^{\omega}\times
            Q^{\omega}, \mathcal{B}(S)\otimes\mathcal{B}(Q)}$ uniquely
            determined by $\int_{\cyl(u)}K(\cdot, \cyl(v))d\mu_{s}$
          \item<7-> $\int_{\cyl(u)}K(\cdot, \cyl(v))d\mu_{s} =
            \mu_{s}(\cyl(u))\cdot\mu_{u}(\cyl(v))$ for $\size{u} = \size{v}$
          \item[$\Rightarrow$]<8-> induces product construction for
            $\mathcal{S}$ and $\mathcal{P}$
        \end{itemize}
      \end{onlyenv}
      \begin{onlyenv}<9->
        \begin{equation*}
          \mathcal{S}\otimes\mathcal{P} = \tuple{
            S\times Q, \tuple{s_{0}, q_{0}}, A, \tuple{\tau_{a}'}_{a\in A},
            \sim'
          }
        \end{equation*}
        \begin{itemize}
          \item<10-> $\tau_{a}'(\tuple{s, q}, \tuple{z, p}) = \tau_{a}(s, z)
            \cdot \delta(p, s, q)$
          \item<11-> $\tuple{s, q}\sim'\tuple{z, p}$ iff $s\sim z$
          \item[$\Rightarrow$]<12-> strategies can be translated from
            $\mathcal{S}$ to $\mathcal{S}\otimes\mathcal{P}$
          \item[$\Rightarrow$]<13-> $\mu_{s}(\Acc_{\buechi}(S\times F)) =
            \int_{S^{\omega}}K(\cdot, \Acc_{\buechi}(F))d\mu_{s}$
          \item[$\Rightarrow$]<14-> almost-surely accepting strategy in
            $\mathcal{S}\otimes\mathcal{P}$ is qualitatively \enquote{good}
            for $\mathcal{S}$ with specification $\mathcal{P}$
        \end{itemize}
      \end{onlyenv}
    \end{overlayarea}
  \end{frame}

  \section{Conclusion}
  \begin{frame}
    \frametitle{Conclusion \& Ideas}
    \begin{itemize}
      \item Introduction of \acp{WDTA}
      \item Synthesis for antagonistic environment for almost-sure accepting
        \acp{PBA} in doubly exponential time
      \item Synthesis for \ac{POMDP} for almost-sure accepting \acp{PBA} in
        exponential time
      \item[*] Co-operative \ac{POSG} as emptiness games for unrestricted
        \acp{WDTA}
      \item[*] Strategy computation for \eve{} in equally informed \ac{POSG}
    \end{itemize}
  \end{frame}
\end{document}
