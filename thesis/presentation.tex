\documentclass{beamer}

\usetheme{metropolis}

% inputenc:
\usepackage[utf8]{inputenc}

\usepackage{xcolor}

% acronyms
\usepackage{acro}

\DeclareAcronym{MDP}{
  short = MDP ,
  long  = Markov Decision Process ,
  class = environment ,
  long-plural  = es
}
\DeclareAcronym{POMDP}{
  short = POMDP,
  long = Partially Observable Markov Decision Process ,
  class = environment ,
  long-plural = es
}
\DeclareAcronym{MC}{
  short = MC ,
  long  = Markov Chain ,
  class = environment ,
  long-plural  = es
}

\DeclareAcronym{BTA}{
  short = BTA ,
  long  = Büchi Tree Automaton ,
  class = automaton ,
  long-plural-form = Büchi Tree Automaton 
}
\DeclareAcronym{RTA}{
  short = RTA ,
  long  = Rabin Tree Automaton ,
  class = automaton ,
  long-plural-form = Rabin Tree Automata 
}
\DeclareAcronym{MTA}{
  short = MTA ,
  long  = Muller Tree Automaton ,
  class = automaton ,
  long-plural-form = Muller Tree Automata 
}
\DeclareAcronym{PTA}{
  short = PTA ,
  long  = Parity Tree Automaton ,
  class = automaton ,
  long-plural-form = Parity Tree Automata
}

\DeclareAcronym{ABTA}{
  short = ABTA ,
  long  = Alternating Büchi Tree Automaton ,
  class = automaton ,
  long-plural-form = Alternating Büchi Tree Automata
}
\DeclareAcronym{ARTA}{
  short = ARTA ,
  long  = Alternating Rabin Tree Automaton ,
  class = automaton ,
  long-plural-form = Alternating Rabin Tree Automata
}
\DeclareAcronym{AMTA}{
  short = AMTA ,
  long  = Alternating Muller Tree Automaton ,
  class = automaton ,
  long-plural-form = Alternating Muller Tree Automata
}
\DeclareAcronym{APTA}{
  short = APTA ,
  long  = Alternating Parity Tree Automaton ,
  class = automaton ,
  long-plural-form = Alternating Parity Tree Automata
}

\DeclareAcronym{WDTA}{
  short = WDTA ,
  long  = Weighted Descent Tree Automaton ,
  class = automaton ,
  long-plural-form = Weighted Descent Tree Automata
}
\DeclareAcronym{PWA}{
  short = PWA ,
  long  = Probabilistic Weighted Automaton ,
  class = automaton ,
  long-plural-form = Probabilistic Weighted Automata
}

\DeclareAcronym{PBA}{
  short = PBA ,
  long  = Probabilistic Büchi Automaton ,
  class = automaton ,
  long-plural-form = Probabilistic Büchi Automata
}
\DeclareAcronym{PRA}{
  short = PRA ,
  long  = Probabilistic Rabin Automaton ,
  class = automaton ,
  long-plural-form = Probabilistic Rabin Automata
}
\DeclareAcronym{PMA}{
  short = PMA ,
  long  = Probabilistic Muller Automaton ,
  class = automaton ,
  long-plural-form = Probabilistic Muller Automata
}
\DeclareAcronym{PPA}{
  short = PPA ,
  long  = Probabilistic Parity Automaton ,
  class = automaton ,
  long-plural-form = Probabilistic Parity Automata
}

\DeclareAcronym{DBA}{
  short = DBA ,
  long  = Deterministic Büchi Automaton ,
  class = automaton ,
  long-plural-form = Deterministic Büchi Automata
}
\DeclareAcronym{NBA}{
  short = NBA ,
  long  = Non-Deterministic Büchi Automaton ,
  class = automaton ,
  long-plural-form = Non-Deterministic Büchi Automata
}
\DeclareAcronym{DMA}{
  short = DMA ,
  long  = Deterministic Muller Automaton ,
  class = automaton ,
  long-plural-form = Deterministic Muller Automata
}
\DeclareAcronym{NMA}{
  short = NMA ,
  long  = Non-Deterministic Muller Automaton ,
  class = automaton ,
  long-plural-form = Non-Deterministic Muller Automata
}
\DeclareAcronym{DRA}{
  short = DRA ,
  long  = Deterministic Rabin Automaton ,
  class = automaton ,
  long-plural-form = Deterministic Rabin Automata
}
\DeclareAcronym{NRA}{
  short = NRA ,
  long  = Non-Deterministic Rabin Automaton ,
  class = automaton ,
  long-plural-form = Non-Deterministic Rabin Automata
}
\DeclareAcronym{DPA}{
  short = DPA ,
  long  = Deterministic Parity Automaton ,
  class = automaton ,
  long-plural-form = Deterministic Parity Automata
}
\DeclareAcronym{NPA}{
  short = NPA ,
  long  = Non-Deterministic Parity Automaton ,
  class = automaton ,
  long-plural-form = Non-Deterministic Parity Automata
}

\DeclareAcronym{LAR}{
  short = LAR ,
  long  = Latest Appearance Record ,
  class = auxilliary
}

\DeclareAcronym{SCC}{
  short = SCC ,
  long  = Strongly-Connected Component ,
  class = auxilliary
}

\DeclareAcronym{POSG}{
  short = POSG ,
  long  = Partial Observation Stochastic Game ,
  class = game
}


\usepackage{amssymb}
\usepackage{pifont}
\usepackage{amsmath}
\usepackage{amsthm}
\usepackage{bbm}

% math operators
\DeclareMathOperator{\cyl}{cyl}
\DeclareMathOperator{\plays}{Plays}
\DeclareMathOperator{\follow}{follow}
\DeclareMathOperator{\parity}{par}
\DeclareMathOperator{\supp}{support}
\DeclareMathOperator{\Inf}{Inf}
\DeclareMathOperator{\prolong}{prolong}
\DeclareMathOperator{\cobuechi}{co-B\ddot{u}chi}
\DeclareMathOperator{\buechi}{B\ddot{u}chi}
\DeclareMathOperator{\muller}{Muller}
\DeclareMathOperator{\reach}{Reachability}
\DeclareMathOperator{\safety}{Safety}
\DeclareMathOperator{\lar}{LAR}
\DeclareMathOperator{\perm}{Perm}
\DeclareMathOperator{\up}{update}
\DeclareMathOperator{\Syn}{Syn}
\DeclareMathOperator{\compRuns}{compRuns}
\DeclareMathOperator{\Runs}{Runs}
\DeclareMathOperator{\Paths}{Paths}
\DeclareMathOperator{\Acc}{Acc}
\DeclareMathOperator{\inner}{inner}
\DeclareMathOperator{\Pot}{Pot}
\DeclareMathOperator{\upK}{updateKnowledge}
\DeclareMathOperator{\currK}{currentKnowledge}
\DeclareMathOperator{\lift}{lift}
\DeclareMathOperator{\obs}{obs}
\DeclareMathOperator{\inpout}{io}
\DeclareMathOperator{\io}{io}

\newcommand{\tuple}[1]{\left(#1\right)}
\newcommand{\set}[1]{\left\{#1\right\}}
\newcommand{\size}[1]{\left\vert{#1}\right\vert}
\newcommand{\interval}[1]{\left[#1\right]}



\title[Syn. for prob. Env.]{
  Automata-theoretic Synthesis for Probabilistic Environments}

\titlegraphic{
  \resizebox{0.9\textwidth}{!}{
    \includegraphics{tikz/titlepic.pdf}
  }
}

\date{\today}
\author{Christoph Welzel}
\institute{Informatik 7, RWTH Aachen}

\metroset{
  sectionpage = simple,
  numbering = fraction
}

\begin{document}

  \acuseall

  \maketitle

  \section{Word-Automata}
  \begin{frame}
    \frametitle{$\omega$-regular Languages}
    \begin{theorem}
      The class of recognizable languages coincides for \alert<2>{\acp{NBA}},
      \alert<3>{\acp{NPA} and \acp{DPA}} and is called $\omega$-regular
      languages. \alert<4>{\acp{DBA} are strictly less expressive.}
      \uncover<3->{\alert<3>{However, \ac{NBA}-determinisation is inherently
      costly ($>n!$).}}
    \end{theorem}
    \begin{overlayarea}{\textwidth}{0.5\textheight}
        \begin{equation*}
          \set{
            \mathcal{L} = \alpha\in\set{a,b}^{\omega}\mid a\notin\Inf(\alpha)
          }
        \end{equation*}
        \begin{columns}
          \begin{column}{0.3\textwidth}<2->
            \includegraphics{tikz/finabuechi.pdf}
          \end{column}
          \begin{column}{0.3\textwidth}<3->
            \includegraphics{tikz/finaparity.pdf}
          \end{column}
          \begin{column}{0.3\textwidth}<4->
            \begin{align*}
              &b^{\omega},\\
              &b^{n_{0}}ab^{\omega},\\
              &b^{n_{0}}ab^{n_{1}}ab^{\omega},\\
              &\vdots
            \end{align*}
          \end{column}
        \end{columns}
    \end{overlayarea}
  \end{frame}

  \begin{frame}
    \frametitle{Probabilistic Automata}
    \begin{equation*}
      \mathcal{A} = \tuple{Q, \Sigma, 
      \delta:Q\times\Sigma\times Q\rightarrow\interval{0,1}, q_{0}, F}
    \end{equation*}
    \begin{overlayarea}{\textwidth}{0.7\textheight}
      \begin{onlyenv}<1>
        \begin{itemize}
          \item $Q$ : finite state set,
          \item $q_{0}\in Q$: initial state,
          \item $\delta:Q\times\Sigma\times Q\rightarrow \interval{0,1}$: 
            transition probability function
          \item $F\subseteq Q$: final states.
        \end{itemize}
      \end{onlyenv}
      \begin{onlyenv}<2>
        \begin{center}
          \resizebox{!}{0.7\textheight}{%
            \includegraphics{tikz/pbaruns.pdf}
          }
        \end{center}
      \end{onlyenv}
      \begin{onlyenv}<3->
        \begin{columns}
          \begin{column}{0.5\textwidth}
            \begin{itemize}
              \item<3-> cylindric sets: $\cyl(u) = 
                \set{u\cdot\alpha:\alpha\in Q^{\omega}}$
            \end{itemize}
            \begin{center}
              \resizebox{0.8\textwidth}{!}{%
                \includegraphics{tikz/verticalcylinder.pdf}
              }
            \end{center}
          \end{column}
          \begin{column}{0.5\textwidth}
            \begin{itemize}
              \item<4-> $\alpha$ induces probability space 
                \begin{equation*}
                  \tuple{Q, \mathcal{B}(Q), \mu_{\alpha}}
                \end{equation*}
              \item<5-> positive: $\mu_{\alpha}(\Acc_{\buechi}(F)) > 0$
              \item<6-> almost-sure: $\mu_{\alpha}(\Acc_{\buechi}(F)) = 1$
            \end{itemize}
          \end{column}
        \end{columns}
      \end{onlyenv}
    \end{overlayarea}
  \end{frame}

  \begin{frame}
    \frametitle{Probabilistic Automata - Examples}
    \begin{columns}
      \begin{column}{0.5\textwidth}
        \small{Positive Acceptance}
        \begin{center}
          \resizebox{0.6\textwidth}{!}{%
            \includegraphics{tikz/posacceptingpba.pdf}
          }
        \end{center}
        \begin{equation*}
          \underbrace{
              \set{a^{k_{1}}ba^{k_{2}}b\dots:
              \substack{k_{i}>0\text{ for all }i>0,\\
              \prod_{i>0}\tuple{1-\frac{1}{2}^{k_{i}}} > 0}
            }
          }_{\mathcal{L}_{1}}
        \end{equation*}
      \end{column}
      \begin{column}<2->{0.5\textwidth}
        \small{Almost-Sure Acceptance}
        \begin{center}
          \resizebox{0.6\textwidth}{!}{%
            \includegraphics{tikz/almostsureacceptingpbapresentation.pdf}
          }
        \end{center}
        \begin{equation*}
          \underbrace{
            \set{
              a^{k_{1}}ba^{k_{2}}b\dots:
              \substack{k_{i}>0\text{ for all }i>0\\
              \prod_{i>0}\tuple{1-\frac{1}{2}^{k_{i}}} = 0}
            }
          }_{\mathcal{L}_{2}}
        \end{equation*}
      \end{column}
    \end{columns}
    \begin{uncoverenv}<3->
      \begin{equation*}
        \mathcal{L}_{1} = \overline{\mathcal{L}_{2}}\setminus\underbrace{
          b\Sigma^{\omega} + \Sigma^{*}bb\Sigma^{\omega} + \Sigma^{*}a^{\omega}
        }_{\omega-\text{regular}}
      \end{equation*}
    \end{uncoverenv}
  \end{frame}

  \begin{frame}
    \frametitle{Probabilistic Automata - Properties}
    \begin{columns}
      \begin{column}{0.5\textwidth}
        \small{Positive Acceptance}
        \begin{itemize}
          \item<2-> subsumes $\omega$-regular
          \item<5-> recognizable languages form Boolean-algebra
          \item<6-> undecidable emptiness
        \end{itemize}
        \small{Almost-Sure Acceptance}
        \begin{itemize}
          \item<7-> incomparable with $\omega$-regular
          \item<10-> decidable emptiness
          \item<11-> \alert{Parity-}condition coincides with positive
            acceptance of Büchi- or Parity-condition
        \end{itemize}
      \end{column}
      \begin{column}{0.5\textwidth}
        \begin{overlayarea}{\textwidth}{\textheight}
          \begin{onlyenv}<3-4>
            \vspace{0.3\textheight}
            deterministic in limit \ac{NBA} with uniform probability measure
            \vfill
            \begin{uncoverenv}<4>
              \begin{equation*}
                \set{
                  a^{k_{1}}ba^{k_{2}}b\dots:
                  \substack{k_{i}>0\text{ for all }i>0,\\
                  \prod_{i>0}\tuple{1-\frac{1}{2}^{k_{i}}} > 0}
                }
              \end{equation*}
            \end{uncoverenv}
          \end{onlyenv}
          \begin{onlyenv}<8-9>
            \vspace{0.2\textheight}
            \begin{equation*}
              \set{\alpha\in\set{a,b}^{\omega}\mid a\notin\Inf(\alpha)}
            \end{equation*}
            \vfill
            \begin{uncoverenv}<9>
              \begin{align*}
                &\overline{\set{
                  a^{k_{1}}ba^{k_{2}}b\dots:
                  \substack{k_{i}>0\text{ for all }i>0,\\
                  \prod_{i>0}\tuple{1-\frac{1}{2}^{k_{i}}} > 0}
                }}\\
                &= \set{
                  a^{k_{1}}ba^{k_{2}}b\dots:
                  \substack{k_{i}>0\text{ for all }i>0\\
                  \prod_{i>0}\tuple{1-\frac{1}{2}^{k_{i}}} = 0}
                }\\
                &\cup b\Sigma^{\omega} + \Sigma^{*}bb\Sigma^{\omega}
                + \Sigma^{*}a^{\omega}\\
              \end{align*}
            \end{uncoverenv}
          \end{onlyenv}
        \end{overlayarea}
      \end{column}
    \end{columns}
  \end{frame}

  \section{Tree-Automata}
  \begin{frame}
    \frametitle{Definition}
    \begin{equation*}
      \mathcal{A} = \tuple{Q, q_{0}, D, \Sigma, \Delta,
        \Acc\subseteq Q^{\omega}}
    \end{equation*}
    \begin{overlayarea}{\textwidth}{0.8\textheight}
      \begin{onlyenv}<1>
        \begin{itemize}
          \item $Q$: finite state set
          \item $q_{0}\in Q$: initial state
          \item $D$: finite set of directions
          \item $\Sigma$: finite alphabet
          \item $\Delta$: transitions of the form
              $\tuple{q, \sigma, \tuple{q_{d}}_{d\in D}}$
          \item $\Acc$: accepted language
        \end{itemize}
      \end{onlyenv}
      \begin{onlyenv}<2-3>
        $D$-ary $\Sigma$-tree: $t:D^{*}\rightarrow\Sigma$\hfill
        \uncover<3>{$D$-ary $Q$-run: $r:D^{*}\rightarrow Q$}
        \begin{center}
          \alt<2>{
            \includegraphics{tikz/tree.pdf}
          }{
            \includegraphics{tikz/run.pdf}
          }
        \end{center}
      \end{onlyenv}
    \end{overlayarea}
  \end{frame}

  \begin{frame}
    \frametitle{Tree Automata - Example}
    \begin{equation*}
      \mathcal{L} = \set{
        t:\set{0,1}^{*}\rightarrow\set{a,b}\mid\substack{
          a\notin \Inf(t(\epsilon)t(\alpha_{1})t(\alpha_{1}\alpha_{2})\dots)\\
          \text{for all }\alpha_{1}\alpha_{2}\dots\in\set{0,1}^{\omega} 
        }
      }
    \end{equation*}
      \begin{columns}
        \begin{column}{0.4\textwidth}
          \vspace{-3cm}
          \begin{itemize}
            \item \alert<2-3>{Parity-condition} \uncover<3->{\checkmark}
            \item \alert<4->{Büchi-condition} \uncover<5->{\ding{55}}
          \end{itemize}
        \end{column}
        \begin{column}{0.6\textwidth}
          \begin{overlayarea}{\textwidth}{0.8\textheight}
            \begin{onlyenv}<3>
              \begin{align*}
                \mathcal{A} = (\set{q_{a}, q_{b}}, q_{a}, \set{0,1}, \set{a, b}&,\\
                \Delta, \set{q_{a}\mapsto 1, q_{b}\mapsto 0}&)
              \end{align*}
              with
              \begin{equation*}
                \Delta = \set{
                  \tuple{q, \sigma, q_{\sigma}, q_{\sigma}}:\substack{
                    q\in Q,\\
                    \sigma\in\set{a,b}
                  }
                }
              \end{equation*}
            \end{onlyenv}
            \begin{onlyenv}<4->
              \begin{center}
                % \resizebox{0.6\textwidth}{!}{%
                  \includegraphics{tikz/finatree.pdf}
                % }
              \end{center}
            \end{onlyenv}
          \end{overlayarea}
        \end{column}
      \end{columns}
  \end{frame}

  \begin{frame}
    \frametitle{Definition Alternating Tree Automata}
    \begin{equation*}
      \mathcal{A} = \tuple{Q, q_{0}, D, \Sigma, \Delta,
        \Acc\subseteq Q^{\omega}}
    \end{equation*}
    \begin{overlayarea}{\textwidth}{0.8\textheight}
      \begin{onlyenv}<2-3>
        $\set{0,1}$-ary $\Sigma$-tree $t$\hfill\uncover<3->{
          $Q\times D$-ary $\set{0,1}$-run $r$}
        \begin{center}
          \alt<2>{
            \includegraphics{tikz/bintree.pdf}
          }{
            \includegraphics{tikz/alternatingrun.pdf}
          }
        \end{center}
      \end{onlyenv}
    \end{overlayarea}
  \end{frame}

  \begin{frame}
    \frametitle{Alternating Tree Automaton - Example}
    \begin{equation*}
      \set{t:\set{0,1}^{*}\rightarrow\set{a,b}\mid\substack{
        \text{there is }u\in\set{0,1}^{*}\\
        \text{ with }t(u) = a}
      }
    \end{equation*}
    \begin{columns}
      \begin{column}{0.5\textwidth}
        \begin{itemize}
          \item search and found state: $q_{s}, q_{f}$
          \item<2-> non-deterministically move $q_{s}$
          \item<3-> transform into positive sink-state $q_{f}$ when reading $a$
          \item<4-> use $\set{q_{f}}$ as Büchi-condition
        \end{itemize}
      \end{column}
      \begin{column}{0.5\textwidth}
        \begin{center}
          \uncover<5>{\includegraphics{tikz/searcharun.pdf}}
        \end{center}
      \end{column}
    \end{columns}
  \end{frame}

  \begin{frame}
    \frametitle{Tree Automata - Properties}
    \begin{theorem}[Simulation Theorem]
      There is an effective construction which, when given an \ac{APTA},
      produces an equivalent \ac{PTA}. Furthermore, given an \ac{ABTA}, there
      is a way to effectively construct an equivalent \ac{BTA}.
    \end{theorem}
    \begin{itemize}
      \item recognizable languages form a Boolean-algebra
      \item decidable emptiness
    \end{itemize}
  \end{frame}

  \begin{frame}
    \frametitle{Weighted Descent Tree Automata}
  \end{frame}

  \section{Synthesis}
  \begin{frame}
  \end{frame}

  \begin{frame}
  \end{frame}

  \section{Conclusion}
  \begin{frame}
  \end{frame}

  \begin{frame}
  \end{frame}
\end{document}
